\documentclass{beamer}

% Prepare for svenska tecken
\usepackage[T1]{fontenc}
\usepackage[utf8]{inputenc}
\usepackage[swedish]{babel}
%\usepackage[]{geometry}
\addto\captionsswedish{\renewcommand{\figurename}{Bild}}
\usepackage{amsmath}
\usepackage{fancyhdr}
\usepackage{wrapfig}
\usepackage{caption}
\usepackage{framed}
\usepackage[fulladjust]{marginnote}
\usepackage{color}
\newcommand{\hilight}[1]{\colorbox{yellow}{#1}}
\usepackage{hyperref}
\hypersetup{
    colorlinks,
    citecolor=black,
    filecolor=black,
    linkcolor=black,
    urlcolor=black
}
\usepackage{stmaryrd} % För symbolen \boxbox, kräver paketet texlive-math-extra

% % % % % % % % % % % 
% Detta är nya environments för review. De bör vara relativt självförklarande hur de används.
% I princip sätter man bara den del av texten som har en viss status mellan\begin{rev-granskat} och \end{rev-granskat} tex.
% Undvik att nästla dem för det är ingen idé det fungerar inte.
% De är testade med ett antal andra environemnt som tabular mm men kolla att det fungerar med de environments du använder.
% % % % % % % % % % % % % % % % % % % % % % % % % % % % % % % % % % % % % % % % % % % % % % % % % % % % % % % % % % % % % % % 
%\usepackage[svgnames,rgb]{xcolor}
%\usepackage{pdfcomment}
%\newenvironment{rev-ogranskat}{\begin{pdfsidelinecomment}[color=black,linewidth=3px,caption=inline]{Ogranskat}}{\end{pdfsidelinecomment}}
%\newenvironment{rev-omarbetas}{\begin{pdfsidelinecomment}[color=red,linewidth=3px,caption=inline]{Omarbetas}}{\end{pdfsidelinecomment}}
%\newenvironment{rev-raderas}{\begin{pdfsidelinecomment}[color=red,linewidth=3px,caption=inline]{Raderas}}{\end{pdfsidelinecomment}}
%\newenvironment{rev-redo}{\begin{pdfsidelinecomment}[color=yellow,linewidth=3px,caption=inline]{Redo att granska}}{\end{pdfsidelinecomment}}
%\newenvironment{rev-granskat}[1][]%
%{\begin{pdfsidelinecomment}[color=green,linewidth=3px,caption=inline]%
%{Granskat #1}}%
%{\end{pdfsidelinecomment}}
%\newenvironment{rev-nytt}[1][]%
%{\begin{pdfsidelinecomment}[color=brown,linewidth=3px,caption=inline]%
%{Nytt #1}}%
%{\end{pdfsidelinecomment}}
%\newenvironment{rev-releasat}{\begin{pdfsidelinecomment}[color=blue,linewidth=3px,caption=inline]{Klart}}{\end{pdfsidelinecomment}}

%\clubpenalty=9990
%\widowpenalty=9999
%\brokenpenalty=4999

\usepackage[europeanvoltages,europeancurrents,europeanresistors,cuteinductors,smartlabels]{circuitikz}
\usepackage[framemethod=TikZ]{mdframed}

\mdfdefinestyle{FactBox}{%
    linecolor=blue,
    outerlinewidth=2pt,
    roundcorner=20pt,
    innertopmargin=\baselineskip,
    innerbottommargin=\baselineskip,
    innerrightmargin=20pt,
    innerleftmargin=20pt,
    backgroundcolor=gray!50!white}
\newcommand{\infobox}[1]{
\begin{wrapfigure}{r}{0.5\textwidth}
  \begin{mdframed}[style=FactBox]
#1
  \end{mdframed}
\end{wrapfigure}
}

% Make some unicode characters usable
%\DeclareUnicodeCharacter{00B0}{\ensuremath{^\circ}} % unicode 00B0 ° degree sign
%\DeclareUnicodeCharacter{00B5}{\ensuremath{\mu}} % unicode 00B5 µ micro sign
%\DeclareUnicodeCharacter{03C0}{\ensuremath{\pi}} % unicode 3C0 π greek small letter pi
%\DeclareUnicodeCharacter{03A9}{\ensuremath{\Omega}} % unicode 3A9 Ω greek capital letter omega
%\DeclareUnicodeCharacter{2206}{\ensuremath{\Delta}} % unicode 2206 ∆ increment


% Prepare for tables
\usepackage{multirow}
\usepackage{xtab}

% Prepare for lists
%\usepackage{enumitem}

% Prepare for graphics
\usepackage{graphicx}

\raggedbottom


%% Frontpage bacground
\usepackage{eso-pic}
\newcommand\BackgroundPic{%
\put(0,0){%
\parbox[b][\paperheight]{\paperwidth}{%
\vfill
\centering
\includegraphics[width=\paperwidth,height=\paperheight,%
keepaspectratio]{images/koncept-front.jpg}%
\vfill
}}}


\newcommand\BackgroundPicLast{%
\put(0,0){%
\parbox[b][\paperheight]{\paperwidth}{%
\vfill
\centering
\includegraphics[width=\paperwidth,height=\paperheight,%
keepaspectratio]{images/koncept-back.pdf}%
\vfill
}}}

\usepackage[swedish]{babel}
\usepackage{lmodern}
\mode<presentation>
{
\usetheme{Madrid}
\usecolortheme{default}
\usefonttheme{serif}
\setbeamertemplate{navigation symbols}{}
\setbeamertemplate{caption}[numbered]
}
\usepackage{tikz}
\usepackage{amsmath,amsfonts,amssymb,bm,mathrsfs}
\usetikzlibrary{shapes,arrows}
\usepackage{siunitx}
\sisetup{
	output-decimal-marker = {,},
	per-mode=symbol,
	range-units=single,
	range-phrase={--}
}

\title[SM7NTJ]{Elsäkerhet}
\author{SSA utbildning}

\begin{document}
	
	\begin{frame}
		\titlepage
		\includegraphics[height=0.3\textheight]{images/ssalogo}
	\end{frame}

\begin{frame}{Innehåll}
\tableofcontents
\end{frame}

\section{Allmänt om elsäkerhet}

\begin{frame}{Det grundläggande}

I Sverige har elnätet frekvensen \SI{50}{\hertz} och spänningen \SI{230/400}{\volt}.

En person som har auktorisation som elinstallatör eller är yrkesverksam som
elektriker och omfattas av ett elinstallationsföretags egenkontroll får ändra,
reparera eller bygga en ny starkströmsanläggning.\\
\vspace{5mm}
Vissa elarbeten får göras av den som vet hur man gör:
\end{frame}

\begin{frame}{Gör elarbete själv}
\begin{itemize}
	\item byta ut en elkopplare (strömbrytare) för högst \SI{16}{\ampere} \SI{400}{\volt}
	\item byta ut ett anslutningsdon (vägguttag, lamputtag, stickpropp,
	skarvuttag eller liknande) för högst \SI{16}{\ampere} \SI{400}{\volt}
	\item byta säkring
	\item byta ljuskälla (lampa, lysrör eller liknande)
	\item reparera apparater
	\item byta ut en ljusarmatur i torrt icke brandfarligt utrymme i bostäder
	\item reparera och tillverka apparatkablar och skarvsladdar.
	\item utföra, ändra eller reparera en starkströmsanläggning som ingår i en
	skyddsklenspänningskrets med nominell spänning om högst \SI{50}{\volt} med effekt
	om högst 200\,VA och ström begränsad av säkring på högst \SI{10}{\ampere}.
\end{itemize}
\end{frame}

\begin{frame}{För allt annat elarbete}
\textbf{Kom ihåg, att auktoriserad installatör ska anlitas för arbete
	i fasta installationer.}
\end{frame}

\section{Hembyggd elektronik}

\begin{frame}{Radioutrustningslagen}
Enligt \emph{radioutrustningslagen} SFS 2016:392 ska radioutrustning som släpps
ut eller tillhandahålls på marknaden inom EU ska vara konstruerad och tillverkad
så att den uppfyller föreskrivna krav, ha en EU-försäkran om överensstämmelse
och vara CE-märkt.
\end{frame}

\begin{frame}{Tillämpning}
Lagens tillämpningsområde och definitioner anger att lagen inte omfattar
radioutrustning som används av radioamatörer för amatörradiotrafik, under
förutsättning att utrustningen inte tillhandahålls på marknaden.
Radioutrustning som används av radioamatörer för amatörradiotrafik ska inte
anses tillhandahållen om det är:
\end{frame}

\begin{frame}{Undantagen}
\begin{itemize}
	\item radiobyggsatser som är avsedda att byggas samman och användas av
	radioamatörer
	\item radioutrustning som har modifierats av radioamatörer för att
	avvändas av radioamatörer
	\item utrustning som har konstruerats av enskilda radioamatörer för
	experimentella och vetenskapliga ändamål i samband med amatörradio.
\end{itemize}
\end{frame}

\begin{frame}{Du får}
Detta innebär att du som radioamatör, utöver vanlig elektronik, får bygga
och använda en radioutrustning.
Du är då ansvarig för att den utrustning du byggt är säker att använda och inte
orsakar störningar.
\end{frame}

\section{Elsäkerhet vid hembyggen}
\begin{frame}{Hembyggen}
När en elektrisk eller elektronisk apparat konstrueras eller byggs finns det
ett antal punkter som ska uppmärksammas för att apparaten ska vara säker att
använda oavsett hur den är avsedd att strömförsörjas.
Som stöd för hur en apparat kunde byggas för att uppfylla kraven gav
dåvarande SEMKO ut \emph{Praktiska råd för självbyggaren}.\\
\vspace{5mm}
Nedanstående punkter bygger på dessa praktiska råd:
\end{frame}

\begin{frame}{Praktiska råd sid 1/3}
\begin{itemize}
	\item Höljet ska vara anpassat till apparaten och inte vara öppningsbart
	utan verktyg.
	
	\item Höljet ska vara försett med nödvändiga ventilationshål för att
	undvika överhettning.
	Observera att spänningsförande delar inte får vara nåbara genom
	ventilationshålen.
	
	\item Höljet får inte bli så varmt att skada kan uppstå på människa
	eller egendom.
\end{itemize}	
\end{frame}

\begin{frame}{Praktiska råd sid 2/3}
\begin{itemize}
		\item Är höljet eller chassiet till en elnätsansluten apparat av ledande
	material och apparaten inte har förstärkt isolering så ska \emph{utsatta}
	delar som riskerar att spänningssättas vid fel anslutas till skyddsjord.
	
	\item Kabeln för nätanslutning ska vara försedd med en för ändamålet lämplig
	dragavlastning som även skyddar kabeln mot nötning när den passerar höljet.
	
	\item Komponenter i apparaten ska vara dimensionerade och godkända
	för den effekt de utvecklar och för den spänning och strömstyrka de
	ansluts till.
	\emph{Not: Ett tips är att ha god marginal vad gäller värmetålighet då det
		ger ökad livslängd och större säkerhetsmarginaler.}
\end{itemize}
\end{frame}

\begin{frame}{Praktiska råd sid 3/3}
\begin{itemize}
\item Apparaten ska vara försedd med korrekt dimensionerad säkring
som skydd mot kortslutning och överbelastning.

\item Elnätsansluten apparat ska vara försedd med 2-polig nätströmbrytare.

\item Spänningsförande delar i apparaten ska vara försedda med
beröringsskydd som skyddar mot oavsiktlig beröring.

\item Komponenter i apparaten ska monteras fast och placeras på lämpliga
inbördes avstånd så att risken för störningar, överslag, kortslutning eller
överhettning minimeras.

\item Kablar och ledningar för starkström ska skyddas mot varma komponenter,
nötning och skarpa kanter samt förläggas separerade från ledningar för
klenspänning och signaler.
\end{itemize}
\end{frame}

\begin{frame}{Använd jordfelsbrytare}
\textbf{Sträva efter att alltid ansluta din apparat via vägguttag
	skyddade av jordfelsbrytare.}
\end{frame}

\begin{frame}{Ej tillåtna hembyggen}
Som radioamatör får du inte:
\begin{itemize}
	\item Bygga en sändare för användning utanför amatörradiobanden.
	\item Modifiera en amatörradiosändare för användning utanför amatörradiobanden
	\item Modifiera en CE-märkt sändare utanför amatörradiobanden.
	\item Återställa en CE-märkt sändare till ursprunget efter modifiering till
	amatörradiosändare på amatörradiobanden.
\end{itemize}
\end{frame}

\section{Skyddsåtgärder}

\begin{frame}{Förstärkt isolering}
Vissa bruksföremål tillverkas med en extra isolering som gör att de inte behöver
skyddsjordas.
De är märkta med Fi-märket och får inte ändras så att de kan skyddsjordas.
Anslutningsledningen till extra isolerade bruksföremålet har en speciell
stickpropp som passar i vägguttag såväl med som utan jorddon.
\end{frame}

\begin{frame}{Skyddsjordning}
Metallhöljen på elektrisk utrustning kan av olika anledningar bli
spänningsförande och är då en elsäkerhetsrisk.
För att minska risken för farlig spänningssättning av metallhöljet ansluts
höljet till skyddsjord.\\
\vspace{5mm}
Skyddsjordning görs med en ledning med gul/grön färgmärkning.
En gul/grön ledning får endast användas för skyddsjordning.
\end{frame}

\begin{frame}{Skyddsjordade uttag}
Vid nybyggnation är alla uttag av skyddsjordat utförande och det rekommenderas
att låta en auktoriserad installatör installera skyddsjordade vägguttag för
radiostationen i äldre anläggningar.\\
\vspace{5mm}
Observera då, att alla uttag i det rummet ska vara skyddsjordade!
\end{frame}

\begin{frame}{Rätt skydd i varje miljö}
Skyddsjordade bruksföremål ska anslutas till skyddsjordade vägguttag när de
används i miljöer med högre risk, till exempel utomhus.\\
\vspace{5mm}
Var även uppmärksam på bruksföremålets kapslingsklass som visar om bruksföremålet är
godkänt för att användas utomhus.
\end{frame}

\begin{frame}{Jordfelsbrytare}
Jordfelsbrytare är en automatisk strömbrytare som snabbt bryter strömmen
då strömmen till och från en apparat är olika.
Detta kan inträffa vid ett jordfel eller vid överledning i en skyddsjordad
apparat eller i andra fall när inkommande ström och utgående ström genom
jordfelsbrytaren inte är lika stora.\\
\vspace{5mm}
Jordfelsbrytaren kan skydda dig:
\end{frame}

\begin{frame}{Jordfelsbrytaren kan skydda dig 1/2}
\begin{itemize}
	\item vid isolations- och jordfel
	\item om chassiet på en apparat blir strömförande
	\item om du kommer åt spänningsförande delar och jord samtidigt
	\item om vägguttagen saknar skyddsjord
\end{itemize}
\end{frame}

\begin{frame}{Jordfelsbrytaren kan skydda dig 2/2}
\begin{itemize}
	\item om du använder en apparat på ett felaktigt sätt i våtutrymmen
	\item om du installerat en apparat på att felaktigt sätt
	\item om apparatens kabel skadats
	\item mot och minimera risken för brand.
\end{itemize}
\end{frame}

\begin{frame}{Inget skydd av jordfelsbrytare}
Jordfelsbrytaren \textbf{skyddar inte} för strömmar som går genom fasledare
och neutralledare eller genom fas till fasledare (3-fas).\\
\vspace{5mm}
Kontakta en auktoriserad installatör för installation av jordfelsbrytare i äldre
anläggningar!
\end{frame}

\section{Säkerhetsåtgärder}
\begin{frame}{Skyddstransformator}
Transformator med förstärkt säkerhet:

\begin{center}
	\begin{minipage}{0.19\columnwidth}
		\Huge{\fontencoding{U}\fontfamily{futs}\selectfont\char 66\relax}
	\end{minipage}
	\begin{minipage}{0.7\columnwidth}
		Om du är osäker på det elsäkerhetsmässiga utförandet på en
		apparat, till exempel en gammal sändare, använd då en skiljetransformator
		(fulltransformator) -- helst av klass II (extraisolerad).
	\end{minipage}
\end{center}
\end{frame}

\begin{frame}{Reparation utan spänning}
Vid reparation ska utrustningen vara spänningslös.
Före arbetet ska du

\begin{enumerate}
	\item stänga av utrustningens nätströmbrytare
	\item dra ur stickproppen ur vägguttaget (dubbel säkerhet).
\end{enumerate}
\end{frame}

\begin{frame}{Felsökning med spänning sid 1/2}
Om trimning eller felsökning måste ske under spänning ska följande iakttas:
\begin{itemize}
	\item Arbeta inte med anläggningen när du är trött eller omotiverad.
	Då är du minst vaksam mot olyckor.
	\item Se till att du inte får ström genom kroppen, arbeta helst bara med en
	hand och håll den andra borta från den utrustning som du arbetar med.
	Stoppa gärna den fria handen i fickan!
\end{itemize}
\end{frame}

\begin{frame}{Felsökning med spänning sid 2/2}
\begin{itemize}
	\item Ha inga hörtelefoner på huvudet.
	Använd högtalare om du trimmar med hörseln.
	\item Helst bör någon finnas i närheten när du arbetar i apparater under
	spänning.
	Visa var nätströmbrytaren sitter.
	Se gärna till att han/hon kan elolycksfallshjälp.
\end{itemize}
\end{frame}

\begin{frame}{Ackumulatorbatterier sid 1/2}
Vid arbete med ackumulatorbatterier:

\begin{itemize}
	\item Trots att spänningen är låg kan ackumulatorbatterier lämna
	mycket höga strömmar vid kortslutning.
	Tag därför av fingerringar, armbandsur med mera.
	
	Använd isolerade verktyg vid arbete med batteripolskor.
	\item Akta dig för elektrolyten i ackumulatorbatterierna -- den är
	starkt frätande.
\end{itemize}
\end{frame}

\begin{frame}{Ackumulatorbatterier sid 2/2}
\begin{itemize}
	\item Varning för explosionsrisk av knallgas och syrastänk i ögonen.
	\item Moderna litium-polymer (LiPo) batterier är oerhört energirika.
	Dessa kan börja brinna med hög temperatur, och bör behandlas varsamt samt
	läggas i därför lämpliga skyddspåsar.
	Olika varianter är olika känsliga, så det är rekommenderat att läsa på om
	de alternativ som finns och hur bäst hantera dem.
	Akta för överladdning!
\end{itemize}
\end{frame}

\begin{frame}{Antenner sid 1/5}
Placera helst antennerna utom räckhåll för obehöriga.
På sändarantenner kan det nämligen uppstå höga HF-spänningar redan vid
låg sändareffekt.
HF bränns vid beröring och en reflexrörelse gör det lätt att tappa balansen och
falla.
Sätt gärna upp skyltar på eller invid antennerna, med varning för högfrekvent
spänning samt uppgift om ägarens namn, adress och telefonnummer.
\end{frame}

\begin{frame}{Antenner sid 2/5}
En obalanserad antenn kan resultera i stor spänning även på ansluten
antennkabel.
Att röra antennkabeln kan därför innebära samma risker som att ta i själva
antennen.
T-antennen är en antenn som är konstruerad för att utnyttja denna obalans då
själva antennkabeln är del av antennens utstrålande delar.
För de flesta andra antenner så ska antennkabeln inte stråla och därför ska inte
beröring innebära fara.
\end{frame}

\begin{frame}{Antenner sid 3/5}
Antenner får inte korsa eller placeras nära hög\-spännings-, låg\-spännings- eller
telefonlinjer.
Det är en olycksrisk om antenner och kraft- eller teleledningar av någon
anledning slår ihop.
Det är också en olycksrisk om antenner faller ner över dessa ledningar.\\
\vspace{5mm}
Endast efter tillstånd från berörd myndighet och eller linjeägare får
man dra ledningar av något slag över väg eller offentlig plats.
\end{frame}

\begin{frame}{Antenner sid 4/5}
Höga likspänningar från sändaren får inte komma ut i antennen.
Se till att antennernas matarledningar är kopplade till god likströmsjord
via HF-drosslar eller försedda med överspänningsavledare.
Som extra säkerhetsåtgärd bör sändaren anslutas till antennledningen över en
stor kondensator.
Undvik att beröra antenner utan att de jordats, särskilt vid vistelse
på tak eller i träd.
\end{frame}

\begin{frame}{Antenner sid 5/5}
Under åskväder, snöfall, regn eller dimma då laddade partiklar är i
rörelse, kan antennerna laddas upp till höga statiska spänningar.
Arbetar man då med antennen kan man överraskas av en elektrisk stöt.
Det är då lätt hänt att tappa taget och falla ner.
\end{frame}

\begin{frame}{Jordning av antenner}
I brist på annan jordpunkt är det frestande att ansluta antennjordledaren till
PE-ledarens anslutningsbleck i vägguttaget eller till ett värmeelement med
förhoppning att på så sätt få ett bättre HF-jordplan för antennen.
Detta är emellertid ett dåligt exempel på särjordning, som både kan innebära
säkerhetsrisker och medföra störningsproblem.
\end{frame}

\begin{frame}{Säkringar}
Det finns snabba och tröga säkringar.
Snabba säkringar är det som normalt används.
Tröga säkringar för samma strömstyrka kan behövas för apparater som har
speciellt hög startström, till exempel stora nättransformatorer med toroidkärna.\\
\vspace{2mm}
Säkringarna ska kunna bryta tillräcklig hög spänning, annars blir det
en kvarstående ljusbåge i dem vid säkringsbrott.
Använd säkringar med rätta strömvärden och välj en säkring med lite marginal
till belastningsströmmen så att säkringen inte löser ut under normal drift.

Det är förbjudet att laga säkringar då det kan orsaka brand.

\end{frame}
\begin{frame}{Överhettning}
Nästan all elektrisk utrustning producerar värme.
För att motverka brandrisk måste värmen ledas bort.
Åtgärder för att leda bort värme får inte påverka den elektriska utrustningens
beröringsskydd.\\
\vspace{2mm}
Självbyggda apparater ska uppfylla kraven på elsäkerhet -- det vill säga
säkerhet mot elchock och brand -- och byggaren bär ensam ansvaret för att
utförandet och hanterandet av apparaterna är betryggande.
\end{frame}

\begin{frame}{Höga spänningar}
Öppna aldrig en apparat om spänningen är tillslagen.
Vid ingrepp till exempel i sändare, mottagare och kraftförsörjningsaggregat är
det lätt att utsätta sig för höga likspänningar.
I sändare med elektronrör förekommer spänningar i storleksordningen hundratals
till tusentals volt.\\
\vspace{2mm}
Observera att även apparater som drivs med batteri eller ackumulatorer kan
innehålla kretsar som omvandlar den låga spänningen till direkt livsfarlig hög
spänning.
\end{frame}
\begin{frame}{Höga spänningar}
Öppna aldrig en apparat om spänningen är tillslagen.
Vid ingrepp till exempel i sändare, mottagare och kraftförsörjningsaggregat är
det lätt att utsätta sig för höga likspänningar.
I sändare med elektronrör förekommer spänningar i storleksordningen hundratals
till tusentals volt.\\
\vspace{2mm}
Observera att även apparater som drivs med batteri eller ackumulatorer kan
innehålla kretsar som omvandlar den låga spänningen till direkt livsfarlig hög
spänning.
\end{frame}

\begin{frame}{Höga strömmar}
Höga spänningar är alltid farliga.
Det är däremot inte så känt att även låga spänningar kan vara det.
Ackumulatorer och anslutna apparater kan ge ifrån sig höga strömmar även om
spänningen är låg.
Oavsiktliga strömvägar till exempel kortslutning genom en klocka eller
fingerring kan medföra allvarliga brännskador.
\end{frame}

\begin{frame}{Restladdning i kondensatorer}
Kondensatorer kan behålla en betydande restladdning under många timmar
sedan kraften brutits.

\begin{itemize}
	\item Koppla urladdningsresistorer (bleeder) över filterkondensatorer,
	så att de laddas ur när matningen stängs av.
	Av säkerhetsskäl ska urladdningsresistorerna tåla fyra gånger så stor effekt
	som de själva förbrukar under drift.
	\item Varning: När du laddar ur en kondensator, kortslut den inte!
	Använd en urladdningsresistor som tål den effekt som utvecklas vid urladdning!
\end{itemize}
\end{frame}

\begin{frame}{Frågor}
	
	\begin{itemize}
		\item ?
		\item ?
		\item ?
	\end{itemize}
\end{frame}

\end{document}
