\section{Trafikförkortningar}
\label{trafikförkortningar}
\harec{b}{3.1}{3.1} --
%\harec{b}{3.2}{3.2}
%\harec{b}{3.3}{3.3}
%\harec{b}{3.4}{3.4}
%\harec{b}{3.5}{3.5}
%\harec{b}{3.6}{3.6}
%\harec{b}{3.7}{3.7}
%\harec{b}{3.8}{3.8}
%\harec{b}{3.9}{3.9}
%\harec{b}{3.10}{3.10}
%\harec{b}{3.11}{3.11}
\harec{b}{3.12}{3.12}

Utöver Q-koden och klartext används vid morsetelegrafering även andra
trafikförkortningar.
Eftersom det internationella radiospråket är engelska, är förkortningar av
engelska ord vanligast.
Förkortningar bör emellertid inte användas i onödan.
En ovan operatör vid motstationen kan då få svårt att förstå meddelandet.

\subsection{Urval för radioamatörer}

I CEPT-rekommendation T/R 61-02 nämns utöver Q-koden följande övriga
trafikförkortningar, som berör amatörradio.
Radioamatörerna använder i praktiken många fler trafikförkortningar än dessa.

I reglementsprovet för amatörradiocertifikat ingår frågor om
trafikförkortningar, se tabell \ref{tab:trafikforkortningar}.

\begin{table}
  \begin{tabular}{lll}
    Förkort- & & \\
    ning & Engelskt uttryck & Svensk betydelse \\
    \hline
    \textbf{BK} & break & avbryt(-a) (sändningen) \\
    \textbf{CQ} & ''seek you'' & allmänt anrop, till alla \\
    \textbf{CW} & continuous waves & telegrafi (A1A) \\
    \textbf{DE} & franska ''de'' & från ..... (anropssignal) \\
    \textbf{K}  & come & ''kom'' \\
    \textbf{MSG} & message & meddelande, telegram \\
    \textbf{PSE} & please & var god (att \dots) \\
    \textbf{R} & received & allt uppfattat, mottaget \\
    \textbf{RST} & readability, & läsbarhet \\
   & signal-strength, & signalstryka \\
   & tone-report & ton \\
    \textbf{RX} & receiver & mottagare \\
    \textbf{TX} & transmitter & sändare \\
    \textbf{UR} & your & din, ditt, dina, er \\
  \end{tabular}
\caption{Trafikförkortningar -- urval för radioamatörer}
\label{tab:trafikforkortningar}
\end{table}

Utöver ovanstående trafikförkortningar upptas i CEPT-rekommendationen
även följande bokstavskombinationer, vilka används i teleprintertrafik
i stället för motsvarande morsetecken, slagna utan teckenmellanrum.
(Strecket ovanför bokstäverna betecknar att det inte finns något mellanrum).

\begin{tabular}{lll}
  \textoverline{AR} & sluttecken & \(+\) \\
  \textoverline{VA} eller \textoverline{SK} & avslutningstecken & @ \\
\end{tabular}

Ett exempel på en avsnitt ur en amatörradiosändning, där
trafikförkortningar används särskilt flitigt:

''gm es tnx vy much om fer ur rprt. u are cmg in hr ufb. my tx is
.... and rx .... ant 3 el beam . condx hr gud mni dx stns hrd . wl nw
nil so tks es 73''

I klartext ser exemplet ut så här: ''good morning and thank you very
much Old Man for your report.
You are coming in here ultra fine business.
My transmitter is .....  and receiver .. ... antenna is a 3 element beam.
Conditions here are good many stations heard.
Well now nothing for you so thanks and kindest regards''
