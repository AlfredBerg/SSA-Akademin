\chapter{S-enheter och dB}
\label{s-enhet}

I kommunikationsradiomottagare brukar det nästan alltid finnas en anordning som
mäter och visar styrkan av mottagna signaler.

Eftersom spänningen från antennen in i mottagaren kan variera över ett stort
område, är det praktiskt att uttrycka styrkevärdena med en logaritmisk
måttenhet, så kallad S-enhet.

Signalspänningen mäts över en impedans av \SI{50}{\ohm}.

Eftersom S-enheten är logaritmisk, så motsvarar till exempel signalstyrkan S8
halva signalspänningen, det vill säga \SI{25}{\micro\volt} eller
\SI{-6}{\decibel} jämfört med S9.
Om halveringen fortsätts, fås att S0 (noll) motsvarar en signalstyrka av
\SI{0,1}{\micro\volt}.

I en kortvågsmottagare alstras det ett internt brus med en nivå av
åtminstone \SI{0,1}{\micro\volt}.
Detta brus blandas med den inkommande signalen.
En insignal med en styrka under brusnivån (S0) kommer alltså inte att
kunna höras.
Vid högre signalstyrkor än S9 anges styrkan som S9 +ett antal dB.
Det är då frågan om mycket starka signaler.

Följande tabell gäller för det ideala sambandet mellan S-enheter och
signalstyrkor över två alternativa brusnivåer.

\begin{table}[h!]
  \vspace{10ex}
	\centering
	\begin{tabular}{l|lll|lll}
    S-Meter  & \multicolumn{3}{c}{Under \SI{30}{\mega\hertz}} & \multicolumn{3}{c}{Över \SI{30}{\mega\hertz}} \\
    värde    & dBm & (U vid \SI{50}{\ohm}) & \si{dB\micro V} & dBm & (U vid \SI{50}{\ohm}) & \si{dB\micro V} \\
    \hline
    S9 +\SI{40}{\decibel} & \num{-33}  & \SI{5,0}{\milli\volt}  & 74        & \num{-53}  & \SI{500}{\micro\volt}  & 54  \\
    S9 +\SI{30}{\decibel} & \num{-43}  & \SI{1,6}{\milli\volt}  & 64        & \num{-63}  & \SI{160}{\micro\volt}  & 44  \\
    S9 +\SI{20}{\decibel} & \num{-53}  & \SI{500}{\micro\volt}  & 54        & \num{-73}  & \SI{50}{\micro\volt}   & 34  \\
    S9 +\SI{10}{\decibel} & \num{-63}  & \SI{160}{\micro\volt}  & 44        & \num{-83}  & \SI{16}{\micro\volt}   & 24  \\
    S9                    & \num{-73}  & \SI{50}{\micro\volt}   & 34        & \num{-93}  & \SI{5}{\micro\volt}    & 14  \\
    S8                    & \num{-79}  & \SI{25}{\micro\volt}   & 28        & \num{-99}  & \SI{2,5}{\micro\volt}  & 8   \\
    S7                    & \num{-85}  & \SI{12,6}{\micro\volt} & 22        & \num{-105} & \SI{1,26}{\micro\volt} & +2  \\
    S6                    & \num{-91}  & \SI{6,3}{\micro\volt}  & 16        & \num{-111} & \SI{0,63}{\micro\volt} & \num{-4}  \\
    S5                    & \num{-97}  & \SI{3,2}{\micro\volt}  & 10        & \num{-117} & \SI{0,32}{\micro\volt} & \num{-10} \\
    S4                    & \num{-103} & \SI{1,6}{\micro\volt}  & +4        & \num{-123} & \SI{0,16}{\micro\volt} & \num{-16} \\
    S3                    & \num{-109} & \SI{0,8}{\micro\volt}  & \num{-2}  & \num{-129} & \SI{0,08}{\micro\volt} & \num{-22} \\
    S2                    & \num{-115} & \SI{0,4}{\micro\volt}  & \num{-8}  & \num{-135} & \SI{0,04}{\micro\volt} & \num{-28} \\
    S1                    & \num{-121} & \SI{0,21}{\micro\volt} & \num{-14} & \num{-141} & \SI{0,02}{\micro\volt} & \num{-34} \\
  \end{tabular}
  \caption{Tabell över S-värden, spännigar och effekter}
\end{table}

\newpage

Signalstyrkan mäts vid mottagarens antenningång, varför skillnaden i
signalstyrkan olika antenner och mottagningsriktningar samt dämpningen
i antenn och nedledning kan behöva bedömas.

I kortvågsområdet (under \SI{30}{\mega\hertz}) uppträder ett atmosfäriskt
bredbandigt brus tillsammans med bruset från den stora mängden rundradio- m.fl.
andra starka sändare.
Detta brus är mer dominerande än mottagarens interna brus.
I praktiken har de flesta KV-mottagare en högre brusnivå än
\SI{0,1}{\micro\volt}.

Över \SI{30}{\mega\hertz} däremot, är det mest mottagarens interna brus som
sätter gränsen för hörbarheten av svaga signaler.
Med samma S-skala som för kortvågsområdet, börjar man uppfatta signaler i
bruset utan att S-metern ger utslag.

Vid IARU Region~1-konferensen 1978 i Miskolcz föreslog de nationella
föreningarna VERON (Nederländerna) och RSGB (Storbritannien) en annan
S-skala över \SI{30}{\mega\hertz}.
Vid konferensen 1981 i Brighton antogs förslaget som rekommendation.

Mätningar ska i båda fallen göras med en kvasi-toppvärdesdetektor
med en stigtid av 10~ms \(\pm\)0,2~ms och en falltid av \SI{500}{\milli\second}.

