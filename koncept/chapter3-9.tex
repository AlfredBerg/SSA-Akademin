\mediumtopfig[0.5]{images/cropped_pdfs/bild_2_3-89.pdf}{A3E-modulator}{fig:BildII3-89}

\section{Modulatorer}
\index{modulatorer}

\subsection{Allmänt}
\index{modulation}

När en signal (bärvåg) påverkas så att den överför informationen i
en annan signal, sägs bärvågen bli modulerad.
Detta förlopp kallas modulation.
Vad som då händer behandlas främst i avsnitt \ssaref{modulation}, med
tillämpningar i kapitel \ssaref{sändare} och delvis i kapitel \ref{mottagare}.

En anordning för modulation kallas för modulator.
En modulator kan ingå som en funktion i sändare, mottagare med flera system.
Beroende på modulationsmetoden används olika kombinationer av delkretsar som
tillsammans utgör modulatorn.

I detta avsnitt ges exempel på några vanliga modulatorer för sändare.

\subsection{Amplitudmodulatorer}
\index{amplitudmodulator}
\index{amplitudmodulation}

Med en amplitudmodulator påverkas bärvågens amplitud proportionellt
mot den modulerande signalens amplitud.

\index{A1A}
\textbf{Vid sändningsslaget A1A} är amplituden på den modulerande signalen
antingen maximal eller ingen.
Då består modulatorn av en nycklingskrets, som påverkar till exempel ett
drivsteg i sändaren så att bärvågen släpps fram helt eller inte alls.

\index{A3E}
\textbf{Vid sändningsslaget A3E} har den modulerande signalens amplitud
ett analogt förlopp, till exempel tal, med vilket bärvågens amplitud moduleras.
Här beskrivs amplitudmodulation i en förstärkare med ett katodkopplat
elektronrör.
En emitterkopplad transistorförstärkare kan moduleras på ett liknande sätt.
I båda fallen moduleras förstärkarens arbetsspänning (anodspänning respektive 
kollektorspänning) med den modulerande signalen.
Det som då händer är att två signaler blandas på ett sätt som beskrivs i
avsnitt \ssaref{modulation} med tillämpning på A3E.
I vila är då bärvågsamplituden halva den möjliga inom arbetskurvans linjära del.
Vid modulation kommer bärvågens amplitud att variera mellan noll
och den möjliga amplituden.

% \mediumtopfig[0.5]{images/cropped_pdfs/bild_2_3-89.pdf}{A3E-modulator}{fig:BildII3-89}

Bild \ssaref{fig:BildII3-89} visar ett sändarslutsteg med en triod.
I serie med tilledningen för anodspänningen finns sekundärlindningen av en
modulationstransformator för LF-signalen.

Den LF-förstärkare som driver transformatorn måste för 100~\% modulationsgrad
kunna avge bärvågens halva effekt.
Eftersom uteffekten från en fullt utmodulerad A3E-sändare är 150~\% av den i
vila, måste slutsteget dimensioneras därefter.
Utöver den egna signalspänningen måste modulationstransformatorn även klara
slutstegets arbetsspänning.

\index{klass A}
\index{klass C}
Om som på bilden anodspänningen i ett förstärkarsteg amplitudmoduleras,
kan förstärkarsteget arbeta olinjärt, till exempel i klass C.
Varje följande förstärkarsteg måste däremot arbeta linjärt, till exempel i klass A.

På grund av den låga verkningsgraden och det stora bandbreddsbehovet 
används i dagens amatörradiosändare knappast ''äkta'' amplitudmodulering,
det vill säga A3E.
I stället används i läget ''AM'' nästan alltid H3E, det vill säga enkelt
sidband med full eller reducerad bärvåg (se nästa stycke).
Trots det lägre effektbehovet på grund av endast ett sidband och eventuellt
reducerad bärvågsamplitud kan av dimensioneringsskäl ändå inte de flesta
H3E-sändare avge sin fulla effekt kontinuerligt!

Som redan sagts i avsnitt \ssaref{modulation}, är det onödigt sända ut två sidband,
eftersom båda innehåller samma information.
Det räcker med ett sidband.
Bärvågen innehåller inte någon information.
Den kan därför undertryckas redan i sändaren för att ersättas i mottagaren.
Därmed uppstår sändningsslaget J3E.

\newpage
\subsection{Sändningsslaget J3E (SSB)}
\index{Single Side Band (SSB)}
\index{SSB}
\index{J3E}

Vid sändningsslaget J3E (SSB) sänds således endast ett sidband.
Det andra sidbandet och bärvågen undertrycks, vilket kan göras på flera sätt.
Numera är den så kallade filtermetoden allra vanligast och den enda som
behandlas här.

\mediumfig{images/cropped_pdfs/bild_2_3-90.pdf}{Alstring av J3E (SSB)}{fig:BildII3-90}

Bild \ssaref{fig:BildII3-90} visar alstring av J3E (SSB).
\index{Upper Side Band (USB)}
\index{USB}
\index{Lower Side Band}
\index{LSB|see {Lower Side Band}}
Med filtermetoden blandas HF- och LF-signalerna i en balanserad blandare där de
undertrycks medan blandningsprodukterna med deras summa- och
skillnadsfrekvenser blir kvar, det vill säga det övre och undre sidbandet.

För att undertrycka det ena sidbandet före sändningen följs blandaren
av ett bandpassfilter med bandbredd och frekvensläge för avsett sidband.
Den signal som sänds ut innehåller på så sätt endast ett sidband (Single Side
Band).

Valet mellan USB och LSB kan göras på två sätt.
Antingen genom att välja mellan ett separat passbandfilter för respektive
sidband eller genom att använda ett enda filter och flytta HF-signalen från ena
sidan till den andra av det filtret (se bild \ssaref{fig:BildII1-28} i
avsnitt \ssaref{modulation}).

En J3E-modulator enligt filtermetoden består således av en balanserad blandare
ofta en så kallad ringblandare (se bild \ssaref{fig:BildII3-87} i avsnitt
\ssaref{blandare}) samt ett bandpassfilter.
För att SSB-signalen ska få avsedd sändarfrekvens kan ytterligare
frekvensblandning behövas (se kapitel \ssaref{sändare}).

\subsection{Vinkelmodulation}
\index{vinkelmodulation}

Vinkelmodulation är samlingsnamnet för frekvensmodulation (FM) och
fasmodulation (PM).

\subsection{Frekvensmodulation}
\index{frekvensmodulation}
\index{F3E}

Vid sändningsslaget F3E (även kallat FM) varierar bärvågens frekvens i
takt med den modulerande signalens amplitud.
Bärvågen kommer på så sätt att pendla omkring en nominell frekvens, det vill
säga vara frekvensmodulerad.
Bärvågsamplituden ändras däremot inte vid frekvensmodulation.

Likspänningsnivåer kan således överföras eftersom en frekvensavvikelse
(deviation) i bärvågen endast påverkas av den modulerande signalens amplitud.

Vid F3E påverkas resonansfrekvensen i den resonanskrets i oscillatorn som
bestämmer dess arbetsfrekvens.
Det görs enklast genom att tillföra en kondensator med variabelt
kapacitansvärde, en kapacitansdiod (se avsnitt \ssaref{varicap}).

\mediumfig[0.8]{images/cropped_pdfs/bild_2_3-91.pdf}{Alstring av F3E (FM)}{fig:BildII3-91}
\mediumfig[0.8]{images/cropped_pdfs/bild_2_3-92.pdf}{Alstring av G3E (PM)}{fig:BildII3-92}

Bild \ssaref{fig:BildII3-91} visar en LC-resonanskrets där det ingår en
kapacitansdiod som styrs av en likspänning med en överlagrad modulerande LF-signal.
En likspänning tjänar som en ställbar förspänning till kapacitansdioden.
På så sätt kan man påverka arbetsfrekvensen.
Med den överlagrade LF-signalen påverkas arbetsfrekvensen i takt med
signalamplituden.

\subsection{Fasmodulation}
\index{fasmodulation}
\index{phasemodulation (PM)}
\index{PM}
\index{G3E}

Vid sändningsslaget G3E (även kallat PM) varierar bärvågens fasläge i
förhållande till en omodulerad referens.
Bärvågens amplitud ändras däremot inte.
Fasändringen -- deviationen -- är direkt proportionell mot hur snabbt fasläget
ändras och till den totala fasändringen.
Hastigheten på fasändringen är direkt proportionell mot frekvensen på den
modulerande signalen och till dess amplitud.

Det betyder att deviationen vid fasmodulation ökar både med amplituden
och frekvensen på den modulerande signalen.
Ändringar i likspänningsnivåer kan därför överföras endast om en fasreferens
används.

Fasmodulation kan alstras till exempel genom att påverka resonansfrekvensen i
en resonanskrets någonstans efter oscillatorn, det vill säga där
oscillatorfrekvensen inte påverkas.
Denna resonanskrets har i viloläge samma resonansfrekvens som oscillatorn.
När kretsen bringas ur resonans genom modulation -- samtidigt som kretsen
påtrycks oscillatorsignalen -- uppstår i kretsen omväxlande en induktiv och
kapacitiv reaktans -- detta inom tiden för varje halv period.
Reaktansen skapar därvid den fasförskjutning som innebär fasmodulation.
Se även avsnitt \ssaref{parallellresonans} och \ref{serieresonans}, bilderna
\ssaref{fig:BildII3-18} och \ref{fig:BildII3-19}

% \mediumfig[0.8]{images/cropped_pdfs/bild_2_3-92.pdf}{Alstring av G3E (PM)}{fig:BildII3-92}

Bild \ssaref{fig:BildII3-92} visar alstring av G3E (PM).
Liksom vid frekvensmodulation kan till exempel en kapacitansdiod användas för att
med en modulerande signal påverka resonansfrekvensen i en krets.
