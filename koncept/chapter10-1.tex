\chapter{EMC}
\label{EMC}
\index{EMC}

\noindent Det moderna samhället blir alltmer tekniskt avancerat och antalet elektroniska
apparater i hemmen och på arbetsplatserna ökar kraftigt.
Den ökande mängden och komplexiteten hos apparaterna kräver därför regler, som
styr både utförande och användning med rimligt bibehållen säkerhet och funktion.
Internationella och nationella väl preciserade regler för radio- och
teletekniskt samexistens är numera helt nödvändiga.

\section{Störningar och störkänslighet}

\subsection{Om EMC-lagen}
\label{EMC-lagen}
\index{EMC!lag}
\index{EMC!förordning}
\index{EMC!apparater}
\index{EMC!störningar}

Samlingsbegreppet är \emph{Electromagnetic Compatibility} (EMC), det vill säga
en apparats förmåga att fungera tillfredsställande i sin elektromagnetiska
omgivning så att den:

\begin{itemize}
\item inte alstrar en elektromagnetisk störning som överskrider en nivå som
  tillåter radio- eller teleutrustning eller annan utrustning att fungera som
  avsett

\item har en sådan tålighet att den elektromagnetiska störning som kan
  förväntas vid avsedd användning inte medför att utrustningens funktion
  försämras i en oacceptabel utsträckning.
\end{itemize}

Lagen om elektromagnetisk kompatibilitet, \emph{SFS 1992:1512} \cite{SFS1992:1512}
ger regeringen eller den myndighet regeringen bestämmer rätt att i fråga om
kommunikationer eller näringsverksamhet eller skydd för liv, personlig säkerhet
eller hälsa meddela föreskrifter om EMC.
Förordning om elektromagnetisk kompatibilitet \emph{SFS 2016:363}
\cite{SFS2016:363} definierar nyckelbegreppen; apparater, EMC, elektromagnetisk
störning och tålighet.

Lagen och förordningen tillsammans med Elsäkerhetsverkets föreskrifter
\emph{ELSÄK-FS} samt direktivet om elektromagnetisk kompatibilitet implementerar
EU-direktiv 2014/30/EU i Sverige.
Elsäkerhetsverket är ansvarig myndighet, med rätt att utfärda föreskrifter om
bland annat skyddskraven, kontroll och märkning samt om vissa undantag.

Ovanstående handlar om störningar orsakade av apparater eller störningar på
apparaters funktion.
Sådana störningar kan anmälas till Elsäkerhetsverket.
Störningar orsakade av radiosändare eller radiomottagare behandlas i avsnitt
\ssaref{LEK}.


\subsection{Utdrag ur LEK}
\label{LEK}
\index{LEK}
\index{störning}
\index{störning!skadlig}
\index{störning!tillåten}

Post- och telestyrelsens föreskrifter om undantag från tillståndsplikt för
användning av vissa radiosändare \emph{PTSFS 2022:19} \cite{PTSFS2022:19} hänvisar
till lag om elektronisk kommunikation \emph{LEK} \emph{SFS 2022:482}
\cite{SFS2022:482}.
Där kan följande läsas om åtgärder vid störningar:
%%
\begin{quote}
	3 Kap. 13\S~Om det uppkommer skadlig störning, skall tillståndshavaren
	omedelbart se till att störningen upphör eller i möjligaste mån minskar, om
	inte störningen är tillåten.
	Detsamma gäller den som använder en radiomottagare som stör användningen av en
	annan radiomottagare.
\end{quote}
%%
Skadlig störning definieras i 1 Kap 7\S~som:
%%
\begin{quote}
	störning som äventyrar funktionen hos en radionavigationstjänst eller någon
	annan säkerhetstjänst, eller som på annat sätt allvarligt försämrar,
	hindrar eller upprepat avbryter en radiokommunikationstjänst som fungerar i
	enlighet med gällande bestämmelser, inbegripet störning av befintliga eller
	planerade tjänster på nationellt tilldelade frekvenser
\end{quote}
%%
Tillåten störning definieras i förarbetena till \emph{LEK} som en störning
orsakad av användares delning av frekvens och anses då vara tillåten.
Observera dock att användare med sekundär status inte får störa användare med
primär status vid delning av frekvens eller frekvensband.
%%
I \emph{LEK} definieras även radioanläggning:
\index{radioanläggning}
\begin{quote}
	anordning som möjliggör radiokommunikation eller bestämning av position,
	hastighet eller andra kännetecken hos ett föremål genom sändning av radiovågor
	(radiosändare) eller mottagning av radiovågor (radiomottagare)
\end{quote}

\subsection{Utstrålning från amatörradiosändare}
\index{uteffekt!begränsning}

Vad som sägs i 3 Kap 4\S~Lag om elektronisk kommunikation och skrivningen i
Post- och telestyrelsens föreskrifter om undantag från tillståndsplikt för
användning av vissa radiosändare 3 Kap 14\S~\emph{De tekniska egenskaperna hos
	amatörradiosändaren ska anpassas så att de inte stör användningen av andra
	radioanläggningar}.
Tillsammans med skrivningarna i Strålsäkerhetsmyndighetens \emph{SSMFS 2008:18}
och Förordning om elektromagnetisk kompatibilitet \emph{SFS 2016:363}.
Medför detta att sändareffekten alltid ska anpassas så att styrkan av utstrålade
fält inte förorsakar störningar eller för höga nivåer av elektromagnetiska fält.

Den enligt undantagsföreskrifterna högsta tillåtna uteffekten kan alltså inte
användas hinderslöst.

Om störningarna inte kan avhjälpas kan PTS komma att anvisa om restriktioner
(begränsningar i sändningstillståndet), det kan vara sändningsförbud under
vissa tider, på vissa frekvenser, över viss sändareffekt etc.

%% k7per: PM?
\subsection{PM vid störningsproblem}
\index{störning!PM}

\begin{itemize}
	\item Störningar är alltid förenade med obehag och ställer grannsämjan på prov.
	Håll dig väl med dem som bor i omgivningen! Observera även att gällande EMC
	har myndigheter inte rättighet att få tillträde till bostäder.
	\item Om det väcks klagomål på dig om störningar, ska du först
	kontrollera din egen sändare och antennanläggning.
	\item Be därefter att få undersöka antennanläggning och apparater hos
	den som besväras av störningar.
	\item Om du ser en lösning, berätta om vad som kan göras.
	Kom överens om vad som får göras.
	Ändra då inte något inne i apparater, men prova gärna ut yttre,
	kompletterande filter etc.
	\item Om det inte går att komma till rätta med störningarna bör de som
	levererat och installerat anläggningen anlitas.
	\item Störningsanmälan gällande radiokommunikation kan göras på PTS webbplats.
	\item Störningsanmälan gällande produkter kan göras till Elsäkerhetsverket.
\end{itemize}

\subsection{Arbeta aktivt med avstörning}
\index{störning!avstörningslåda}

\begin{itemize}
	\item Låna hem en av SSA:s avstörningslådor och försök att finna en lösning.
	I lådan finns ett sortiment av frekvensfilter för avstörning.
	\item Undvik att störa i onödan.
	Sänk sändareffekten och begränsa sändningstiden under utprovningen av en
	lösning.
\end{itemize}
%%
Lyckas du inte själv med att störa av:
%%
\begin{itemize}
	\item Ta gärna hjälp av en radioamatör med erfarenhet av avstörning.
	\item Anlita annan sakkunnig hjälp.
\end{itemize}
