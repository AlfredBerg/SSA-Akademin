\section[Vågutbredning på VHF-EHF]{Vågutbredning på VHF, UHF, SHF och EHF}
\label{vågutbredning_vhf}
\index{vågutbredning!VHF}
\index{vågutbredning!UHF}
\index{vågutbredning!SHF}
\index{vågutbredning!EHF}
\index{vågutbredning}
\index{troposfären}
\index{sked}
\index{VHF}
\index{UHF}
\index{SHF}
\index{EHF}
\index{UKV}

\subsection{Allmänt}
Frekvensområdet \SIrange{30}{300000}{\mega\hertz} delas upp i följande mindre
avsnitt som kallas:

\begin{itemize}
  \item Very High Frequency (VHF), \SIrange{30}{300}{\mega\hertz}
  \item Ultra High Frequency (UHF), \SIrange{300}{3000}{\mega\hertz}
  \item Super High Frequency (SHF), \SIrange{3}{30}{\giga\hertz}
  \item Extremely High Frequency (EHF), \SIrange{30}{300}{\giga\hertz}.
\end{itemize}

På VHF och högre frekvenser (tidigare UKV) förekommer sällan någon
vågutbredning via jonosfären annat än under tiden för maximal solaktivitet.
I stället utnyttjas den lägre delen av atmosfären och
knappast högre än \SIrange{4}{5}{\kilo\metre} över jordytan.
Denna del av atmosfären kallas för troposfär och vågutbredningen benämns därför
troposfärisk vågutbredning.

All vågutbredning i troposfären förutsätter i princip optisk sikt.
Emellertid förekommer en viss vågavböjning utmed jordytan, varför den praktiska
räckvidden utmed siktlinjen är något längre än till den optiska horisonten.
Man talar om radiohorisont.

På de högre frekvenserna är det på grund av vågutbredningen ofta svårare att
få radiokontakt med andra radioamatörer.
En metod för att veta att det finns någon som lyssnar i andra ändan är att komma
överens om frekvens, riktning och tidpunkt i förväg.
En sådan överenskommelse kallas \emph{sked} förkortat av
(eng. \emph{Scheduled QSO} via \emph{sched}) och är vanlig även på
kortvågsfrekvenser.

\emph{Brytningsindex} i atmosfären är en viktig faktor för vågutbredning bortom
frisiktsavståndet, speciellt vid frekvenser över \SI{100}{\mega\hertz}.
Även den splittring av vågorna som uppstår när de träffar oregelbundenheter i
atmosfären kan utnyttjas för kommunikation på avstånd som är flera gången
frisiktsavståndet.

Vid högre frekvenser begränsas emellertid räckvidden av atmosfärens
dämpande inverkan.
Likaså förloras vågenergi i den topografi, vegetation och bebyggelse som ligger
i siktlinjen mellan sändare och mottagare.
I gynnsamma fall är det dock möjligt att överbrygga avstånd på upp till
\SI{1000}{\kilo\metre} genom troposfären.
Sådana avstånd kallas för överräckvidd.

\subsection{Troposfären -- Troposcatter}
\harecsection{\harec{a}{7.10}{7.10}}
\index{vågutbredning!troposcatter}
\index{troposfären}

När en kallfront nära jordytan stöter samman med en varmfront uppstår turbulens
i luften med elektriska uppladdningar i gränsskiktet som följd.

Under sådana väderförhållanden kan radiovågor i VHF-området och däröver att
brytas eller splittras upp när de träffar det laddade gränsskiktet --
\emph{troposcatter}.
Då kan oväntade radiokontakter uppnås.

\subsection{Temperaturinversion}
\harecsection{\harec{a}{7.12}{7.12}}
\index{temperaturinversion}
\index{vågutbredning!temperaturinversion}
\index{duct}

När ett varmt luftskikt lägger sig över ett kallare luftskikt uppstår
en så kallad \emph{temperaturinversion}.

Vågor på VHF och UHF bryts då mot gränsskiktet och böjs av mot jordytan.
Om det finns två inversionsskikt samtidigt, så kan de bilda
en slags vågledare, så kallad \emph{dukt} (eng. \emph{duct} = ledning).
En räckvidd på \SIrange{600}{1300}{\kilo\metre} kan uppnås.
Denna typ av vågutbredning förekommer ofta vid högt atmosfärstryck under
sommaren.

\subsection{Reflektion mot E\raisebox{-.4ex}{s} (sporadiskt E)}
\harecsection{\harec{a}{7.13}{7.13}}
\index{sporadiskt E}
\index{vågutbredning!sporadiskt E}

Vid stark solinstrålning bildas, på de lägre latituderna, joniserade gasmoln på
en höjd av cirka \SI{120}{\kilo\metre} och med en oregelbunden fördelning.

Den kritiska frekvensen är hög för \(\mathrm{E_s}\)-skiktet och det kan även
reflektera vågor på VHF och UHF så effektivt att avstånd av
\SIrange{1000}{4000}{\kilo\metre} kan överbryggas.

\subsection{Aurorareflektion}
\harecsection{\harec{a}{7.14}{7.14}}
\index{aurora}
\index{vågutbredning!Aurora}
\index{norrsken}
\index{polarsken}

Soleruptioner (flares) utstrålar stora mängder ultraviolett ljus och kastar ut
elektriskt laddade partiklar, som efter 1--2 dagar fångas upp av jordens
magnetosfär och tränger ner i polarzonerna.
När partiklarna kolliderar med atmosfären bildas det polarsken i form av lysande
''draperier'' -- \emph{Aurora borealis} kallat norrsken på norra halvklotet
eller \emph{Aurora australis} kallat sydsken på södra halvklotet -- samtidigt
som atmosfären joniseras.
Aurora är joniserade skikt i samma plan som jordens magnetfält och speciellt
vågor med frekvenser över \SI{30}{\mega\hertz} reflekteras emot dessa.

VHF- och UHF-kommunikation kan ske med hjälp av aurorareflektion.
De signaler som reflekteras av Aurora är kraftigt distorderade och har förlorat
all ton.
Den reflekterade signalen blir bred i frekvens, vilket emellertid gynnar
kommunikation med telegrafi när signalerna är svaga.
Oftast är endast telegrafiförbindelser i långsam takt möjliga.
Vid starkare Aurora går också SSB att använda.

\subsection{Reflektion mot meteorer -- Meteorscatter}
\harecsection{\harec{a}{7.15}{7.15}}
\index{meteorscatter}
\index{vågutbredning!meteorscatter}

Radiovågor på VHF och UHF reflekteras mot joniserade spår efter det
meteorgrus som faller in i jordatmosfären.
Detta fenomen kan utnyttjas för radioförbindelser.

Joniseringen sker när partiklarna passerar genom E-skiktet och brinner upp.
Eftersom joniseringen har en varaktighet av endast 0,1--10~sekunder måste
\emph{MS-förbindelser} planeras och förberedas väl.
Förbindelserna begränsas vanligen till utbyte av anropssignaler och
signalrapporter med höghastighetstelegrafi med en hastighet av
300--3000~tecken per minut.
Under de större meteorskurarna kan kontakter uppnås utan överenskommelser på
förhand (''sked''), både på telegrafi (CW) och telefoni (SSB).

\subsection{EME-förbindelser}
\harecsection{\harec{a}{7.16}{7.16}}
\index{EME}
\index{vågutbredning!EME}
\index{månstuds}
\index{vågutbredning!månstuds}

Radioförbindelse från en punkt på jorden till en annan kan åstadkommas
genom reflektion av VHF-/UHF-signaler mot månen.
\emph{EME-förbindelser} (eng. \emph{Earth--Moon--Earth}) kallas även
\emph{månstuds} (eng. \emph{Moon Bou\-n\-ce}).
EME-förbindelser kräver antenner med mycket hög riktverkan, mycket hög
sändareffekt och känsliga mottagare.

\newpage
\subsection{Markbaserade relästationer}
\index{repeater}

\mediumfig{images/cropped_pdfs/bild_2_7-12.pdf}{Markbaserad repeater}{fig:bildII7-12}

På VHF och högre frekvenser kan man, som tidigare beskrivits, endast
uppnå radiokontakter hitom den så kallade radiohorisonten.

För att överbrygga detta hinder används relästationer, se bild
\ssaref{fig:bildII7-12}.
Den slags relästation, som allmänt kallas \emph{repeater}, tar emot det den hör
på en viss fast frekvens och återutsänder detta på en viss annan fast frekvens.
Se bandplan i bilaga \ssaref{bandplaner}.

\subsection{Rymdsatellit-baserade relästationer}
\index{satellit}
\index{OSCAR}

Radiovågor med tillräckligt hög frekvens kan passera genom jonosfärskikten.
Detta möjliggör radioförbindelser VHF/UHF/SHF mellan
stationer på jorden med hjälp av relästationer i rymdsatelliter.

För amatörradiotrafik över rymdsatelliter används vanligen den slags
relästation, som kallas \emph{transponder}.
En sådan tar emot allt det den hör inom ett helt frekvensband och återutsänder
detta i ett helt annat frekvensband.
På så sätt kan trafik över satellit ske på ett jämförbart sätt som vid
direktkontakt mellan jordbaserade stationer.

Satellitbaserade linjärtranspondrar med amatörradioutrustning finns i
OSCAR-satelliterna (OSCAR = Orbiting Satellite Carrying Amateur Radio).
Dessa har konstruerats och byggts av amatörradiogrupper.

OSCAR-satelliterna har många olika transpondrar i funktion, vilka var och en
arbetar med olika kombinationer av sändningsslag (moder) och frekvensband.
Detta kallas numera att de har olika konfiguration.

En vanlig transponderkonfiguration är CONFIG-V/U (f.d. MOD-J) där upplänken
är på VHF-bandet, till exempel \SIrange{145,900}{146,000}{\mega\hertz} och
nerlänken på UHF-bandet till exempel \SIrange{435,800}{435,900}{\mega\hertz}.
Varje upplänkfrekvens motsvarar en bestämd nerlänkfrekvens, till exempel 
145,950 upp och \SI{435,850}{\mega\hertz} ner.
Trafiken över transpondern kan därför ske i full duplex.

Man kan då prata och lyssa samtidigt i båda riktningarna, vilket
starkt förbättrar trafiken och gör samtalen roligare och intressantare.

En så kallad linjär transponder kan inte bara överföra FM, utan även SSB,
tontelegrafi och SSTV.
Dessutom även RTTY och andra digitala trafiksätt.

Nästan alla amatörradioband med tillräckligt hög frekvens används i
olika kombinationer som upp- och nerlänkar i de olika OSCAR-satelliterna.
AMSAT är den organisation, som fortlöpande informerar om amatörradiosatelliter.
Den svenska grenen på AMSAT är AMSAT-SM som är aktiva och har på sin webbplats
beskrivningar både för nybörjare och de som kommit lite längre om hur man
använder amatörsatelliter.

\mediumfig{images/cropped_pdfs/bild_2_7-13.pdf}{Transponder i rymdsatellit}{fig:bildII7-13}

Amatörradion utvecklas mycket snabbt genom den satellitbaserade
verksamheten och det kommer upp allt mer sofistikerade OSCAR-satelliter.
Tendensen är att man efter hand går över till allt
högre frekvensband och allt mera av digitala sändningsslag.

Med hjälp av satellit kan förbindelseavståndet bli mycket stort även
med enkel utrustning och små antenner.
En fördel med kommunikation över rymdsatellit är också att den till största
delen är oberoende av vågutbredningsvillkoren.
%
Se bild \ssaref{fig:bildII7-13}.
