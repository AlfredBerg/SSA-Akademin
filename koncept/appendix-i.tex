\chapter[Svenska repeatrar]{Frekvenser för amatörradiorepeatrar}
\label{svenska repeatrar}

Vid direktförbindelser på höga frekvenser är räckvidden begränsad,
särskilt vid låg effekt och små antenner.
Med repeatrar med högt belägna antenner kan räckvidden förbättras,
vilket underlättar kommunikation med rörliga (mobila) radiostationer.
Eftersom sändaren och mottagaren i en repeater arbetar samtidigt, måste
avståndet mellan deras arbetsfrekvenser vara så stort att det inte uppstår
ömsesidiga störningar.
Dessa arbetsfrekvenser kallas frekvenspar eller kanal och avståndet mellan dem
kallas repeaterskift, vilket är enhetligt inom repeaterbandet.

Frekvensparet i en repeater måste arbeta med omvänt frekvensläge i förhållande
till det i de stationer som den betjänar.
Kanalavståndet mellan repeatrarna i ett band är också enhetligt och sändningar
över repeatrarna måste naturligtvis ha mindre bandbredd än kanalavståndet.
Inom IARU har man enats bland annat om frekvensparen för smalbandiga
FM-repeatrar.
Se IARU:s bandplaner för 10~m repeatrar i bilaga \ref{IARU bandplan}.
För VHF-repeatrar finns bandplanen i bilaga \ref{Bandplan VHF och högre}.
Frekvensplaner finns för repeatrar inom banden \SIrange{51}{52}{\mega\hertz}
(\SI{6}{\metre}), \SIrange{145}{146}{\mega\hertz} (\SI{2}{\metre}),
\SIrange{432}{438}{\mega\hertz} (\SI{70}{\centi\metre}),
\SIrange{1240}{1300}{\mega\hertz} (\SI{10}{\centi\metre}) samt
\SIrange{28000}{29700}{\kilo\hertz} (\SI{10}{\metre}).

\section{Kanalnumreringsmetod}
Vid införandet av \SI{12,5}{\kilo\hertz} kanalavstånd på 2~meter- och
\SI{70}{\centi\metre}-banden infördes ett nytt system.
Man börjar med en bokstav som talar om vilket band det är
\begin{itemize}
  \item F för \SI{51}{\mega\hertz}, kanalavstånd \SI{10}{\kilo\hertz}
  \item V för \SI{145}{\mega\hertz}, kanalavstånd \SI{12,5}{\kilo\hertz}
  \item U för \SI{430}{\mega\hertz}, kanalavstånd \SI{12,5}{\kilo\hertz}.
\end{itemize}
Kanalnumret börjar med 00 på varje sådant band och ökar med ett (1) för varje
kanal i bandet.
På 51 och \SI{145}{\mega\hertz} används tvåsiffrig numrering och på
\SI{430}{\mega\hertz} tresiffrig.
För repeaterkanaler sätts ett R före bandbokstaven.

\newpage

\section{70-centimetersbandet}
Repeaterskift \SI{2000}{\kilo\hertz}.

\begin{tabular}{ r | c | l | l }
	Kanal &       & Sändar-        & Mottagar-  \\
	Ny    & fd    & frekvens       & frekvens \\
          &       & [\si{\mega\hertz}] & [\si{\mega\hertz}] \\
	\hline
	RU361 &       & 432,5125       & 434,5125 DV    \\
	RU362 &       & 432,5250       & 434,5250 DV    \\
	RU363 &       & 432,5375       & 434,5375 DV    \\
	RU364 &       & 432,5500       & 464,5500 DV    \\
	RU365 &       & 432,5625       & 434,5625 DV    \\
	RU366 &       & 432,5750       & 464,5750 DV    \\
	RU367 &       & 432,5875       & 434,5875 DV    \\
	RU368 & RU0   & 432,6000       & 434,6000       \\
	RU369 & RU0x  & 432,6125       & 434,6125       \\
	RU370 & RU1   & 432,6250       & 434,6250       \\
	RU371 & RU1x  & 432,6375       & 464,6375       \\
	RU372 & RU2   & 432,6500       & 434,6500       \\
	RU373 & RU2x  & 432,6625       & 464,6625       \\
	RU374 & RU3   & 432,6750       & 434,6750       \\
	RU375 & RU3x  & 432,6875       & 434,6875       \\
	RU376 & RU4   & 432,7000       & 434,7000       \\
	RU377 & RU4x  & 432,7125       & 434,7125       \\
	RU378 & RU5   & 432,7250       & 434,7250       \\
	RU379 & RU5x  & 432,7375       & 434,7375       \\
	RU380 & RU6   & 432,7500       & 434,7500       \\
	RU381 & RU6x  & 432,7625       & 434,7625       \\
	RU382 & RU7   & 432,7750       & 434,7750       \\
	RU383 & RU7x  & 432,7875       & 434,7875       \\
	RU384 & RU8   & 432,8000       & 434,8000       \\
	RU385 & RU8x  & 432,8125       & 434,8125       \\
	RU386 & RU9   & 432,8250       & 434,8250       \\
	RU387 & RU9x  & 432,8375       & 434,8375       \\
	RU388 & RU10  & 432,8500       & 434,8500       \\
	RU389 & RU10x & 432,8625       & 434,8625       \\
	RU390 & RU11  & 432,8750       & 434,8750       \\
	RU391 & RU11x & 432,8875       & 434,8875       \\
	RU392 & RU12  & 432,9000       & 434,9000       \\
	RU393 & RU12x & 432,9125       & 434,9125       \\
	RU394 & RU13  & 432,9250       & 434,9250       \\
	RU395 & RU13x & 432,9375       & 434,9375       \\
	RU396 & RU14  & 432,9500       & 434,9500       \\
	RU397 & RU14x & 432,9625       & 434,9625       \\
	RU398 & RU15  & 432,9750       & 434,9750       \\
	RU399 & RU15x & 432,9875       & 434,9875       \\
\end{tabular}

\newpage

\section{2-metersbandet}
Repeaterskift \SI{600}{\kilo\hertz}.

\begin{tabular}{ r | c | l | l }
	Kanal & & Din sändar- & Din mottagar- \\
	Ny    & fd & frekvens [\si{\mega\hertz}] & frekvens [\si{\mega\hertz}] \\
	\hline
	RV46 & & 144,975 & 145,575\\
	RV47 & & 144,9875 & 145,5875\\
	RV48 & R0 & 145,000 & 145,600 \\
	RV49 & R0x & 145,0125 & 145,6125 \\
	RV50 & R1 & 145,025 & 145,625 \\
	RV51 & R1x & 145,0375 & 145,6375 \\
	RV52 & R2 & 145,050 & 145,650 \\
	RV53 & R2x & 145,0625 & 145,6625 \\
	RV54 & R3 & 145,075 & 145,675 \\
	RV55 & R3x & 145,0875 & 145,6875 \\
	RV56 & R4 & 145,100 & 145,700 \\
	RV57 & R4x & 145,1125 & 145,7125 \\
	RV58 & R5 & 145,125 & 145,725 \\
	RV59 & R5x & 145,1375 & 145,7375 \\
	RV60 & R6 & 145,150 & 145,750 \\
	RV61 & R6x & 145,1625 & 145,7625 \\
	RV62 & R7 & 145,175 & 145,775 \\
	RV63 & R7x & 145,1875 & 145,7875 \\
\end{tabular}

\section{23-centimetersbandet}
Repeaterskift \SI{6000}{\kilo\hertz}.

\begin{tabular}{ l | l | l }
	Kanal & Din sändar- & Din mottagar- \\
	& frekvens [MHz] & frekvens [MHz] \\
	\hline
	RM0 & 1291,000 & 1297,000 \\
	RM1 & 1291,025 & 1297,025 \\
	RM2 & 1291,050 & 1297,050 \\
	RM3 & 1291,075 & 1297,075 \\
	RM4 & 1291,100 & 1297,100 \\
	RM5 & 1291,125 & 1297,125 \\
	RM6 & 1291,150 & 1297,150 \\
	RM7 & 1291,175 & 1297,175 \\
	RM8 & 1291,200 & 1297,200 \\
	RM9 & 1291,225 & 1297,225 \\
	RM10 & 1291,250 & 1297,250 \\
	RM11 & 1291,275 & 1297,275 \\
	RM12 & 1291,300 & 1297,300 \\
	RM13 & 1291,325 & 1297,325 \\
	RM14 & 1291,350 & 1297,350 \\
	RM15 & 1291,375 & 1297,375 \\
	RM16 & 1291,400 & 1297,450 \\
	RM17 & 1291,425 & 1297,475 \\
	RM18 & 1291,450 & 1297,450 \\
	RM19 & 1291,475 & 1297,475 \\
\end{tabular}

\newpage

\section[Speciella band]{Repeaterband med speciella egenskaper}
\subsection{6-metersbandet}
Repeaterskift \SI{600}{\kilo\hertz}.

\begin{tabular}{ l | l | l }
  Kanal & Din sändar- & Din mottagar- \\
        & frekvens [MHz] & frekvens [MHz] \\
  \hline
  RF81 & 51,210 & 51,810 \\
  RF83 & 51,230 & 51,830 \\
  RF85 & 51,250 & 51,850 \\
  RF87 & 51,270 & 51,870 \\
  RF89 & 51,290 & 51,890 \\
  RF91 & 51,310 & 52,910 \\
  RF93 & 51,330 & 52,930 \\
  RF95 & 51,350 & 52,950 \\
  RF97 & 51,370 & 52,970 \\
  RF99 & 51,390 & 52,990 \\
\end{tabular}

(dvs endast udda kanalnummer används).

På grund av den relativt låga frekvensen uppnås ofta överräckvidder på grund av
sporadisk vågutbredning via E-skiktet.
Man kan då uppnå förbindelser utan hjälp av repeater.

\subsection{10-metersbandet}
Repeaterskift \SI{100}{\kilo\hertz}.

\begin{tabular}{ l | l | l }
  Kanal & Din sändar- & Din mottagar- \\
        & frekvens [kHz] & frekvens [kHz] \\
  \hline
  RH1 & 29520 & 29620 \\
  RH2 & 29530 & 29630 \\
  RH3 & 29540 & 29640 \\
  RH4 & 29550 & 29650 \\
  RH5 & 29560 & 29660 \\
  RH6 & 29570 & 29670 \\
  RH7 & 29580 & 29680 \\
  RH8 & 29590 & 29690 \\
\end{tabular}

På grund av den relativt låga frekvensen uppnås stora räckvidder genom
jonosfärisk vågutbredning, särskilt under år med högt solfläckstal.
Även sporadisk vågutbredning via E-skiktet förekommer.
I båda fallen bör repeatertrafik undvikas.
