\chapter{Måttenheter}

\noindent




\section{Närmevärden och Noggrannhet}

När vi skriver ett tal så är det oftast inte exakt det \emph{korrekta
värdet}, vi skriver ner ett \emph{närmevärde} eller en
\emph{approximation}.  Det kan vara för att vi inte vet exakt vad
taket är eller att vi bara kunnat mäta det med en vis
noggrannhet. Antalet siffror, \emph{värdesiffror} eller
\emph{signifikanta siffror}, vi använder visar hur noggrant
närmevärdet är.  Antalet värdesiffror är lika med antalet siffror i
talet, exklusive inledande nollor.  Om avslutande nollor är
signifikanta eller inte beror på hur närmevärdet är avrundat, se
tabellen nedan för exempel.

\bigskip
\begin{centering}
\begin{tabular}{|l|l|}
\hline
Tal & Antal värdesiffror \\
\hline
0,04711 & 4\\
4711 & 4 \\
4711,000 & 7 \\
4\,711\,000 & 4 till sju beroende på avrundning \\
\hline
\end{tabular}
\end{centering}
\bigskip

Den sista raden i tabellen visar problemet att storheten och
noggrannheten inte kan separeras om man skriver ett tal på vanligt
vis, det finns inget sätt att veta vilka av de avslutande nollorna som
är värdesiffror och hur talet är avrundat.  För att inte blanda ihop
storhet och noggrannhet hos ett tal använder man \emph{vetenskaplig
notation}. 
% Heter det grundpotensform?

Vetenskaplig notation skriver bara ut värdesiffrorna, med ett
decimalkomma efter den första värdesiffran, multiplicerat med en
tiopotens som bestämmer storheten. Till exempel ljusets hastighet
avrundat till 2 värdesiffror är $3,0 \cdot 10^8
\si{\metre\per\second}$.

Om $|\Delta a|\leq 0,5\cdot 10^{-t}$, där $\Delta a$ är skillnaden
mellan det korrekta värdet och närmevärdet, sägs närmevärdet $\bar
{a}$ ha $t$ korrekta decimaler.
I ett närmevärde med $t>0$ korrekta decimaler sägs alla siffror i
positioner med enhet större än eller lika med $10^{-t}$ vara
\emph{signifikanta siffror}, utom inledande nollor, som endast anger
decimaltecknets läge.

Om man utför beräkningar med tal med olika antal värdesiffror så är en
bra tumregel att man i slutet har ett svar med lika många värdesiffror
som i det minst noggranna talet. En mer detaljerad analys av hur
beräkningar påverkar noggrannheten studeras inom \emph{numerisk
analys}.

Det är viktigt att man inte använder fler värdesiffror än man behöver.
De flesta mätningar man kan göra levererar ganska få värdesiffror.
För att sätta noggrannhet i perspektiv så använder NASA 15 siffror för
att skicka rymdfarkoster runt i solsystemet och det behövs ungefär
39--40 värdesiffror för att beskriva universums omkrets till en atoms
storlek.

Enheterna i tabellerna är praktiska förkortningar av tiopotenserna i steg om tre.

%% tabell

 \section{Representation}

Irrationella tal kan inte utryckas exakt utan ett bråktecken. Till
exempel 1/3 kan approximeras som 0,333, med just tre värdesiffror.
Ibland skriver man 0,333…, tre punkter, för att visa att serien
fortsätter för evigt. Om man kan är det bäst att fullfölja beräkningar
med bråktal och förenkla och avrunda på slutet.

Ibland kan ett och samma flyttal representeras på två olika sätt. Ett
exempel som ofta förvånar är att tio kan skrivas som 9,999… och som
10.


Inom fysiken förekommer allt mellan mycket höga och mycket låga
  värden på frekvens, spänning, ström, resistans etc.
  I en radiomottagares antenningång är signalspänningen ofta mindre än
  \SI{0,000001}{\volt}.
  I slutsteget i en amatörradiosändare kan anodspänningen vara mer än
  \SI{2000}{\volt} och uteffekten upp till \SI{1000}{\watt}.
  I spektrum för elektromagnetiska vågor finns mycket höga frekvenser
  så som \SI{10000000000}{\hertz}.

  För att ange storheten på måttenheter används ofta ett \emph{prefix} före
  måttenheten (av latinets \emph{pre}, före och \emph{fixare}, att tillägga).
  Med prefixet anges från fall till fall vilken multiplikations- eller
  divisionsfaktor (talfaktor) som används. (Se tabell \ref{tab:prefix}.)

  I exemplen ovan blir signalspänningen \SI{1}{\micro\volt}, anodspänningen
  \SI{2}{\kilo\volt}, uteffekten \SI{1}{\kilo\watt} samt frekvensen
  \SI{10}{\giga\hertz} vilket i många fall kan vara lättare att läsa och svårare
  att misstolka.

  Märk, att enhetens sort inte har något att göra med själva prefixet.
  Nedan ges sorterna Hz, W, V, F etc. som exempel.

  Exponenter, till exempel siffran 6 i uttrycket \(10^6\), förklaras i
  bilaga \ref{potenser}.

  %% k7per: What is this table supposed to explain, why do only some lines have units in the example?  Fixed one error with 1, but I am not sure what to do here?
  \begin{table*}[b]
    \begin{center}
      \begin{tabular}{|llll|}
        \hline
        1\,000\,000\,000\,000\,000\,000 Hz & = 1 EHz & = \(1 \cdot 10^{18}\) Hz & (E är exa) \\
        1\,000\,000\,000\,000\,000 Hz & = 1 PHz & = \(1 \cdot 10^{15}\) Hz & (P är peta) \\
        1\,000\,000\,000\,000 Hz & = 1 THz & = \(1 \cdot 10^{12}\) Hz & (T är tera) \\
        1\,000\,000\,000 W & = 1 GW & = \(1 \cdot 10^9\) W & (G är giga) \\
        1\,000\,000 W & = 1 MW & = \(1 \cdot 10^6\) W & (M är mega) \\
        1\,000 W & = 1 kW & = \(1 \cdot 10^3\) W & (k är kilo) \\
        100 & & = \(1 \cdot 10^2\) & (h är hekto) \\
        10 & & = \(1 \cdot 10^1\) & (da är deka) \\
        1 V & & = \(1 \cdot 10^0\) V & (\(1 = 10^0\) är grundenhet) \\
        0,1 & & = \(1 \cdot 10^{-1}\) & (d är deci) \\
        0,01 & & = \(1 \cdot 10^{-2}\) & (c är centi) \\
        0,001 V & = 1 mV & = \(1 \cdot 10^{-3}\) V & (m är milli) \\
        0,000\,001 V & = 1 \(\mu V\) & = \(1 \cdot 10^{-6}\) V & (\(\mu \) är mikro) \\
        0,000\,000\,001 F & = 1 nF & = \(1 \cdot 10^{-9}\) F & (n är nano) \\
        0,000\,000\,000\, 001 F & = 1 pF & = \(1 \cdot 10^{-12}\) F & (p är piko) \\
        0,000\,000\,000\,000\,001 C & = 1 fC & = \(1 \cdot 10^{-15}\) & (f är femto) \\
        0,000\,000\,000\,000\,000\, 001 C & = 1 aC & = \(1 \cdot 10^{-18}\) & (a är atto) \\
        \hline
      \end{tabular}
    \end{center}
      \caption{Prefix}
      \label{tab:prefix}
\end{table*}


\section{Flyttal}

%%  k7per: Need to extend with the koncept of significant digits, 1000 means the same as 1.000e3, but not 1e3.

En decimal talstorhet uttrycks ofta med ett så kallat tekniskt flyttal.
Decimaltecknet placeras då så att den visade tio-exponenten i talet
blir en multipel av 3.
Se exempel i ovanstående uppställning.

Decimaltecknet kan även placeras så att tioexponenten är något annat
än en multipel av 3.
Talstorheten uttrycks då med ett så kallat allmänt flyttal.

I miniräknare med mera visas ofta exponenten som bokstaven E, åtföljt av
ett värde.
Ibland utelämnas själva bokstaven medan exponentvärdet står kvar.

\noindent\textbf{Exempel:}

\begin{tabular}{rllll}
1000  & visas som & \(1    \cdot 10^3  \) & eller & 1 E+03 \\
125   & visas som & \(1,25 \cdot 10^2  \) & eller & 1,25 E+02 \\
10    & visas som & \(1    \cdot 10^1  \) & eller & 1 E+01 \\
1     & visas som & \(1    \cdot 10^0  \) & eller & 1 E+00 \\
0,1   & visas som & \(1    \cdot 10^{-1}\) & eller & 1 E-01 \\
0,01  & visas som & \(1    \cdot 10^{-2}\) & eller & 1 E-02 \\
0,012 & visas som & \(12   \cdot 10^{-3}\) & eller & 12 E-03 \\
\end{tabular}

%% \section{Metallers resistivitet}
%% \label{metallersresistivitet}

%% \begin{tabular}{l|l}
%%   Ämne & Resistivitet vid \SI{20}{\degreeCelsius} \([\dfrac{\si{\ohm} \cdot mm^2}{m}]\) \\
%%   \hline
%%   Aluminium   & 0,028 \\
%%   Bly         & 0,22  \\
%%   Guld        & 0,024 \\
%%   Järn        & 0,105 \\
%%   Koppar      & 0,018 \\
%%   Kvicksilver & 0,958 \\
%%   Nickel      & 0,078 \\
%%   Platina     & 0,108 \\
%%   Silver      & 0,016 \\
%%   Tenn        & 0,115 \\
%%   Volfram     & 0,056 \\
%%   Zink        & 0,058 \\
%% \end{tabular}


\section{Grekiska alfabetet}

  Bokstäver ur bland annat grekiska alfabetet används som symboler för
  tekniska begrepp.
  Märk, att samma symboler används olika inom olika teknikområden.
  I tabell \ref{tab:a.grekiska} anges några användningar inom elektroniken.

\begin{table*}
  \begin{center}
  \begin{tabular}{ll|l|l}
    Versaler  & Gemener   &       & \\
    ''stora'' & ''små''   &       & \\
    bokstäver & bokstäver & Uttal & Användningsexempel \\
    \hline
    \(A\) & \(\alpha\) & Alpha & \\
    \(B\) & \(\beta\) & Beta & \\
    \(\Gamma\) & \(\gamma\) & Gamma & Ledningsförmåga \\
    \(\Delta\) & & Delta & Del av .. storhet \\
    & \(\delta\) & Delta & Förlustvinkel etc. \\
    \(E\) & \(\varepsilon\) & Epsilon & Dielektricitetskonstant etc.\\
    \(Z\) & \(\zeta\) & Zeta & \\
    \(H\) & \(\eta\) & \AE ta & Verkningsgrad\\
    \(\Theta\) & \(\vartheta\) & Teta & Vinklar \\
    \(I\) & \(\iota\) & Jota & \\
    \(K\) & \(\kappa\) & Kappa & Kopplingskoefficient \\
    \(\Lambda\) & \(\lambda\) & Lambda & Våglängd \\
    \(M\) & \(\mu\) & My & Permeabilitet \\
    \(N\) & \(\nu\) & Ny & Frekvens \\
    \(\Xi\) & \(\xi\) & Xi & \\
    \(O\) & \(o\) & Omikron & \\
    \(\Pi\) & \(\pi\) & Pi & 3,14159\dots \\
    \(P\) & \(\rho\) & Rho & Resistivitet \\
    \(\Sigma\) & \(\sigma\) & Sigma & Summa \\
    \(T\) & \(\tau\) & Tau & Tidskonstant \\
    \(Y\) & \(\upsilon\) & Ypsilon &  \\
    \(\Phi\) & & Fi & Magnetiskt flöde \\
    & \(\varphi\) & Fi & Fasvinkel \\
    \(X\) & \(\chi\) & Chi & \\
    \(\Psi\) & \(\psi\) & Psi & \\
    \(\Omega\) & & Omega & Resistans \\
    & \(\omega\) & Omega & Vinkelfrekvens \\
  \end{tabular}
  \caption{Grekiska alfabetet}
  \label{tab:a.grekiska}
  \end{center}
\end{table*}
