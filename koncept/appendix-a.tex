\chapter{Måttenheter}

  Inom fysiken förekommer allt mellan mycket höga och mycket låga
  värden på frekvens, spänning, ström, resistans etc.
  I en radiomottagares antenningång är signalspänningen ofta mindre än
  \SI{0,000001}{\volt}.
  I slutsteget i en amatörradiosändare kan anodspänningen vara mer än
  \SI{2000}{\volt} och uteffekten upp till \SI{1000}{\watt}.
  I spektrum för elektromagnetiska vågor finns mycket höga frekvenser
  så som \SI{10000000000}{\hertz}.

  För att ange storheten på måttenheter används ofta ett \emph{prefix} före
  måttenheten (av latinets \emph{pre}, före och \emph{fixare}, att tillägga).
  Med prefixet anges från fall till fall vilken multiplikations- eller
  divisionsfaktor (talfaktor) som används. (Se tabell \ref{tab:prefix}.)

  I exemplen ovan blir signalspänningen \SI{1}{\micro\volt}, anodspänningen
  \SI{2}{\kilo\volt}, uteffekten \SI{1}{\kilo\watt} samt frekvensen
  \SI{10}{\giga\hertz} vilket i många fall kan vara lättare att läsa och svårare
  att misstolka.

  Märk, att enhetens sort inte har något att göra med själva prefixet.
  Nedan ges sorterna Hz, W, V, F etc. som exempel.

  Exponenter, till exempel siffran 6 i uttrycket \(10^6\), förklaras i
  bilaga \ref{potenser}.

  \begin{table*}[ht]
    \begin{center}
      \begin{tabular}{|llll|}
        \hline
        1\,000\,000\,000\,000\,000\,000 Hz & = 1 EHz & = \(1 \cdot 10^{18}\) Hz & (E är exa) \\
        1\,000\,000\,000\,000\,000 Hz & = 1 PHz & = \(1 \cdot 10^{15}\) Hz & (P är peta) \\
        1\,000\,000\,000\,000 Hz & = 1 THz & = \(1 \cdot 10^{12}\) Hz & (T är tera) \\
        1\,000\,000\,000 W & = 1 GW & = \(1 \cdot 10^9\) W & (G är giga) \\
        1\,000\,000 W & = 1 MW & = \(1 \cdot 10^6\) W & (M är mega) \\
        1\,000 W & = 1 kW & = \(1 \cdot 10^3\) W & (k är kilo) \\
        100 & & = \(1 \cdot 10^2\) & (h är hekto) \\
        10 & & = \(1 \cdot 10^1\) & (da är deka) \\
        1 & & = \(1 \cdot 10^0\) V & (\(1 = 10^0\) är grundenhet) \\
        0,1 & & = \(1 \cdot 10^{-1}\) & (d är deci) \\
        0,01 & & = \(1 \cdot 10^{-2}\) & (c är centi) \\
        0,001 V & = 1 mV & = \(1 \cdot 10^{-3}\) V & (m är milli) \\
        0,000\,001 V & = 1 \(\mu V\) & = \(1 \cdot 10^{-6}\) V & (\(\mu \) är mikro) \\
        0,000\,000\,001 F & = 1 nF & = \(1 \cdot 10^{-9}\) F & (n är nano) \\
        0,000\,000\,000\, 001 F & = 1 pF & = \(1 \cdot 10^{-12}\) F & (p är piko) \\
        0,000\,000\,000\,000\,001 C & = 1 fC & = \(1 \cdot 10^{-15}\) & (f är femto) \\
        0,000\,000\,000\,000\,000\, 001 C & = 1 aC & = \(1 \cdot 10^{-18}\) & (a är atto) \\
        \hline
      \end{tabular}
    \end{center}
      \caption{Prefix}
      \label{tab:prefix}
\end{table*}


\section{Flyttal}

En decimal talstorhet uttrycks ofta med ett så kallat tekniskt flyttal.
Decimaltecknet placeras då så att den visade tio-exponenten i talet
blir en multipel av 3.
Se exempel i ovanstående uppställning.

Decimaltecknet kan även placeras så att tioexponenten är något annat
än en multipel av 3.
Talstorheten uttrycks då med ett så kallat allmänt flyttal.

I miniräknare med mera visas ofta exponenten som bokstaven E, åtföljt av
ett värde.
Ibland utelämnas själva bokstaven medan exponentvärdet står kvar.

\noindent\textbf{Exexmpel:}

\begin{tabular}{rllll}
1000  & visas som & \(1    \cdot 10^3  \) & eller & 1 E+03 \\
125   & visas som & \(1,25 \cdot 10^2  \) & eller & 1,25 E+02 \\
10    & visas som & \(1    \cdot 10^1  \) & eller & 1 E+01 \\
1     & visas som & \(1    \cdot 10^0  \) & eller & 1 E+00 \\
0,1   & visas som & \(1    \cdot 10^{-1}\) & eller & 1 E-01 \\
0,01  & visas som & \(1    \cdot 10^{-2}\) & eller & 1 E-02 \\
0,012 & visas som & \(12   \cdot 10^{-3}\) & eller & 12 E-03 \\
\end{tabular}

\section{Metallers resistivitet}
\label{metallersresitivitet}

\begin{tabular}{l|l}
  Ämne & Resistivitet vid 20~\degree C \(\dfrac{\Omega\cdot mm^2}{m}\) \\
  \hline
  Aluminium   & 0,028 \\
  Bly         & 0,22  \\
  Guld        & 0,024 \\
  Järn        & 0,105 \\
  Koppar      & 0,018 \\
  Kvicksilver & 0,958 \\
  Nickel      & 0,078 \\
  Platina     & 0,108 \\
  Silver      & 0,016 \\
  Tenn        & 0,115 \\
  Volfram     & 0,056 \\
  Zink        & 0,058 \\
\end{tabular}


\section{Grekiska alfabetet}



  Bokstäver ur bland annat grekiska alfabetet används som symboler för
  tekniska begrepp.
  Märk, att samma symboler används olika inom olika teknikområden.
  I tabell \ref{tab:a.grekiska} anges några användningar inom elektroniken.

\begin{table*}
  \begin{center}
  \begin{tabular}{ll|l|l}
    Versaler  & Gemener   &       & \\
    ''stora'' & ''små''   &       & \\
    bokstäver & bokstäver & Uttal & Användningsexempel \\
    \hline
    \(A\) & \(\alpha\) & Alpha & \\
    \(B\) & \(\beta\) & Beta & \\
    \(\Gamma\) & \(\gamma\) & Gamma & Ledningsförmåga \\
    \(\Delta\) & & Delta & Del av .. storhet \\
    & \(\delta\) & Delta & Förlustvinkel etc. \\
    \(E\) & \(\varepsilon\) & Epsilon & Dielektricitetskonstant etc.\\
    \(Z\) & \(\zeta\) & Zeta & \\
    \(H\) & \(\eta\) & \AE ta & Verkningsgrad\\
    \(\Theta\) & \(\vartheta\) & Teta & Vinklar \\
    \(I\) & \(\iota\) & Jota & \\
    \(K\) & \(\kappa\) & Kappa & Kopplingskoefficient \\
    \(\Lambda\) & \(\lambda\) & Lambda & Våglängd \\
    \(M\) & \(\mu\) & My & Permeabilitet \\
    \(N\) & \(\nu\) & Ny & Frekvens \\
    \(\Xi\) & \(\xi\) & Xi & \\
    \(O\) & \(o\) & Omikron & \\
    \(\Pi\) & \(\pi\) & Pi & 3,14159\dots \\
    \(P\) & \(\rho\) & Rho & Resistivitet \\
    \(\Sigma\) & \(\sigma\) & Sigma & Summa \\
    \(T\) & \(\tau\) & Tau & Tidskonstant \\
    \(Y\) & \(\upsilon\) & Ypsilon &  \\
    \(\Phi\) & & Fi & Magnetiskt flöde \\
    & \(\varphi\) & Fi & Fasvinkel \\
    \(X\) & \(\chi\) & Chi & \\
    \(\Psi\) & \(\psi\) & Psi & \\
    \(\Omega\) & & Omega & Resistans \\
    & \(\omega\) & Omega & Vinkelfrekvens \\
  \end{tabular}
  \caption{Grekiska alfabetet}
  \label{tab:a.grekiska}
  \end{center}
\end{table*}
