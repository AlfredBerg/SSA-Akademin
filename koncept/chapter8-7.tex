\section{Brus och länkbudget}

\subsection{Allmänt}

Den mottagna signalens kvalitet kan ofta sammanfattas med dess signal-brus
förhållande.
För att kunna estimera det behöver man dels förstå själva länk-budgeten som
ger en uppfattning om hur stark signal man får, men även förstå de olika
bidragen av brus som sätter det effektiva brusgolvet.

\subsection{Brus}
\harec{a}{7.17}{7.17}
\harec{a}{7.18}{7.18}
\index{brus}
\index{atmosfäriskt brus}
\index{brus!atmosfäriskt}
\index{galaktiskt brus}
\index{brus!galaktiskt}
\index{termiskt brus}
\index{brus!termiskt}

Det finns flera källor till brus, atmosfäriskt brus, galaktiskt brus samt
termiskt brus.

\emph{Atmosfäriskt brus} (eng. \emph{atmospheric noise}) uppstår på grund av
blixturladdningar.
Över hela jorden sker hela tiden blixtnedslag, och dess starka impulser sprider
sig precis som radiovågor och ger en grundläggande störning i kortvågsbandet.
Atmosfäriskt brus identifierades 1925 av Karl Jansky.

\emph{Galaktiskt brus} (eng. \emph{galactic noise}) kommer huvudsakligen från
centrum av Vintergatan, och är huvudsakligen termiskt brus från den stora
ansamlingen av stjärnor i mitten av Vintergatan.
Galaktiskt brus kommer från den delen av himlen som för stunden har mitten av
Vintergatan, så det är riktningskänsligt.

Termiskt brus är mottagarens interna brus, se \ref{termisktbrus}.

\subsection{Länkbudget}
\harec{a}{7.20}{7.20}
\index{länkbudget}

För att kunna estimera den upplevda signalkvaliteten så gör man en så kallad
\emph{länkbudget} (eng. \emph{link budget}).
Länkbudgeten sammanställer hur signalstyrkan respektive brus varierar längs en
länk med dess förstärkningar och dämpningar.
I slutet av länkbudgeten kan sedan det upplevda signal-brus-förhållandet
enkelt estimeras.
En väl utförd länkbudget kan därför skapa god förståelse för länkens brister
så att förbättringar kan göras.

\subsubsection{Dominant bruskälla}
\harec{a}{7.20.1}{7.20.1}
\index{bandbrus}
\index{band noise}
\index{brus!band}
\index{interntbrus}
\index{brus!internt}

I en noggrann modell så ska alla bruskällor, från källa till mottagare,
listas, justeras för gain och sammanställas.
I praktiken så har man en dominant störkälla, typiskt bruset på bandet eller
kort \emph{bandbruset} (eng. \emph{band noise}) eller
\emph{mottagarens interna brus} (eng. \emph{receiver noise}),
varvid de övriga bidragen har liten påverkan på den totala uträkningen.
Det är därför praktiskt att fort estimera om det är brus på bandet eller
mottagarens brus som dominerar, vartefter man enbart räknar med den
dominerande bruskällan.

Som tumregel kan man säga att för kortvåg är oftast bruset på bandet
den dominerande bruskällan, medan för högre band så kommer det interna
bruset att dominera, och dämpningar i kablar bli allt mer märkbart.

\subsubsection{Signal-brus-förhållande}
\harec{a}{7.1.2}{7.1.2}
\index{signal-brus-förhållande}
\index{brus!signal-brus-förhållande}
\index{S/N}
\index{brus!S/N}

Upplevelsen av en signals kvalitet kan mätas på många sätt, dock är just
signalens brusmängd en viktig sådan relation och därför så använder man
begreppet \emph{signal-brus-förhållande} (eng.
\emph{signal to noise ratio (S/N)}).

Signal-brus förhållandet uttrycks oftast i \si{\decibel} och kan enkelt räknas
fram som skillnaden i nivå på signal och på brus, det vill säga signal minus
brus, räknat i \si{\decibel}.
Med en signalnivå på \SI{45}{\decibel} och brusnivå på \SI{22}{\decibel} har man
således \SI{+23}{\decibel} S/N.

\subsubsection{Minimal signal-brus-förhållande}
\harec{a}{7.20.2}{7.20.2}

Det är också till stor hjälp att fort etablera det \emph{minimala
signal-brus-förhållandet} (eng. \emph{minimum signal to noise ratio})
som man kan tolerera.
Genom att jämföra länkbudgeten mot detta kan man fort avgöra om det är
tillräckligt bra eller behöver ändras.

Har man för lågt signal-brus-förhållande mot minimum, så behöver man öka
förstärkningen eller oftast minska förlusterna i länkbudgeten.

\subsubsection{Minimal mottagen signalstyrka}
\harec{a}{7.20.3}{7.20.3}

Om den dominanta störkällan är det interna bruset och man har för lågt
signal-brus-förhållande, måste signalstyrkan in till mottagaren ökas
tills signalen är stark nog för att ge tillräckligt högt
signal-brus-förhållande.
Detta ger nivån för \emph{minimalt mottagen signalstyrka} (eng.
\emph{minimum receiver signal power}) som mottagaren kräver.

\subsubsection{Signaldämpning}
\harec{a}{7.1.1}{7.1.1}

Så väl kablar, filter, kopplingar och vågutbredning innebär
\emph{signaldämpning}.
Det innebär att man tappar energi i förhållande till den tillförda energin.
Förhållandet mellan uttagen och inmatad energi uttrycks oftast i form av
\si{\decibel}.
Man ska vara noga att notera dämpningen är oftast relaterad till frekvensen,
så den ska uppskattas eller mätas för den frekvens som avses.

\subsubsection{Brusfaktor}
\label{brusfaktor}
\index{brus}
\index{brusfaktor}
\index{brus!brusfaktor}
\index{noise factor}
\index{NF}
\index{brus!noise factor}
\index{brus!NF}
\index{LNA}

För högre frekvenser tenderar brus domineras av mottagarens interna brus,
samtidigt som kabelförluster börjar bli märkbara.
För sådana fall kan det vara lämpligt att installera en
\emph{lågbrusig förstärkare} (eng. \emph{low noise amplifier (LNA)}) före
mottagaren.
Även den förstärkaren har dock egenbrus som sedan förstärks.
\emph{Brusfaktor} (eng. \emph{noise factor (NF)}) ger förhållandet mellan en
förstärkares egenbrus i förhållande till det termiska bruset för ett motstånd
på dess ingång.
Brusfaktor redovisas oftast i \si{\decibel}, som varande dB över brusgolvet.

En förstärkares egenbrus kommer givetvis att förstärkas, och därför kommer
bruset på utgången vara brusfaktorn plus förstärkningen, räknat i \si{\decibel}.
Exempelvis kommer en förstärkare med \SI{4,5}{\decibel} i brusfaktor och
\SI{20}{\decibel} förstärkning ha brus på \SI{24,5}{\decibel} över brusgolvet.
En efterföljande förstärkare med \SI{10}{\decibel} brusfaktor kommer inte
signifikant bidra med brus, eftersom föregående steg har \SI{14,5}{\decibel}
högre brus än egenbruset.
För detta fall är den första förstärkaren dominant.

Eftersom kabeldämpning kan vara signifikant, så kommer signalen dämpas genom
kabeln.
Givet att vi har \SI{15}{\decibel} dämpning i kabeln, en signal
\SI{40}{\decibel} över brusgolvet och en \SI{20}{\decibel} förstärkare med
brusfaktor \SI{4,5}{\decibel}, var ska vi sätta förstärkaren?

Om förstärkaren sitter efter kabeln så kommer signalen att dämpas först i
kabeln, bruset kommer att läggas på och sedan kommer det att förstärkas
\SI{20}{\decibel}.
Det ger \SI{40}{\decibel} minus \SI{15}{\decibel} plus \SI{20}{\decibel} för
signalen, det vill säga \SI{45}{\decibel}.
För bruset får vi \SI{4,5}{\decibel} plus \SI{20}{\decibel} det vill säga
\SI{24,5}{\decibel}.
För detta fallet får vi ett signal-brus-förhållande på \SI{45}{\decibel} minus
\SI{24,5}{\decibel}, det vill säga \SI{20,5}{\decibel}, givet att det är det
interna bruset som är dominerande.

Om förstärkaren sitter före kabeln så kommer signalen först förstärkas och
sedan dämpas i kabeln.
Det ger \SI{40}{\decibel} plus \SI{20}{\decibel} minus \SI{15}{\decibel} för
signalen, det vill säga \SI{45}{\decibel}.
För bruset får vi \SI{4,5}{\decibel} plus \SI{20}{\decibel} minus
\SI{15}{\decibel} det vill säga \SI{9,5}{\decibel}.
För detta fallet får vi ett signal-brus-förhållande på \SI{45}{\decibel} minus
\SI{9,5}{\decibel}, det vill säga \SI{35,5}{\decibel}, givet att det är det
interna bruset som är dominerande.

Med detta exempel ser vi hur en länkbudget hjälper oss att få
signal-brus-förhållandet att gå från \SI{20,5}{\decibel} till
\SI{35,5}{\decibel} enbart genom att ändra placeringen av förstärkaren i
systemet.

\subsubsection{Vägförlust}
\harec{a}{7.20.4}{7.20.4}
\index{vägförlust}
\index{path loss}
\index{fresnelzon}
\index{närfält}

Den dämpning som signalen har i fri-rymds förlust, på grund av dess avtagande
fältstyrka kallas även för \emph{vägförlust} (eng. \emph{path loss}).
Fri-rymds förlusten beror förenklat på frekvens och avstånd, en enkel model
\cite[\S 19.1.2]{ARRLHDB2015}:
%%
\[L_{fs} = 32.45 + 20\log d + 20\log f\]
%%
där \(d\) är avståndet i \si{\kilo\metre} och \(f\) är frekvensen i
\si{\mega\hertz} och det ger \(L_{fs}\) är vägförlusten i \si{\decibel}.
Fri-rymds förlusten avtar med kvadraten på avståndet, det vill säga
\SI{6}{\decibel} på dubblat avstånd och det är i allmänhet den dominerande
effekten när man lämnat antennens närfält.

Ytterligare förluster kan förekomma på grund av vegetation, delvis täckt
\emph{Fresnelzon}, studsar mot jonosfär med mera.
Fresnelzonen är den zon som befinner sig inom den ellips vars form definieras
av en våglängd längre väg än direkt väg mellan sändare och mottagare.
Merparten av energin mellan två antenner rör sig i denna fresnelzon och
således så om den regionen är påverkad av hinder så kommer signalen dämpas
märkbart.

\subsubsection{Antennförstärkning och kabelförluster}
\harec{a}{7.20.5}{7.20.5}
\index{antennvinst}
\index{antenn!vinst}
\index{antennförstärkning}
\index{antenn!förstärkning}
\index{kabelförluster}

En antenns direktivitet ger antennen en \emph{antennförstärkning} (eng.
\emph{antenna gain}) då den i en viss riktning har en förmåga att ha högre
förstärkning än en enkel dipol.

Antennens förmåga att undertrycka andra signaler, till exempel som mätt med
fram-back-relationen, ger också en undertryckning av oönskade signaler och
atmosfäriskt brus.
Det kan därför vara värt att inte enbart mäta antennens förstärkning av
önskad signal, utan även räkna på dess förmåga att ta in oönskat brus och
störande signaler.

Anslutna kablar kan ha signifikant påverkan på både sänd- och mottagen
signalstyrka då \emph{kabelförlusterna} (eng. \emph{transmission line losses})
kommer dämpa signaler.
Kabelförluster beror på hur lång kabeln är, vilken kabel det är samt vid
vilken frekvens man använder.
Som regel har högre frekvenser högre dämpning.
Både storleken på kabeln och val av dielektrium påverkar förlusten i kabeln.

\subsubsection{Minsta sända signalstyrkan}
\harec{a}{7.20.6}{7.20.6}

En sändare har en varierande uteffekt, och eftersom man försöker åstadkomma en
uppskattning på sämsta signal-brus-förhållandet så är det inte maxeffekten
eller ens medel effekten som blir den intressanta, utan den \emph{minsta sända
signalstrykan} (eng. \emph{minimum transmitter power}).
Genom att använda sig av den i beräkningen på sin länk-budget så försäkrar man
sig om att länk-budgeten hanterar sämsta tänkbara fall, och för de fall som
sändaren är starkare så får man alltså bättre signal än den lägsta man
tolererar.
På detta sätt bygger man sig marginaler i beräkningen.

\subsubsection{Sammanställning av länkbudget}

En komplett länkbudget fås genom att räkna på signalstyrka respektive brusnvå
för varje steg i kedjan, genom att gå igenom alla förstärkningar och förluster
längs vägen.
När man sedan har räknat fram mottagarens upplevda signalstyrka och brusnivå så
kan man räkna fram signal-brus förhållandet, se exemplet i \ref{brusfaktor}.

För att försäkra sig om att det fungerar brukar man räkna konservativt, det
vill säga man väljer de sämsta siffrorna, till exempel minsta mottagen effekt
och minsta sända effekt.

En väl utförd länkbudget ger god förståelse över var den svaga länken är,
och genom att experimentera med olika alternativa lösningar så kan man
förstå var man ska börja göra åtgärder och var det är lönlöst eller har
ringa påverkan.
