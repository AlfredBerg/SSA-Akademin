\onecolumn
\chapter{KonCEPT litteraturförteckning}
\label{konceptlitteratur}

\noindent Litteraturförteckningen har utgått i väntan på uppdatering.
Den här tomma bilagan ligger kvar för att bibehålla bilagenumrering
gentemot tidigare varianter av andra upplagan.

\twocolumn
%% \noindent
%% Andra upplagan av KonCEPT bygger av naturliga skäl på en stor del av innehållet
%% i den första upplagan \cite{KonCEPT1}.
%% Därför är det nödvändigt att den litteratur som ligger till grund för den första
%% upplagan av KonCEPT redovisas även i den här upplagan.
%% I förekommande fall har referenser till samma eller modernare upplaga lagts
%% till.

%% \section{Litteratur}

%% The American Radio Relay League, Inc. (ARRL):
%% \emph{The ARRL Handbook for Radio Amateurs}, Seventy-First Ed.,
%% The ARRL Radio Amateurs Library. 1994. ISBN 0-87259-171-9 \cite{ARRLHDB2015}

%% Cuno, Hans. H.: \emph{Vorbereitung auf die Amateur Funk Lizenzpr\"ufnung.},
%% 16. Auflage, frech~verlag, 1993. ISBN 3-7724-5402-X

%% Deutscher Amateur Radio Club eV (DARC): \emph{Ausbildungsunterlagen},
%% DARC~Verlag, 1993. \cite{DARCaus}

%% Ekbom Lennart: \emph{Tabeller och formler N T Te}, 2. upplagan, Esselte Studium,
%% 1986. ISBN 91-24-34594-6

%% Experimenterande Danske Radioamat\o rer (EDR):
%% \emph{Vejen til sendetilladelsen}, 6. udgave, 2. oplag, EDR, 1976.
%% ISBN 87-85149-02-0

%% Follbring Tommy: \emph{Radioteknik för sändareamatörer},
%% Ljudbandsinstruktioner AB.

%% Föreningen Sveriges Sändaramatörer (SSA): \emph{Populär Amatöradio}, SSA, 1952.

%% Föreningen Sveriges Sändaramatörer (SSA):
%% \emph{Grundläggande Amatörradioteknik}, 2. upplagan, SSA, 1970.

%% Hall T; Perdeby Bo; Elmgren Bo: \emph{Fysikboken för högstadiet}, 2. upplagan,
%% Esselte Studium, 1974. ISBN 91-24-69278-6

%% Haraldsson Tore: \emph{Radioteknik för radioamatörcertifikat}, 8. upplagan,
%% Radio TV KB Haraldsson \& söner, 1989. ISBN 91-970362-1-8

%% Isännäinen Antti: \emph{Amatöörtekniikkaa Perusluokan Kursseille}, Suomen
%% Radioamatööriliitto r.y. (SRAL), 1987. ISBN 951-96056-1-4

%% Lindkvist Olle: \emph{Antenner}, Ljudbandsinstruktioner AB, 1993.
%% ISBN 99-0830753-3

%% Lindkvist Olle: \emph{Radiosändare}, Ljudbandsinstruktioner AB, 1989.

%% Lundqvist Hans; Roos Olle: \emph{Elektronik}, 2. upplagan, Esselte Studium,
%% 1982. ISBN 91-24-32173-7

%% Moltrecht Eckart K. W.: \emph{Amateurfunk Lehrgang}, Teil 1, 2. auflage,
%% frech~verlag, 1987. ISBN 3-7724-5386-4

%% Moltrecht Eckart K. W.: \emph{Amateurfunk Lehrgang}, Teil 2, 2. auflage,
%% frech~verlag, 1989. ISBN 3-7724-6387-8

%% Moltrecht Eckart K. W.: \emph{Amateurfunk Lehrgang}, Teil 3, 2. auflage,
%% frech~verlag, 1987. ISBN 3-7724-5388-0

%% Moltrecht Eckart K. W.: \emph{Amateurfunk Lehrgang}, Teil 4, 2. auflage,
%% frech~verlag, 1989. ISBN 3-7724-5389-9

%% Norsk Radio Rel\ae Liga (NRRL):
%% \emph{Radioamat\o rens ABC L\ae rebok i radioteknikk}, 1. utgave 2. opplag,
%% NRRL 1995.

%% Pietsch Hans-Joachim: \emph{Kurzwellen Amateurfunktechnik}, Franzis Verlag,
%% 1984. ISBN 3-7723-6592-2

%% Rothammel Karl: \emph{Antennenbuch}, Franckhsche Verlagshandlung, 1978.
%% ISBN 3-7723-6592-2 \cite{Rothammel2001}

%% de Sousa Pires Jorge: \emph{Electronics Handbook}, Studentlitteratur, 1989.
%% ISBN 91-44-21021-3

%% Svenska Elverksföreningen, Elektriska Installatörsorganisationen (EIO),
%% Elsäkerhetsverket, Röda Korset: \emph{Livräddning vid elskada},
%% Elsäkerhetsverkets Publikationsservice, 1996. ISBN 91-88924-00-9

%% Svenska Elverksföreningen, Elektriska Installatörsorganisationen (EIO):
%% \emph{Elkunskap för vardagsbruk}, Energikontorets Förlagsservice, 1994.
%% ISBN 91-76221-04-0

%% Svenska Elverksföreningen, Elektriska Installatörsorganisationen (EIO):
%% \emph{Händig med el}, utg.2 rev 1995:08 Energikontorets Förlagsservice.

%% Södra Vätterbygdens Amatörradioklubb (SVARK): \emph{Möt världen genom etern},
%% Föreningen Sveriges Sändaramatörer (SSA), 1993. ISBN 91-86368-08-9

%% Wallander Per: \emph{Bli sändaramatör}, 3. omarbetade upplagan,
%% PERANT Per Wallander AB, 1995. ISBN 91-86296-06-X

%% Wiberg Lennart: \emph{EL-LÄRA och RADIOTEKNIK}, textdel,
%% Föreningen Sveriges Sändaramatörer (SSA), 1990. ISBN 91-86368-05-2

%% Wiberg Lennart: \emph{EL-LÄRA och RADIOTEKNIK}, bilddel,
%% Föreningen Sveriges Sändaramatörer (SSA), 1990. ISBN 91-86368-04-4

%% ÖVSV ADXB-OE (Österrike):
%% \emph{Amateurfunk Lizenzlehrgang. Der Weg zum Amateurfunk}, 25. auflage,
%% Orbit, 1982. ISBN 3-85216-001-4
