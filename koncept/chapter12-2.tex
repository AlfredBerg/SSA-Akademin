\section{Allmänna elnätet}
\harecsection{\harec{a}{10.2}{10.2}}
\label{jordning}
\index{elnätet}

Elektrisk energi levereras till förbrukarna över transformatorstationer där
högspänning först transformeras till lågspänning.
Från transformatorstationerna förgrenas lågspänningsnätet till serviceskåp ute
i kvarter och byar.

I Sverige är fördelningstransformatorns sekundärlindningar oftast sammankopplade
till ett Y (s.k. Y- eller stjärnkoppling) där mittpunkten är jordad.

De i Sverige vanligast förekommande 3-fas lågspänningsnäten har huvudspänningen
\qty{400}{\volt} och fasspänningen \qty{230}{\volt}.
Spänningen mellan fasledarna kallas för huvudspänning och spänningen mellan
respektive fasledare och nolledaren kallas för fasspänning.

Bruksföremålen i huset ansluts oftast 1-fasigt, det vill säga mellan någon av
fasledarna och nolledaren.
Någorlunda lika belastning mellan faserna är önskvärd.
Mer effektkrävande apparater som el-pannor och spisar ansluts därför till alla
tre faserna (3-fasigt).
Amatörradioutrustningar ansluts oftast 1-fasigt.

Nybyggnad, förändring eller reparation av starkströmsanläggning,
fast anslutning av elektrisk utrustning till en starkströmsanläggning
eller att koppla loss fast ansluten elektrisk utrustning från en
starkströmsanläggning, klassas som elinstallationsarbete och får endast
utföras av person som har auktorisation som elinstallatör eller av
yrkesverksam som omfattas av ett elinstallationsföretags egenkontroll.

Om du har \emph{erforderlig kunskap} om elsäkerhet får du

\begin{itemize}
\item byta ut en elkopplare (strömbrytare) för högst \qty{16}{\ampere} \qty{400}{\volt}
\item byta ut ett anslutningsdon (vägguttag, lamputtag, stickpropp,
skarvuttag eller liknande) för högst \qty{16}{\ampere} \qty{400}{\volt}
\item byta ut en ljusarmatur i torrt icke brandfarligt utrymme i bostäder
\item utföra, ändra eller reparera en starkströmsanläggning som ingår i en
skyddsklenspänningskrets med nominell spänning om högst \qty{50}{\volt} med
effekt om högst \qty{200}{VA} och ström begränsad av säkring på högst \qty{10}{\ampere}
\item byta säkring
\item byta ljuskälla (lampa, lysrör eller liknande)
\item reparera apparater
\item reparera och tillverka apparatkablar och skarvsladdar.
\end{itemize}

\begin{center}
\begin{minipage}{0.19\columnwidth}
\Huge{\fontencoding{U}\fontfamily{futs}\selectfont\char 66\relax}
\end{minipage}
\begin{minipage}{0.7\columnwidth}
\textbf{Kom ihåg, att auktoriserad installatör ska anlitas för arbete
i fasta installationer.}
\end{minipage}
\end{center}

\smallfig[0.1]{images/cropped_pdfs/CE-mark.pdf}{CE-märke}{fig:CE-mark}

% \newpage % layout
\subsection{Radioamatören och hembyggd elektronik}
\index{hembyggd elektronik}
\index{praktiska råd för självbyggaren}
\index{CE-märkning}

Enligt \emph{radioutrustningslagen} SFS 2016:392 \cite{SFS2016:392} ska
radioutrustning som släpps ut eller tillhandahålls på marknaden inom EU ska vara
konstruerad och tillverkad så att den uppfyller föreskrivna krav, ha en
EU-försäkran om överensstämmelse och vara CE-märkt.

När CE-märket bild \ssaref{fig:CE-mark} sätts på en produkt eller en
radioutrustning så innebär det att tillverkaren eller importören intygar att
alla föreskrivna krav är uppfyllda.

I många fall har det dock vid kontroll av CE-märkt utrustning funnits brister
i elsäkerhet och elektromagnetisk kompatibilitet (EMC) villkoren för CE-märkning
har inte varit uppfyllda. Lär mer om EMC i avsnitt \ssaref{EMC-lagen}.

Som \emph{radioutrustning} räknas en elektrisk eller elektronisk produkt som
avsiktligt avger eller tar emot radiovågor för radiokommunikation eller
radiobestämning, eller en elektrisk eller elektronisk produkt som måste
kompletteras med ett tillbehör, såsom en antenn, för att avsiktligt avge
eller ta emot radiovågor för radiokommunikation eller radiobestämning.

Lagens tillämpningsområde och definitioner anger att lagen inte omfattar
radioutrustning som används av radioamatörer för amatörradiotrafik, under
förutsättning att utrustningen inte tillhandahålls på marknaden.
Radioutrustning som används av radioamatörer för amatörradiotrafik ska inte
anses tillhandahållen om det är:

\begin{itemize}
  \item radiobyggsatser som är avsedda att byggas samman och användas av
  radioamatörer
  \item radioutrustning som har modifierats av radioamatörer för att
  avvändas av radioamatörer
  \item utrustning som har konstruerats av enskilda radioamatörer för
  experimentella och vetenskapliga ändamål i samband med amatörradio.
\end{itemize}

Detta innebär att du som radioamatör, utöver vanlig elektronik, får bygga
och använda en radioutrustning.
Du är då ansvarig för att den utrustning du byggt är säker att använda och inte
orsakar störningar.
Detta innebär dock inte att du får göra följande:

\begin{itemize}
  \item Bygga en sändare för användning utanför amatörradiobanden.
  \item Modifiera en amatörradiosändare för användning utanför amatörradiobanden
  \item Modifiera en CE-märkt sändare utanför amatörradiobanden.
  \item Återställa en CE-märkt sändare till ursprunget efter modifiering till
    amatörradiosändare på amatörradiobanden.
  \item Montera avstörningsfilter inuti en CE-märkt apparat.
\end{itemize}

Radioutrustningen får vara avsedd att anslutas till en starkströmsanläggning
om utrustningen vid användning inte orsakar någon typ av skada på person
egendom eller husdjur.
Kom även ihåg att \qty{12}{\volt} från ett bilbatteri räknas som en
starkströmsanläggning.

När en elektrisk eller elektronisk apparat konstrueras eller byggs finns det
ett antal punkter som ska uppmärksammas för att apparaten ska vara säker att
använda oavsett hur den är avsedd att strömförsörjas.
Som stöd för hur en apparat kunde byggas för att uppfylla kraven gav
dåvarande SEMKO ut \emph{Praktiska råd för självbyggaren}.
Nedanstående punkter bygger på dessa praktiska råd:

\begin{itemize}
\item Höljet ska vara anpassat till apparaten och inte vara öppningsbart
  utan verktyg.

\item Höljet ska vara försett med nödvändiga ventilationshål för att
  undvika överhettning.
  Observera att spänningsförande delar inte får vara nåbara genom
  ventilationshålen.

\item Höljet får inte bli så varmt att skada kan uppstå på människa
  eller egendom.

\item Är höljet eller chassiet till en elnätsansluten apparat av ledande
  material och apparaten inte har förstärkt isolering så ska \emph{utsatta}
  delar som riskerar att spänningssättas vid fel anslutas till skyddsjord.

\item Kabeln för nätanslutning ska vara försedd med en för ändamålet lämplig
  dragavlastning som även skyddar kabeln mot nötning när den passerar höljet.

\item Komponenter i apparaten ska vara dimensionerade och godkända
  för den effekt de utvecklar och för den spänning och strömstyrka de
  ansluts till.
  \emph{Not: Ett tips är att ha god marginal vad gäller värmetålighet då det
    ger ökad livslängd och större säkerhetsmarginaler.}

\item Apparaten ska vara försedd med korrekt dimensionerad säkring
  som skydd mot kortslutning och överbelastning.

\item Elnätsansluten apparat ska vara försedd med 2-polig nätströmbrytare.

\item Spänningsförande delar i apparaten ska vara försedda med
  beröringsskydd som skyddar mot oavsiktlig beröring.

\item Komponenter i apparaten ska monteras fast och placeras på lämpliga
  inbördes avstånd så att risken för störningar, överslag, kortslutning eller
  överhettning minimeras.

\item Kablar och ledningar för starkström ska skyddas mot varma komponenter,
  nötning och skarpa kanter samt förläggas separerade från ledningar för
  klenspänning och signaler.
\end{itemize}

\begin{center}
\begin{minipage}{0.19\columnwidth}
\Huge{\fontencoding{U}\fontfamily{futs}\selectfont\char 66\relax}
\end{minipage}
\begin{minipage}{0.7\columnwidth}
\textbf{Sträva efter att alltid ansluta din apparat via vägguttag
	skyddade av jordfelsbrytare.}
\end{minipage}
\end{center}


\subsection{Strömbrytare}

Kraftförsörjningen av radiostationens apparater bör ske över en
gemensam huvudströmbrytare, som lätt kan nås.
En indikatorlampa får gärna markera att den brytaren är tillslagen och att
stationen är under spänning.
Informera familjen och övriga i din omgivning om hur den brytaren fungerar.
Det är en säkerhetsåtgärd om något skulle hända.

Apparaternas nätströmbrytare ska vara utförda för den aktuella arbetsspänningen
och ha ett godkänt utförande.

\begin{description}
\item[Vid 1-fassystem] ska nätströmbrytaren i apparaterna vara 2-polig och bryta fas-
och N-ledare, men aldrig PE-ledaren.

\item[Vid 3-fassystem] ska nätströmbrytaren vara 3-polig och bryta fasledarna, men
aldrig N-ledare och PE-ledare.
\end{description}

\begin{center}
\begin{minipage}{0.19\columnwidth}
\Huge{\fontencoding{U}\fontfamily{futs}\selectfont\char 66\relax}
\end{minipage}
\begin{minipage}{0.7\columnwidth}
\textbf{Kom ihåg, att auktoriserad installatör ska anlitas för arbete
i fasta installationer.}
\end{minipage}
\end{center}
%%\noindent\textbf{Kom ihåg, att en auktoriserad elinstallatör ska
%%  anlitas vid ingrepp i fasta elinstallationer.}

\subsection{Liten terminologi vid elinstallationer}
\begin{description}[style=nextline]
\item[Gruppcentral] Den säkringscentral som följer efter elmätaren,
  till exempel i villor och lägenheter.

\item[Gruppledningar] Ledningar efter en gruppcentral, dvs.
  ledningar till belysning, el-spisar, uttag med mera.

\item[Fasledare] En ledare som för fasspänning.

\item[Nolledare (N-ledare)] En ledare som är ansluten till elnätets så kallade
  nollpunkt (nollskena) och som normalt inte ska föra spänning till jord.

\item[Skyddsledare (PE-ledare)] De ledare i kablar och sladdar, som är
  speciellt avsedda för skyddsjordning.

\item[Bruksföremål] Ett i princip flyttbart elanslutet föremål,
  till exempel handverktyg och radioapparater.

\item[Förstärkt isolering] Vissa bruksföremål tillverkas med en så god
  isolering att de inte behöver skyddsjordas.
  Så isolerade får anslutningsledningen förses med en speciell stickpropp,
  som passar i vägguttag, såväl med som utan jorddon.
  Sådana bruksföremål är märkta med Fi-märket bild \ssaref{fig:Fi-mark} och får
  inte ändras så att de kan skyddsjordas.
\end{description}

\smallfig[0.1]{images/cropped_pdfs/Fi-mark.pdf}{Dubbel isolering, Fi-märke}{fig:Fi-mark}

Bild \ssaref{fig:Fi-mark} visar Fi-märket, symbolen som finns på all elektrisk
utrustning som har dubbel isolering.

\subsection{Färgkoder för fas, noll- och skyddsledare}

Isoleringsmaterialet omkring gruppledarna i fasta elinstallationer har
färger som fyller en viktig funktion.
Tyvärr har användningen av dessa färger ändrats flera gånger under årens lopp,
vilket skapar risker för förväxling.
Ledarnas färger och funktion får aldrig förväxlas då det kan medföra fara för
allvarlig skada genom brand, elchock eller ljusbåge.

Fasledaren har numera brun färg vid nyinstallation, men har tidigare varit
både svart, grå, vit eller röd.
N-ledaren (nollan) har numera blå färg vid nyinstallation, men har tidigare
varit både svart och vit.
Skyddsledaren (PE-ledare) med gul/grön längsgående randig färgmärkning är
alltid en skyddsjordledare och får endast användas för det ändamålet.
I äldre installationer kan emellertid skyddsledarens isolering vara till
exempel röd.

Det är till fas och N-ledarna i vägguttagen, som man kopplar apparaterna för
att få ström.
Helst ska uttagen vara i skyddsjordat utförande det vill säga med ett
jordningsbleck.
Detta bleck är anslutet till den gul/gröna ledaren för skyddsjord.

\subsection{Uttag och stickproppar med jorddon}

Jorddonet ger förbindelse med elsystemets skyddsjord (PE).
Det är tidigare rummets utförande som avgjorde om vägg- och lamputtagen skulle
ha uttag med jorddon.
Kök och tvättstugor med ledande plåtbänkar, vattenkranar och så vidare anses
som riskfyllda rum och måste ha uttag med jorddon.
Samma gäller källare och liknande andra rum med ledande golv, väggar och
inredningar.
Bostadsrum var klassade som inte särskilt riskfyllda och har därför tidigare
inte försetts med lamp- och vägguttag med jorddon.

Vid nybyggnation är emellertid numera alla uttag är av skyddsjordat utförande!
Det rekommenderas att installera skyddsjordade vägguttag för radiostationen.
Observera då, att alla uttag i det rummet ska vara skyddsjordade!

\subsection{Skyddsjordning}

Att jorda är det vanliga uttrycket för att ansluta ett föremål till skyddsjord.
Men uttrycket används även lite slarvigt i andra fall utan att syfta på
skyddsjordning av elsäkerhetsskäl.

Metallhöljen på elektrisk utrustning kan av olika anledningar bli
spänningsförande och är då en elsäkerhetsrisk.
För att minska risken för farlig spänningssättning av metallhöljet ansluts
höljet till skyddsjord.

\begin{center}
\begin{minipage}{0.19\columnwidth}
\Huge{\fontencoding{U}\fontfamily{futs}\selectfont\char 66\relax}
\end{minipage}
\begin{minipage}{0.7\columnwidth}
Om det blir isolationsfel mellan en strömförande del och höljet kommer
säkringen att bryta strömtillförseln och risken för skada minskar.
\textbf{PE-ledaren får därför aldrig brytas!}
\end{minipage}
\end{center}


\noindent\emph{För skyddsjordning finns särskilda föreskrifter.
  Kontakta därför en auktoriserad elinstallatör.}

\subsection{Jordfelsbrytare}
\index{jordfelsbrytare}

Jordfelsbrytare är en automatisk strömbrytare som snabbt bryter strömmen
då strömmen till och från en apparat är olika.
Detta kan inträffa vid ett jordfel eller vid överledning i en skyddsjordad
apparat eller i andra fall när inkommande ström och utgående ström genom
jordfelsbrytaren inte är lika stora.
Jordfelsbrytaren kan skydda dig:

\begin{itemize}
\item vid isolations- och jordfel
\item om chassiet på en apparat blir strömförande
\item om du kommer åt spänningsförande delar och jord samtidigt
\item om vägguttagen saknar skyddsjord
\item om du använder en apparat på ett felaktigt sätt i våtutrymmen
\item om du installerat en apparat på att felaktigt sätt
\item om apparatens kabel skadats
\item mot och minimera risken för brand.
\end{itemize}

Jordfelsbrytaren \textbf{skyddar inte} för strömmar som går genom fasledare
och neutralledare eller genom fas till fasledare (3-fas).

Jordfelsbrytare får inte ersätta skyddsjordning, men kan under särskilda
förutsättningar komplettera skyddsjordningen som en extra säkerhetsåtgärd.
Vid nyinstallation av bostäder är det numera krav på att minst en
jordfelsbrytare ska installeras.
Beställ gärna installation av jordfelsbrytare i äldre anläggningar!

\subsection{Särjordning}
\index{särjordning}

Särjordning är ett uttryck för att jorda apparater till en separat jordpunkt,
det görs via separat jordlina till ett jordtag, det vill säga jordplåt eller
jordspett.
Särjordning ska ske på rätt sätt eftersom det avsedda skyddet annars kan bli en
fara.

\emph{Om du har planer på särjordning, fråga en auktoriserad installatör.}

\subsection{Jordning av antennsystem}

I brist på annan jordpunkt är det frestande att ansluta antennjordledaren till
PE-ledarens anslutningsbleck i vägguttaget eller till ett värmeelement med
förhoppning att på så sätt få ett bättre HF-jordplan för antennen.
Detta är emellertid ett dåligt exempel på särjordning, som både kan innebära
säkerhetsrisker och medföra störningsproblem.

\subsection{Snabba och tröga säkringar}
\index{säkringar}

Det finns snabba och tröga säkringar.
Snabba säkringar är det som normalt används.
Tröga säkringar för samma strömstyrka kan behövas för apparater som har
speciellt hög startström, till exempel stora nättransformatorer med toroidkärna.

Säkringarna ska kunna bryta tillräcklig hög spänning, annars blir det
en kvarstående ljusbåge i dem vid säkringsbrott.
Använd säkringar med rätta strömvärden och välj en säkring med lite marginal
till belastningsströmmen så att säkringen inte löser ut under normal drift.

Det är förbjudet att laga säkringar då det kan orsaka brand.
