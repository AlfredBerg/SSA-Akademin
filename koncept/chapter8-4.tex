\section[Solens inverkan]{Solens inverkan på jonosfären}
\harec{a}{7.5}{7.5}
\label{solens_inverkan_jonosfären}

\subsection{Solaktivitet}
\index{solaktivitet}
\index{solaktivitet!korona}
\index{solaktivitet!solfläckar}
\index{solaktivitet!flares}

Solen är ett gasklot, i vars inre pågår en ständig kärnreaktion där
väteatomer omvandlas till helium.
Vid denna process frigörs en del av solmaterian som partikelstrålning och
elektromagnetisk strålning inom ett brett frekvensregister, bland annat
kortvågig radiostrålning och gammastrålning.
Solatmosfärens yttre består av två skikt, kromosfären och \emph{koronan}.
Vissa områden på solens yta har en lägre temperatur och uppfattas som mörka
fläckar -- \emph{solfläckar}.
Från kromosfären kastas det ut gasmassor, så kallade protuberanser, ofta från
områden nära solfläckarna.

Det förekommer även kortvariga eruptioner, så kallade \emph{flares}, som syns
som lysande fläckar i närheten av solfläckarna.
Flares sänder ut stark elektromagnetisk strålning och partiklar.
Koronan är solatmosfärens yttersta skikt.
Från denna utstrålas partiklar i form av atomer, elektroner och protoner, som
fångas upp av jordens magnetfält och skapar polarsken, så kallad aurora.
Den ökade partikelstrålningen från flares kan orsaka magnetiska oväder med
åtföljande radiostörningar och ökning av polarskenet.
Antalet synliga solfläckar står i samband med solaktiviteten.

\subsection{Solfläckstal}
\index{solaktivitet!solfläckstal}
\index{solaktivitet!11-årscykel}
\index{solaktivitet!solflux}
\index{solfläckstal}
\index{solflux}

Ett mått på solaktiviteten är antalet solfläckar, vilket det görs
fortlöpande observationer på.
Ur detta statistikmaterial beräknas ett vägt solfläckstal \(R\) (Wolf-talet).
Med stöd av solobservationer under mer än 200~år har det kunnat fastställas att
solfläckstalet varierar någorlunda periodiskt mellan ungefär 200 och 5.
En solfläcksperiod varar mellan cirka 7,5 och 17~år, med ett medelvärde
av cirka 11~år -- den så kallade 11-årscykeln.

I december 2008 inleddes cykel 24 sedan observationerna av dem påbörjades.
När cykel 25 börjar betyder det bättre möjligheter till DX på kortvåg under
några år.

På senare tid har ännu en metod börjat användas för mätning av solaktiviteten.
Då mäts styrkan av radiobruset från solen (solflux \(F\)) i våglängdsområdet
\SI{10}{\centi\metre}.
Båda mätmetoder ger i huvudsak samma tendenser och det finns ett
statistiskt samband mellan dem.

Vågutbredningen i jonosfären påverkas av solaktiviteten.
Under solfläcksmaximum blir jonosfären starkt joniserad, speciellt F-skiktet
under dagtid.
Då reflekteras även vågor med kortare våglängder mot jonosfären i stället för
att passera igenom denna ut i rymden.
20-metersbandet är då ''öppet'' nästan dygnet runt, 15-metersbandet från före
gryningen till efter solnedgången och 10-metersbandet nästan varje dag till
efter solnedgången.
Långa förbindelser med mycket låga effekter är möjliga.

Under solfläcksminimum är det emellertid nödvändigt att använda avsevärt lägre
arbetsfrekvens än vid solfläcksmaximum.
20-metersbandet förblir till exempel inte öppet under hela natten.
Öppningar på 15-metersbandet uppstår endast tillfälligtvis och öppningar på
10-metersbandet är sällsynta.
Goda antenner och högre effekter används då för att i någon mån kompensera den
sämre vågutbredningen.
Vid låg solaktivitet kan de högre banden vara så tysta, att operatören kan
undra om utrustningen verkligen fungerar.
