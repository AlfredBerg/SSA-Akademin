\section{Radiovågornas egenskaper}
\label{radiovågornasegenskaper}
\index{radiovågor!egenskaper}

\subsection{Radiovågors utbredning}
\harec{a}{7.2}{7.2}
\index{radiovågor!utbredning}

Ett elektromagnetiskt fält som alstras i ett givet tidsmoment breder
ut sig åt alla håll i rymden likt en ständigt växande sfär.

Fältstyrkan inom ett givet avsnitt av sfärens yta sjunker därför
alltefter som avståndet från sändaren ökar.
Det är därför som en sändare hörs svagare ju mera avlägsen den är ifrån
mottagaren.
Jämför med ljuset från en rundstrålande lampa.

I rymden breder radiovågor ut sig mycket långt. Det uppstår dock även
där utbredningsförluster i materia som finns i vägen.

När radiovågorna passerar genom jordatmosfärens olika skikt uppstår mycket
större utbredningsförluster än i rymden och därmed blir räckvidden kortare.

Elektromagnetiska fält från alla slags sändare (emittörer) genomkorsar alla
slags material och alstrar strömmar i de material som är elektriskt ledande.

\paragraph{Radiovågorna}
\begin{itemize}
  \item breder ut sig rätlinjigt i alla riktningar i rymden med ljusets
  hastighet som är cirka \SI{300000}{\kilo\metre\per\second} (se även avsnitt
  \ref{ljushastigheten})
  \item tränger igenom fasta kroppar, som inte är elektriskt ledande
  \item dämpas eller reflekteras, bland annat av metaller, joniserade vätskor
  och joniserade atmosfärskikt
  \item är polariserade
  \item förstärker eller motverkar varandra.
\end{itemize}

\paragraph{Radiovågorna breder ut sig}
\begin{itemize}
  \item utmed jordytan
  \item upp från jordytan
  \item upp från jordytan efter en första reflexion mot denna.
\end{itemize}
Det första sättet kallas för markvåg och de två senare kallas med ett
samlingsbegrepp för rymdvåg.

\subsection{Böjning av radiovågor}
\harec{a}{7.11}{7.11}
\index{take-off vinkel}
\index{NVIS}
\index{radiovågor!böjning}
\index{radiovågor!dämpning}

\paragraph{Radiovågornas riktning kan böjas av genom}
\begin{itemize}
  \item reflexion eller splittring mot naturliga reflektorer i
  atmosfären och i jordytan
  \item konstgjorda såväl passiva som aktiva reflektorer (relästationer)
  på jordytan och i rymden.
\end{itemize}

\paragraph{Radiovågorna kan dämpas}
\begin{itemize}
  \item i jordytan
  \item i topografin
  \item i atmosfärsskikten.
\end{itemize}

En antenns höjd påverkar riktningen av vågen, på grund av reflektion mot
marken, eller snarare den fuktiga delen lite under ytan.
Denna reflekterade våg är fasförskjuten genom den längre sträckan den går,
och den riktning där reflektionen samverkar med en, eller flera, hel våglängds
fördröjning kommer bli den riktning som antennen har bäst effektiv strålning.

Ju högre en antenn sitter, ju lägre så kallad take-off vinkel har den.
Därför är en antenn som sitter lågt i förhållande till sin våglängd riktad i
huvudsak uppåt, vilket gör att den reflekterar mot jonosfären och ned i
närområdet, detta kallas för Near Vertical Incidence Skywave (NVIS).

En antenn som sitter högt får en lägre vinkel, vilket lämpar sig väl för att
sända långväga, då en eller få studsar behövs.

Vågutbredningens natur är mycket sammansatt och kan inte enkelt beskrivas.
Några starkt påverkande faktorer på vågutbredningen kan ändå urskiljas, till exempel
\begin{itemize}
  \item utbredningsvägens höjd över jordytan
  \item radiovågens frekvens
  \item solstrålningens jonisering av jordatmosfären
  \item väderförhållandena.
\end{itemize}

\subsection{Olika slags vågavböjning}

Olika faktorer påverkar vågutbredningen inom olika avsnitt i
frekvensspektrum. Här följer de viktigaste:

\subsubsection{Reflexion}
\index{radiovågor!reflexion}
\index{reflektor}

Reflexion innebär att vågorna böjs tillbaka från den yta som de träffar.
Ljus- och radiovågor reflekteras på samma villkor eftersom att båda är
elektromagnetiska till sin natur.
Den stora skillnaden är vågfrekvensen.

Reflektorns storlek uttrycks i termer av antal våglängder vid den
aktuella frekvensen.
En 80-metersvåg reflekteras inte bra mot en yta med bara någon meters sida.
Däremot reflekteras en 2-metersvåg mycket bättre mot en lika stor yta och en
ljusvåg (med våglängderna \SIrange{440}{740}{\nano\metre}) ojämförligt mycket bättre.

Olika materials förmåga att reflektera en infallande radiovåg beror av
vågens frekvens samt av materialets tjocklek och elektriska ledningsförmåga.
Vågen tränger djupare in i materialet vid låg frekvens respektive vid låg
ledningsförmåga.

\subsubsection{Refraktion}
\index{radiovågor!refraktion}

Refraktion (brytning) innebär att vågen ändrar riktning, när den passerar
gränsen mellan två media eller material med olika ledningsförmåga.
När ledningsförmågan ändras successivt till exempel i ett atmosfärskikt, blir
vågens avböjning mjuk.
Refraktion sker exempelvis när man tittar på ett föremål under vattnet där den
skenbara positionen kan avvika från den riktiga vilket man märker om man
sträcker ned handen i vattnet.

\subsubsection{Diffraktion}
\index{radiovågor!diffraktion}

Diffraktion innebär att vågens infallsriktning splittras upp i flera nya
riktningar, när vågen passerar nära över ett hinder.
Det är på grund av detta fenomen som radiosignaler i viss mån kan höras även
bortom en bergrygg.
Diffraktionen tilltar med minskande frekvens.
Det finns några olika typer av diffraktion för radiovågor varav en av de
viktigaste i terrängen benämns ''knivseggsdiffraktion'' och bryter radiovågen
ned mot marken.
