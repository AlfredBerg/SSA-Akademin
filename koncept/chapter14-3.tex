\section{Svensk lag och föreskrift}
\label{svensk lag och föreskrift}

Lagar, föreskrifter och anvisningar tillämpas för
  amatörradioanvändning.
Märk, att ändringar kan förekomma.
\textbf{Använd därför aktuella versioner!}


\subsection{Lag om elektronisk kommunikation m.fl.}
\harec{c}{3.1}{3.1}
\index{LEK}

\emph{Lag (2003:389) om elektronik kommunikation} (LEK) \cite{SFS2003:389}
reglerar all radiokommunikation Sverige.
Tillstånd behövs för all radiosändning som inte är undantagen tillståndsplikt.

\emph{Post- och telestyrelsen} (PTS) är enligt förordning (2003:396) om
elektronisk kommunikation den svenska myndighet som handlägger ärenden gällande
telekommunikation.
PTS ska bland annat svara för att möjligheterna till radiokommunikationer
utnyttjas effektivt och har därvid att beakta den internationella regleringen
inom området.
Regleringen av amatörradioanvändningen begränsas nu till den minsta omfattning
som följer av internationella avtal och europeiska rekommendationer,
CEPT-rekommendationer.

\subsection{Post- och telestyrelsens föreskrifter om undantag från tillståndsplikt för användning av vissa radiosändare}
\harec{c}{3.2}{3.2}
\index{amatörradiocertifikat}
\index{amatörradiosändare}
\index{amatörradiotrafik}
\index{antennvinst}
\index{antennförstärkning}
\index{ERP}
\index{effektivt utstrålad effekt (ERP)}
\index{antenn!effektivt utstrålad effekt (ERP)}
\index{Effective Radiated Power (ERP)}
\index{antenn!Effective Radiated Power (ERP)}
\index{ERP}
\index{antenn!ERP}
\index{ekvivalent isotropiskt utstrålad effekt (EIRP)}
\index{antenn!ekvivalent isotropiskt utstrålad effekt (EIRP)}
\index{Equivalent Isotropically Radiated Power (EIRP)}
\index{antenn!Equivalent Isotropically Radiated Power (EIRP)}
\index{EIRP}
\index{antenn!EIRP}
\index{PEP}
\label{PTSFS2018:3}
\index{T/R 61-02}
\index{PTSFS 2018:3}

Post- och telestyrelsen föreskriver i PTSFS 2018:3 \cite{PTSFS2018:3} med stöd
av 12\S förordningen (2003:396) \cite{SFS2003:396} om elektronisk kommunikation
att användningen av amatörradiosändare är undantagen tillståndsplikt.
Notera att PTS med viss regelbundenhet uppdaterar undantagsföreskrifterna,
och därför bör man kontrollera på PTS webbplats vad som är den senaste versionen
och använda den när den trätt i kraft.

Den som använder en amatörradiosändare ska ha ett amatörradiocertifikat.
För att få ett amatörradiocertifikat krävs kunskaper i enlighet med Annex~6 i
CEPT rekommendation T/R~61-02 \cite{TR6102}, examinering för
amatörradiocertifikat.

Undantag från kravet på amatörradiocertifikat gäller för den som under en
tidsbegränsad period utbildar sig för att få ett sådant certifikat och för
den som under en förevisning tillfälligt använder amatörradiosändare, under
förutsättning att användningen av radiosändaren sker under uppsikt av en
innehavare av amatörradiocertifikat.
(Läs mer om användningen i avsnitt \ref{secondoperator})

Den som innehar amatörradiocertifikat ska ha en egen anropssignal.
Denna framgår av certifikatet, eller tidigare av amatörradiotillståndet.

I undantagsföreskriften \cite{PTSFS2018:3} finns följande definitioner som är
relevanta för amatörradiotjänsten:

\begin{description}
\item[amatörradiocertifikat] kunskapsbevis utfärdat eller godkänt av
Post- och telestyrelsen, som utvisar att godkänt kunskapsprov avlagts.

\item[amatörradiosändare] radiosändare som är avsedd att användas av personer
som har amatörradiocertifikat, för sändning på frekvenser som är avsedda för
amatörradiotrafik.

\item[amatörradiotrafik] icke yrkesmässig radiotrafik för övning,
kommunikation och tekniska undersökningar, bedriven i personligt radiotekniskt
intresse och utan vinstsyfte.

\item[antennvinst] förstärkning i förhållande till en referensantenn som
antingen är isotropisk eller en dipol och som mäts i dBi eller dBd.
Antennvinsten anger hur bra riktverkan en antenn har.

\item[e.i.r.p.] equivalent isotropically radiated power (ekvivalent
isotropiskt utstrålad effekt).

\item[e.r.p.] effective radiated power (effektivt utstrålad effekt relativt en
halvvågsdipol).

\item[p.e.p.] peak envelope power.
\end{description}
%%
Vidare anges ytterligare villkor i kapitel 3 \S 14 av undantagsföreskriften
\cite{PTSFS2018:3}:
%%
\begin{quote}
De tekniska egenskaperna hos amatörradiosändaren ska anpassas så att de inte
stör användningen av andra radioanläggningar.
Den som använder en amatörradiosändare ska ha ett amatörradiocertifikat.
För att få ett amatörradiocertifikat krävs kunskaper i enlighet med Annex 6 i
CEPT Rekommendation T/R 61-02, Examinering för amatörradiocertifikat,
Vilnius 2004, version 4 oktober 2011.11 \cite{TR6102}.
\end{quote}
%%
Undantag från kravet på amatörradiocertifikat gäller för den som under en
tidsbegränsad period utbildar sig för att få ett sådant certifikat och för den
som under en förevisning tillfälligt använder amatörradiosändare, under
förutsättning att användningen av radiosändaren sker under uppsikt av en
innehavare av amatörradiocertifikat.

Den som innehar amatörradiocertifikat ska ha en egen anropssignal.
Denna framgår av certifikatet, eller tidigare av amatörradiotillståndet.
Mottagare- och sändarestationens anropssignaler ska sändas i början och i
slutet av varje radioförbindelse.
Anropssignalerna ska också upprepas med korta mellanrum under pågående
radioförbindelse. Under de utbildnings- och förevisningstillfällen som anges i
stycket ovan ska anropssignal användas som tillhör den innehavare av
amatörradiocertifikat som har uppsikt över användningen av radiosändaren.
Vid dessa tillfällen får även anropssignal som tillhör den amatörradioförening
eller institution som anordnar utbildnings- eller förevisningstillfället
användas om företrädare för föreningen eller institutionen har uppsikt över
användningen av radiosändaren.

Automatiska amatörradiosändare, till exempel en radiofyr, repeater eller
sändare för positionering ska alltid kunna identifieras genom att en
anropssignal regelbundet sänds med morsetelegrafi, röstmeddelande eller
på annat sätt.
Anropssignalen ska ange vem som är ansvarig för den automatiska sändaren.
Den som startar eller använder automatiska amatörradiosändare ska ha eget
amatörradiocertifikat och ska använda egen anropssignal.
Sådan start och användning får även utföras av den som inte har
amatörradiocertifikat, om det sker under uppsikt av en innehavare av
amatörradiocertifikat och dennes anropssignal används.

\subsection{Litteraturhänvisning om lagar och föreskrifter}

\begin{itemize}
\item CEPT rekommendation T/R 61-01 \cite{TR6101}
\item CEPT rekommendation T/R 61-02 \cite{TR6102}
\item Lag (2003:389) om elektronik kommunikation \cite{SFS2003:389}
\item Förordning (2003:396) om elektronisk kommunikation \cite{SFS2003:396}
\item Post- och telestyrelsens föreskrifter om undantag från tillståndsplikt för
användning av vissa radiosändare PTSFS 2018:3 \cite{PTSFS2018:3}
\end{itemize}
