\section{Mottagningskonvertern}
\index{mottagningskonverter}
\index{konverter}
\index{mottagare!konverter}

\mediumfig{images/cropped_pdfs/bild_2_4-18.pdf}{Mottagningskonverter UHF till KV}{fig:bildII4-18}

Konverter betyder i detta sammanhang frekvensomvandlare.
När det är önskvärt att flytta över alla signalerna inom ett helt frekvensområde
till ett annat, så används en mottagningskonverter där frekvensblandning och
frekvensfilter används, så som illustreras i bild \ref{fig:bildII4-18}.

Konvertern fungerar som tillsats före en mottagare för att denna även
ska kunna användas inom ett annat frekvensområde.
I en konverter är oscillatorfrekvensen fast, medan avsökningen av
frekvensområdet görs med VFO i mottagaren.
Mellanfrekvensfiltret i mottagaren är så brett som hela det frekvensområde
som tas emot av konvertern och avsöks med mottagaren.

\textbf{Exempel:}
I en KV-mottagare för området \SIrange{28}{30}{\mega\hertz} vill man även kunna
lyssna i området \SIrange{432}{434}{\mega\hertz} (UHF).
Den i konvertern mottagna UHF-signalen förstärks för att sedan blandas med
\SI{404}{\mega\hertz}, en frekvens som multiplicerats upp från en
kristalloscillator (CO) i konvertern.
De blandningsprodukter som filtreras fram kommer att ligga inom området
\SIrange{28}{30}{\mega\hertz} och kan alltså avlyssnas i KV-mottagaren.
Övriga blandningsprodukter blir undertryckta i KV-mottagarens ingångskretsar.
Blandningsfrekvensen \SI{404}{\mega\hertz} i konvertern är beräknad på följande
sätt:

\noindent
Mittfrekvensen i UHF-bandet är
%%
\[\frac{432+434}{2} = 433\text{ MHz } = f_1\]
%%
Mittfrekvensen i KV-mottagarens frekvensband är
%%
\[\frac{28 + 30}{2} = 29\text{ MHz}\]
%%
Med vilken frekvens \(f_2\) måste \SI{433}{\mega\hertz} blandas för att erhålla
en blandningsprodukt av \SI{29}{\mega\hertz}?
\SI{29}{\mega\hertz} är mindre än \(f_1\) , alltså kan endast
skillnadsfrekvensen komma i fråga (vid summafrekvens skulle blandningsfrekvensen
bli högre än \SI{433}{\mega\hertz}).
Vid användning av skillnadsfrekvensen ges två möjligheter:
%%
\begin{gather*}
  \text{för }f_2 - f_1 = f_2 - 433 = 29\text{ MHz är }f_2 = 462\text{ MHz} \\
  \text{för }f_1 - f_2 = 433 - f_2 = 29\text{ MHz är }f_2 = 404\text{ MHz}
\end{gather*}
%%
Vi bestämmer oss för alternativet \SI{404}{\mega\hertz} av ett speciellt skäl.
Här motsvaras den högsta UHF-frekvensen 434~MHz av \(434 - 404 = 30\)~MHz
och den lägsta UHF-frekvensen 432~MHz av \(432 - 404 = 28\)~MHz.
På så sätt kan kHz-graderingen på KV-mottagarens skala användas direkt utan
omräkning.

Fördelen med en konverter är att kostnaden för en sådan är låg jämfört
med den för en komplett mottagare för ett tillkommande band.
Förutsättningen är att en mottagare redan finns.
Nackdelen är att mottagaren inte samtidigt kan användas för sin
ordinarie funktion.
