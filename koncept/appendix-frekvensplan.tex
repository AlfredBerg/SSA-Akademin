\chapter{Svensk frekvensplan}
\label{svensk frekvensplan}

Varje lands teleadministration utfärdar föreskrifter för hur amatörradio får
användas i landet.
Dessa föreskrifter bygger på de internationella överenskommelserna i ITU-RR
\cite[ARTICLE 5]{ITU-RR} om hur frekvenserna och radiospektrum ska används för
att minska störningarna mellan olika tjänster och länder.

Överenskommelserna leder fram till en frekvensplan som förutom frekvensband även
redovisar för vilka tjänster som har primär status och därvid har företräde före
tjänster med lägre status.
Observera att flera tjänster kan ha delad primär status i ett band, som till
exempel i frekvensbanden \SIrange{3500}{3800}{\kilo\hertz} och
\SIrange{432}{438}{\mega\hertz}.

Med stöd av SFS~2022:511 har Post- och telestyrelsen (PTS)
\emph{Förordning om elektronisk kommunikation} \cite{SFS2022:511}
publicerat PTSFS~2019:1 \emph{Allmänna råd om den svenska frekvensplanen}
\cite{PTSFS2019:1}.
Notera att PTS med viss regelbundenhet uppdaterar undantagsföreskrifterna,
och därför bör man kontrollera på PTS webbplats vad som är den senaste versionen
och använda den när den trätt i kraft.

Bilaga 1 till de allmänna råden utgör den svenska frekvensplanen som reglerar
vilka frekvenser som är upplåtna i Sverige samt vilka tjänster som kan nyttja
frekvenserna och vilka tjänster som har primär respektive sekundär status inom
frekvensbanden.

\newpage

De allmänna råden och den svenska frekvensplanen omfattar även av
EU-kommissionen bindande genomförandebeslut gällande effektiv användning av
radiospektrum och villkor för inom EU harmoniserade frekvensband.

Detaljregleringen gällande amatörradio i Sverige som bygger på ovanstående
överenskommelser, förordningar och föreskrifter sker sedan i PTSFS~2020:5
\emph{PTS föreskrift om undantag från tillståndsplikt för användning av vissa
  radiosändare} \cite{PTSFS2022:19}.
I föreskriften anges de frekvensband som i Sverige är tilldelade för
amatörradio, och under vilka villkor frekvensbanden får användas för
amatörradio.

Det är denna så kallade undantagsföreskrift som strikt reglerar vad som är
tillåtna frekvensband och uteffekter för amatörradio i Sverige.

Denna föreskrift har naturligtvis företräde över IARU:s bandplaner, vilka
endast är rekommendationer för hur tilldelade frekvensband bör disponeras.

I tabell \ssaref{frekvensplan} som bygger på PTSFS 2019:1 och PTSFS 2022:19 visas
vilka frekvensband som är upplåtna för amatörradio i Sverige, maximal uteffekt
och om amatörradio har primär eller sekundär status i frekvensbandet.

I och med införandet av PTSFS 2020:5 begränsades den maximala uteffekten på
frekvensbanden för amatörradio till \qty{200}{\watt} \pep.
PTS införde samtidigt en möjlighet att för enskilda platser ansöka om tillstånd
för högre effekt.

Mer information gällande ansökan om tillstånd för högre effekt finns på PTS
webbplats under rubriken tillstånd för amatörradio.

Observera att det inte går att ansöka om tillstånd för högre effekt på alla
frekvensband. Detta har markerats i tabellens kolumn Högeffekt där maximal
effekt på dessa frekvensband är angiven.

\begin{table*}[b!]
  \centering
\caption{Frekvensband för amatörradio i Sverige}
\label{frekvensplan}
\begin{tabular}{clr|rrl|l}
Frekvensband &  & band & Effekt & Högeffekt & & Amatörradio\\ \hline
135,7--137,8 & kHz & 2200~m & 1~W & 1~W & \erp & sekundär\\
472--479 & kHz & 600~m & 1~W & 1~W & \eirp & sekundär\\
1810--1850 & kHz & 160~m & 200~W & & \pep & primär\\
1850--1900 & kHz & 160~m & 10~W & 10~W & \pep & sekundär\\
1900--1950 & kHz & 160~m & 100~W & 100~W & \pep & sekundär\\
1950--2000 & kHz & 160~m & 10~W & 10~W & \pep & sekunder\\
3500--3800 & kHz & 80~m  & 200~W & & \pep & primär\\
5351,5--5366,5 & kHz & 60~m & 15~W & 15~W & \eirp & sekundär\\
7000--7200 & kHz & 40~m  & 200~W & & \pep & primär\\
10100--10150 & kHz & 30~m & 150~W & 150~W & \pep & sekundär\\
14000--14350 & kHz & 20~m & 200~W & & \pep & primär\\
18068--18168 & kHz & 17~m & 200~W & & \pep & primär\\
21000--21450 & kHz & 15~m & 200~W & & \pep & primär\\
24890--24990 & kHz & 12~m & 200~W & & \pep & primär\\
28000--29700 & kHz & 10~m & 200~W & & \pep & primär\\
50000--52000 & kHz & 6~m & 200~W & 200~W & \pep & sekundär\\ \hline
144--146 & MHz & 2~m & 200~W & & \pep & primär\\
432--438 & MHz & 70~cm & 200~W & & \pep & primär\\
1240--1300 & MHz & 23~cm & 200~W & & \pep & sekundär\\
2400--2450 & MHz & 11~cm & 100~mW & 100~mW & \pep & sekundär\\
5650--5850 & MHz & 5~cm & 200~W & & \pep & sekundär\\
10,0--10,5 & GHz & 3~cm & 200~W & & \pep & sekundär\\
24,00--24,25 & GHz & 11~mm & 200~W & & \pep & pri/sek\\
47,0--47,2 & GHz & 6~mm & 200~W & & \pep & primär\\
75,5--81,0 & GHz & 4~mm & 200~W & & \pep & pri/sek\\
122,25--123,00 & GHz & 2~mm & 200~W & & \pep & sekundär\\
134--141 & GHz & 2~mm & 200~W & & \pep & pri/sek\\
241--250 & GHz & 1~mm & 200~W & & \pep & pri/sek\\
\end{tabular}
\end{table*}
