\newpage
\section{Magnetiskt fält}
\harec{a}{1.4}{1.4}
\label{elektromagnetiskafält}
\index{elektromagnetiska fält}

\subsection{Magnetism}
\index{magnetism}
\index{Plinius}
\index{Magnes}
\index{Lithos herakleia}
\index{Herakleia}
\index{Magnesia}
\index{Magnetes}

\begin{quote}
\emph{Enligt den romerske författaren \emph{Plinius} lär, vid tiden ungefär
160~år f.Kr. herden \emph{Magnes} en dag ha känt hur järnstiften i
sandalerna häftade vid en viss sorts sten.
Det kunde ha varit svart järnmalm, som grekerna i äldsta tider benämnde
\emph{Lithos herakleia} efter staden \emph{Herakleia} i Lydien,
där sådan malm förekommer.
Staden fick sedermera namnet \emph{Magnesia} och man kan tänka sig att stenen
kom att kallas \emph{Magnetes}.
En hel mineralgrupp med liknande egenskaper, såsom järn, nickel m.fl. kallas
magnetiska.}
\end{quote}

\emph{Magnetism} uppstår av elektriska laddningar i rörelse.
Elektronernas rörelser i en atom skapar nämligen magnetfält.
Det gör att atomerna var för sig fungerar som en magnetisk dipol -- en magnet.
I de flesta material är atomerna orienterade så att deras magnetiska krafter
tar ut varandra.
Materialet som helhet är då omagnetiskt och utövar inga yttre krafter.
Men vid påverkan från ett yttre magnetfält kan dipolerna (atomerna) i ett
material orienteras i samma riktning och deras magnetfält kommer då att
samverka. Hela materialet blir då magnetiskt.
När det yttre magnetfältet avlägsnas, kvarstår orienteringen endast delvis --
\emph{magnetisk remanens}.
I ferromagnetiska legeringar kvarstår en större del av orienteringen, även om
påverkan från det yttre magnetfältet har upphört.
Materialet är då permanentmagnetiskt.

\smallfig{images/cropped_pdfs/bild_2_1-07.pdf}{Kraftfält omkring magneter}{fig:BildII1-7}

\newpage
\subsection{Kraftfält i och omkring magneter}

Bild \ref{fig:BildII1-7} visar kraftfält omkring magneter.
Varje magnet omges av ett magnetiskt kraftfält.
Magnetfältets fördelning, styrka och riktningar beskrivs som kraftlinjer med
slutna kretslopp.

Utanför magneten går kraftlinjerna från nord- till sydpol och inne i magneten
motsatt riktning.
Kraftriktningen i varje punkt av fältet är den som nordändan på en kompassnål
skulle peka åt.
Om man hänger upp en magnet i en tråd, så kommer den att inta samma riktning
som jordens magnetfält.

\begin{itemize}
	\item Poler med samma polaritet stöter bort varandra (repellerar).
	\item Poler med olika polaritet dras till varandra (attraherar).
\end{itemize}

\tallfig{images/cropped_pdfs/bild_2_1-08.pdf}{Magnetiska fält omkring strömledare}{fig:BildII1-8}

\subsection{Magnetiska fält omkring strömbanor}
\harec{a}{1.4.1}{1.4.1}
\label{magfält_ström}

Bild \ref{fig:BildII1-8} visar magnetiska fält omkring strömledare.
Omkring varje ledare, som det flyter en elektrisk ström igenom, alstras det ett
magnetiskt kraftfält.
Magnetiska kraftlinjerna fördelar sig koncentriskt omkring en rak ledare och
vinkelrätt mot denna.
Mellan ändarna av en ledare med bågformad utsträckning bildas kraftlinjer som
verkar med varandra.
En strömgenomfluten cylindrisk spole -- induktor -- uppvisar samma magnetiska
fältbild som en stavformad permanentmagnet.

\subsection{Bestämma magnetiska fältriktningen}

Magnetfältets riktning omkring en ledare kan bestämmas med
\emph{högerhandsregeln}.
När en \emph{ledare} fattas med höger hand och med tummen i strömmens
riktning, kommer fingrarna att peka i fältriktningen (B).

I bild \ref{fig:BildII1-8} (övre) så går strömmen från pluspolen (+) till
minuspolen (--) varvid strömmen kommer gå nedåt i bilden på ovansidan,
det vill säga precis så tummen pekar om man greppar ledaren med tummen nedåt,
och magnetfältet kommer att snurra som pilarna precis som de övriga fingrarna
på högerhanden.

När en ledare formas som en spole och en elektrisk ström flyter genom den,
kommer magnetfältet att ha ett utseende som liknar det omkring en
permanentmagnet.
En sådan spole kallas \emph{elektromagnet}.

Magnetfältets riktning i en spole kan också bestämmas med högerhandsregeln.
När \emph{en spole} fattas med höger hand och med fingrarna i strömmens
riktning, kommer den utsträckta tummen att peka mot spolens nordpol.

I bild \ref{fig:BildII1-8} (undre) så går strömmen från pluspolen (+) till
minuspolen (--) varvid strömmen kommer gå inåt i bilden på ovansidan, dvs.
precis så fingrarna pekar när man lägger handen på spolen, och magnetfältet
kommer att peka mot nord (N) precis som tummen på högerhanden.

Fälten omkring alla slags magneter, såväl permanentmagnetiska som
elektromagnetiska, återverkar på varandra.
Även enkla elektriska ledare är elektromagneter.

\tallfig{images/cropped_pdfs/bild_2_1-09.pdf}{Exempel på elektromagneter}{fig:BildII1-9}

\subsection{Exempel på elektromagneter}

Bild \ref{fig:BildII1-9} visar exempel på elektromagneter.

\subsubsection{Elektromagnet}
Det bildas ett magnetfält genom en spole så länge som det flyter ström genom
den.
En järnkärna i spolen koncentrerar fältet på grund av den större magnetiska
ledningsförmågan.

Elektromagneter används för att sätta magnetiska material i rörelse eller hålla
fast dem.

\subsubsection{Elektrisk ringklocka}
Anordningen består av en elektromagnet och en järnplatta på en fjäder.
På plattan sitter en självbrytande kontakt samt en kläpp som kan slå på en
klocka.

Kontakten åstadkommer en växelvis brytning och slutning av strömmen genom
elektromagneten.
Armaturen med kläppen kommer då i svängning och slår på klockan.

\subsubsection{Telefon}
I en enkel telefon finns bland annat en mikrofon, ett batteri och en
hörtelefon.

Särskilt i äldre telefoner består mikrofonen av en kolkornskammare med ett
membran.
Tryckvariationer (ljud) får membranet att vibrera, varvid resistansen genom
kolkornen varierar i motsvarande grad.
Därmed varierar talströmmen genom mikrofonen.

Hörtelefonen består av en elektromagnet och ett membran av mjukjärn.
Variationer i talströmmen genom mikrofonen passerar även hörtelefonen och får dess
magnetfält att variera.
Hörtelefonens membran alstrar då trycksvariationer, det vill säga ljud.

\subsubsection{Elektromagnetiskt relä}
Reläet består av en elektromagnet, en järnplatta (ankare) på en fjäder och en
elektrisk kontakt.
Med en svag ström / låg spänning genom spolen i manöverkretsen kan man med
reläets arbetskontakt styra starkare ström / högre spänning i huvudkretsen.

\subsection{Magnetisk fältstyrka}
\label{magnetisk_fältstyrka}
\index{magnetisk fältstyrka}
\index{symbol!\(H\) magnetisk fältstyrka}

Som magnetisk fältstyrka förstår man flödet per meter fältlinje, det vill säga:

\begin{equation*}
  H = \frac{\Theta}{l} = \frac{I \cdot N}{l} \\
\end{equation*}
%
\begin{equation*}
  H = \frac{\Theta}{l} = \frac{I \cdot N}{l} \\
\end{equation*}

\begin{table}[H]
	\centering
	\begin{tabular}{rl}
	$H$ & [A/m]\\
	$I$ & [A]\\
	$N$ & varvtal\\
	$l$ & fältlinjelängd\\
\end{tabular}
\end{table}

\emph{Magnetisk fältstyrka uttrycks således som ampere per meter flödesväg.}

\subsection{Magnetisk flödestäthet}
\index{magnetisk flödestäthet}
\index{tesla (T)}
\index{enheter!tesla (T)}
\index{symbol!\(B\) magnetisk flödestäthet}
\index{permeabilitet}
\index{symbol!\(\mu\) permeabilitet}

Den magnetiska flödestätheten \(B\) mäts i enheten tesla \(\text{[T]}\) (förut gauss):
%%
\[B = \mu \cdot H \quad B\, \text{[Vs/m$ ^2 $]}\quad H\, \text{[A/m]}\]
%%
%%Flödestäthet \(B\ [Vs/m^2]\) Fältstyrka \(H\ [A/m]\)

\(\mu\) är permeabilitetstalet för materialet.
\(\mu_0\) är permeabilitetstalet (fältkonstanten) för den magnetiska
ledningsförmåga för vakuum.

För järn eller annat magnetiskt ledande material tillkommer permeabilitetstalet
\(\mu_r\).
Det anger hur många gånger bättre än luft etc., som materialet leder ett
magnetisk flöde.
Permeabilitetstalet kan då skrivas
\(\mu = \mu_r\mu_0\):
%%
\[B = \mu_0 \cdot \mu_r \cdot H\]
%%

\newpage %layout 

\subsection{Magnetiskt flöde}
\index{magnetiskt flöde}
\index{symbol!\(\Phi\) magnetiskt flöde}

Det magnetiska flödet är produkten av flödestätheten \(B\) och tvärsnittsytan
\(A\) av flödesvägen:
%%
\[\Phi = B \cdot A\]
\[\Phi\, \text{[weber eller Vs]}\quad B\, \text{[T eller tesla]} \quad A\, [\text{m}^2]\]

%%
\subsection{Skärmning av magnetiska fält}
\harec{a}{1.4.2}{1.4.2}
\index{magnetiska fält!skärmning}
\label{elektromagnetisk skärmning}

I grunden finns det två slags fält, det elektriska och det magnetiska. Det
finns även elektromagnetiska fält som är sammansatta av båda dessa.
Fält kan vara permanenta eller rörliga, varav här avses de rörliga.
Ett rörligt magnetiskt fält genererar ett elektriskt fält.
Omvänt generar ett rörligt elektriskt fält ett rörligt magnetiskt fält.
Denna växelverkan gör att fälten kan hållas igång med tillförsel av yttre
energi.

Fält i rörelse alstrar elektromagnetisk strålning, som påverkar funktioner i
omgivningen.
När påverkan inte är önskvärd, måste fältet skärmas av.
Ett sätt att skärma magnetiska fält är en metallisk kapsling.
Kapslingen ska vara tät och bilda en sluten magnetisk krets.
Kapslingen ska vara utförd i ett material som är en god ledare av magnetiskt
flöde.
(Jämför \ref{elektrostatik skärmning})
