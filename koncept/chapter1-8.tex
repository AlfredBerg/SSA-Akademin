\section{Modulation}
\harecsection{\harec{a}{1.8}{1.8}}
\label{modulation}
\index{modulation}

\subsection{Allmänt}
\index{modulation}
\index{modulerande signal}
\index{basband}
\index{modulerad signal}
\index{bärvåg}

\emph{Modulera} (lat. \emph{modulari}, rytmiskt avmäta, eng. \emph{modulate})
är att med hjälp av en oftast högfrekvent elektrisk signal (bärvågen) överföra
informationen i en lågfrekvent signal.
På så sätt kan lågfrekvens, till exempel tal och musik, först omvandlas till en
elektrisk signal, som får påverka (modulera) en högfrekvent elektrisk signal.
Denna modulerade signal strålas ut från antennen som ett elektromagnetiskt fält.

Den signal som innehåller informationen kallas \emph{modulerande signal},
\emph{basband} eller \emph{underbärvåg}.

Den signal som informationen överförts till kallas \emph{modulerad signal},
\emph{bärvåg} eller \emph{huvudbärvåg}.

\subsection{Modulationssystem}
\label{modulationssystem}

Den största gruppen av modulationssystem är definierad med avseende på hur
huvudbärvågen är modulerad.
Vanligast är då amplitud- och vinkelmodulation.
Av vinkelmodulation finns främst två slag, frekvensmodulation och fasmodulation.
Därutöver finns system för pulsmodulation.

\subsection{Sändningsslag}
\index{sändningsslag}
\label{sändningsslag}

Sätten att modulera kallas \emph{sändningsslag}.
Gemensamt för sändningsslagen är att en givare -- det kan vara en mikrofon, en
telegrafnyckel, en fjärrskriftsmaskin, en dator, en TV-kamera -- alstrar
en analog eller digital signal.
Denna styr underbärvågen så att huvudbärvågen moduleras med den avsedda
informationen och sänds ut.

Det enklaste sändningsslaget får anses vara morsetelegrafi med
''nycklad bärvåg''.
Då förekommer bara två tillstånd, nedtryckt och icke nedtryckt telegrafnyckel,
dvs. antingen bärvåg med någon varaktighet eller ingen bärvåg alls.
Kombinationer av bärvågselement med olika längd motsvarar skrivtecken.

För att återge tal, musik etc. behövs en noggrannare tillståndsstyrning av
bärvågen.
Det innebär att bärvågen måste moduleras av en underbärvåg och att denna
motsvarar lufttrycksvariationerna i ljudet.

\subsection{Kännetecken för modulerade signaler}
\label{kännetecken_modulerade_signaler}
\harecsection{\harec{a}{1.8.5}{1.8.5a}}
\index{amplitudmodulation}
\index{frekvensmodulation}
\index{fasmodulation}
\index{pulsmodulation}

\mediumfig{images/cropped_pdfs/bild_2_1-22.pdf}{Modulerade signaler}{fig:BildII1-22}

Bild \ref{fig:BildII1-22} illustrerar modulerade signaler.
En modulerad signal kännetecknas av dess amplitud, frekvens och fasläge.

Vid \emph{amplitudmodulation} påverkas huvudbärvågens amplitud, så att den i
varje tidpunkt motsvarar den modulerande signalens variation.

Vid \emph{frekvensmodulation} påverkas huvudbärvågens frekvens, så att den i
varje tidpunkt motsvarar den modulerande signalens variation.

Vid \emph{fasmodulation}, som är besläktad med frekvensmodulation, påverkas i
stället för frekvensen huvudbärvågens fasläge i förhållande till en
referenssignal, så att fasläget i varje tidpunkt motsvarar den modulerande
signalens variation.

Frekvens- och fasmodulation liknar varandra och kan sammanfattas som
vinkelmodulation, eftersom fasvinkeln mellan bärvågens spänning och ström
varierar i båda fallen.

Vid \emph{pulsmodulation} används pulståg (korta upprepade bärvågspaket), till exempel
pulsamplitud-, pulslängds-, pulsläges- och pulskodmodulation.
Pulskodmodulation används till exempel vid samtidig överföring av flera telesamtal på
samma linje, bärvåg etc.

% \newpage % layout

\subsection{Bandbredd vid olika sändningsslag}
\harecsection{\harec{a}{1.8.5}{1.8.5b}}
\index{bandbredd}
\index{frekvenseffektivitet}
\label{bandbredd_modulation}

Varje radiosändning tar upp plats omkring den nominella bärvågsfrekvensen --
tillsammans \emph{bandbredden}.

Radioamatören måste veta detta ''platsbehov'', främst för att inte sända utanför
de frekvensband som är tilldelade för amatörradioanvändning, men även för att
kunna umgås med annan trafik inom banden.

I alla sändningsslag ökar den använda bandbredden med ökad modulation.
Eftersom största \emph{frekvenseffektivitet} alltid ska eftersträvas så upptar
en sändare med kraftigare modulation än vad som behövs för en överföring alltid
onödigt frekvensutrymme.

\subsection{Beskrivningskod för sändningsslagen}
\index{sändningsslag}
\label{modulation_beskrivningskod}

Vid 1979 års radioförvaltningskonferens (WARC 79) i Geneve reviderades det
internationella radioreglementet (RR), som i huvudsak trädde i kraft 1982.
Däri ingår bland annat ett nytt system för klassindelning och beteckning av
sätten att utsända information över radio med mera.
Reglementet har reviderats senare, men i detta stycke gäller det ännu.

Indelningen i \emph{sändningsslag} behövs för att känneteckna utsändningarna,
till exempel i frekvenslistor, författningar och föreskrifter.
Indelningen är också av stort värde vid teknisk beskrivning av apparater och
system för radiokommunikation.

Emellertid används av många även äldre benämningar, vilka lever kvar i
litteraturen, i märkning av manöverdonen på sändare och mottagare.

Dessa äldre benämningar är dock inte entydiga och skapar lätt missförstånd,
varför beskrivningskoden enligt WARC~79 bör användas för tydlighetens skull.

Här följer avkortade koder enligt WARC~79 för några av de sändningsslag som
amatörer använder mest, samt för jämförelse även de benämningar som fortfarande
används jämsides (se vidare i bilaga \ref{sändslag}).

\mediumfig[0.67]{images/cropped_pdfs/bild_2_1-23.pdf}{Modulerande signaler}{fig:BildII1-23}

\begin{description}
\item[NON] Bärvåg utan modulerande signal. Ingen information.

\item[A1A] Bärvåg med dubbla sidband. En enda kanal med kvantiserad bärvåg.
Ingen modulerande underbärvåg. Telegrafi. Även kallat nycklad bärvåg (CW).

\item[A3E] Linjärt modulerad huvudbärvåg. Dubbla sidband. En enda kanal med
analog information. Telefoni. Även kallat amplitudmodulation (AM).

\item[J3E] Linjärt modulerad huvudbärvåg. Ett sidband med undertryckt bärvåg.
  En enda kanal med analog information. Telefoni.
  Även kallat enkelt sidband, Single Side Band (SSB).

\item[F3E] Vinkelmodulerad bärvåg. Frekvensmodulering. En enda kanal med analog
information. Telefoni. Även kallat frekvensmodulering (FM).

\item[G3E] Vinkelmodulerad bärvåg. Fasmodulering. En enda kanal med analog
information. Telefoni. Även kallat fasmodulering (PM).
\end{description}

Såväl A1A, A3E som J3E är sändningsslag där amplituden moduleras.
Därför är termen \emph{amplitudmodulation} inte tillräcklig för att beskriva
flera likartade sändningsslag.

\subsection{Modulerande signaler}
\harecsection{\harec{a}{1.7.1}{1.7.1}}
\index{modulerande signaler}

\subsubsection{Basband}
\index{basband}

Basband är ett frekvensområde för en modulerande signal.
Det finns ett basband för alla slags modulerande signaler, vare sig de är
analoga eller digitala.
Det kan finnas mer än ett basband i en komplett modulationsprocess.
Till exempel är en nycklad ton, som går till sändaren genom mikrofoningången,
dess analoga basband medan nycklingspulserna till tongeneratorn är dess
digitala basband.

Bild \ref{fig:BildII1-23} illustrerar modulerade signaler.
Ett vanligt sätt att överföra information över radio är med telefoni, det vill
säga tal.

Frekvensområdet \SIrange{300}{3000}{\hertz} räcker för god förståelighet av tal.
Dels är örat känsligast inom det området och dels finns där den mesta energin
i talet.

Mikrofonen tar upp de lufttrycksvariationer som uppstår när man talar och
omvandlar dem till elektriska svängningar.
Svängningarna varierar mellan positiva och negativa spänningsvärden.

\bigskip

\textbf{Försök}

\begin{enumerate}
\item Anslut en mikrofon till ett oscilloskop och studera spänningsförloppen
  för olika slags ljud, toner, tal osv. som funktion av tiden.
  På bilden är dessa svängningar mycket förenklade, till exempel sinusformade.

\item Anslut en högtalare och ett oscilloskop till en LF-generator, vars
frekvens och amplitud kan ändras. Lyssna på ljud med låg och hög frekvens samt
på svaga och starka ljud.
En baston har låg frekvens och en diskantton har hög frekvens.
En svag ton har liten amplitud och en stark ton har stor amplitud.
\end{enumerate}

\subsection{Sändningsslaget A3E (AM)}
\harecsection{\harec{a}{1.8.2}{1.8.2}, \harec{a}{1.8.6b}{1.8.6b}, \harec{a}{1.8.7b}{1.8.7b}}
\index{amplitudmodulation}
\index{A3E}
\index{AM|see {amplitudmodulation}}
\label{modulation_am}

\mediumfig{images/cropped_pdfs/bild_2_1-24.pdf}{Sidband vid A3E-modulation}{fig:BildII1-24}

Bild \ref{fig:BildII1-24} visar frekvensspektrum av en signal vid
amplitudmodulation med

\begin{enumerate}[label=\alph*.,noitemsep]
\item en sinuston,
\item en blandning av tre sinustoner,
\item ett frekvensspektrum.
\end{enumerate}

\noindent\textbf{Försök}
%
Modulera en A3E-sändare med en \SI{3}{\kilo\hertz}-signal.
Med en mottagare utrustad med ett smalt filter för telegrafi, kan man urskilja
och påvisa bärvågen och de båda sidbanden.

\subsubsection{A3E-modulation med en ton}

\mediumfig{images/cropped_pdfs/bild_2_1-25.pdf}{A3E-modulation med toner med olika styrka och frekvens}{fig:BildII1-25}

Bild \ref{fig:BildII1-25} visar A3E-modulation med toner av olika styrka och
frekvens.
En omodulerad bärvåg har konstant amplitud.
En amplitudmodulerad signal är i grunden resultatet av svävning mellan
frekvenser eller av icke linjär blandning av frekvenser.
När bärvåg och basband blandas är särskilt tre blandningsprodukter av
intresse.

Dessa är:
\begin{itemize}
\item bärvågen
\item det lägre sidbandet (förkortat LSB)
\item det övre sidbandet (förkortat USB).
\end{itemize}

AM-signalen består således inte bara av bärvågsfrekvensen \(f_{HF}\) utan även
av övre och nedre sidofrekvenser, vilka är summan och skillnaden av
bärvågsfrekvensen \(f_{HF}\) och den modulerande frekvensen \(f_{LF}\).
Alltså \(f_{HF} + f_{LF}\) (övre sidofrekvens) och skillnadsfrekvensen
\(f_{HF} - f_{LF}\) (undre sidofrekvens).

Eftersom tal inte bara omfattar en enda frekvens utan ett helt frekvensspektrum
(ca \SIrange{0,3}{3}{\kilo\hertz}) uppstår inte bara två sidofrekvenser utan två
sidband, det lägre sidbandet (LSB, Lower Side Band) och det övre (USB, Upper
Side Band).

LF-signalens frekvens bestämmer sidofrekvensens avstånd från bärvågen.
Bandbredden på en amplitudmodulerad signal med full bärvåg och två sidband är
dubbelt så stor som den högsta modulerande LF-frekvensen:
\(b= 2 \cdot f_{LFmax}\)

Om de modulerande LF-frekvenserna är mellan 0,3 och \SI{3}{\kilo\hertz} blir
sändningens totala bandbredd \SI{6}{\kilo\hertz}.

LF-signalernas amplitud påverkar sidbandens och sidofrekvensernas amplitud.
Vid maximal modulation (100~\% modulationsgrad) varierar signalamplituden mellan
noll och dubbla värdet av det för en omodulerad bärvåg.

Som mest kan vardera sidbandet överföra en fjärdedel så mycket effekt som
bärvågen, dvs. en sjättedel av den totalt utsända effekten.
Då avger sändaren dubbelt så stor medeleffekt som utan modulation.
Toppeffekten (PEP, Peak Envelope Power) är till och med fyra gånger så stor.

Slutförstärkaren och kraftförsörjningen måste dimensioneras för toppeffekten vid
full modulation eller att modulationsgraden anpassas så att överbelastning inte
sker.

\subsubsection{Fördelar med A3E-modulation}

En A3E-sändare är enkel jämfört med en J3E-sändare, vilken har en mer
komplicerad signalbehandling.

\pagefig{images/cropped_pdfs/bild_2_1-26.pdf}{Amplitudmodulation med morsetecken}{fig:BildII1-26}

\subsubsection{Nackdelar med A3E-modulation}

Eftersom samma information finns i båda sidbanden och ingen finns i bärvågen,
så sänds effekten i bärvågen och ett av sidbanden ut till ingen nytta.
I talpauser sänds endast bärvågseffekten och till ingen nytta.
Även frekvensutrymme slösas bort.
Då en annan, alltför närliggande sändares bärvåg blandas med den egna,
alstras interferenstoner i mottagarna.

\mediumplustopfig{images/cropped_pdfs/bild_2_1-27.pdf}{Sidband vid DSB}{fig:BildII1-27}

\subsection{Sändningsslaget A1A (CW)}
\harecsection{\harec{a}{1.8.1}{1.8.1}, \harec{a}{1.8.6a}{1.8.6a}, \harec{a}{1.8.7a}{1.8.7a}}
\index{A1A}
\index{CW}
\label{modulation_cw}


Bild \ref{fig:BildII1-26} visar amplitudmodulation med morsetecken.
Man kan överföra meddelanden med morsetelegrafi på olika sätt.
Det enklaste sättet är att koppla in och ur sändarens bärvåg i takt med
teckendelarna i morsetecknen.
Man kan kalla det för bärvågstelegrafi.
Förfarandet kallas sedan mycket länge även för CW (continous waves), vilket
egentligen anger att bärvågen svänger med konstant amplitud, om man bortser
från att den nycklas.
Detta står i motsats till de dämpade bärvågssvängningar som var fallet i sedan
mycket länge förbjudna gnistsändare.

Fastän en sändare ''moduleras utan ton'', har den en viss bandbredd.
Det beror på att den takt, som sändaren nycklas med, egentligen är en ton --
låt vara med låg frekvens.
Antag att sändaren nycklas med en serie korta morsetecken.
Vid telegraferingshastigheten 60~tecken/minut alstrar bärvågspulserna en kantvåg
med frekvensen \SI{5}{\hertz}.
Som tidigare beskrivits, består en sådan kantvåg av summan av sinussignaler med
frekvenserna \SI{5}{\hertz}, \SI{15}{\hertz}, \SI{25}{\hertz}, \SI{35}{\hertz}
och så vidare.

Det innebär att det uppstår sidofrekvenser över och under bärvågens frekvens och
med ett avstånd till bärvågen av \SI{5}{\hertz}, \SI{15}{\hertz},
\SI{25}{\hertz}, \SI{35}{\hertz} osv.
Telegrafisändaren har alltså liksom vid A3E en bandbredd, som dels står i
förhållande till nycklingshastigheten och dels till ''kantigheten'' på tecknen,
vilket bestämmer övertonshalten i bärvågen.
Vid så kallad mjuk nyckling kan den 9:e övertonen antas vara den högsta som
uppfattas av en motstation.
Med en nycklingsfrekvens av \SI{5}{\hertz} blir bandbredden inte större än
\(2 \cdot 10 \cdot 5 = \SI{100}{\hertz}\).

En hård (kantig) och snabb teckengivning ökar bandbredden och kan resultera i
att så kallade nycklingsknäppar kan uppfattas långt vid sidan om
sändningsfrekvensen.
Ju hårdare nycklingen är, desto längre bort från bärvågsfrekvensen hörs
nycklingsknäpparna.
Detta stör andra stationer.

Kännetecken för sändningsslaget A1A, telegrafi genom nycklad bärvåg:

Mycket liten bandbredd, extremt gott utnyttjande av sändareffekten, stor
överföringssäkerhet, lång räckvidd, enkla sändare.

\subsection{Sändningsslaget J3E (SSB)}
\harecsection{\harec{a}{1.8.3c}{1.8.3c}, \harec{a}{1.8.6c}{1.8.6c}, \harec{a}{1.8.7c}{1.8.7c}}
\index{Single Side Band (SSB)}
\index{J3E}
\index{SSB}
\label{modulation_ssb}

\subsubsection{Princip}

Som sagts är det onödigt att sända ut två sidband, eftersom båda innehåller samma
information.

Signaler med endast ett sidband och undertryckt bärvåg kan alstras på flera
sätt.
Numera är den så kallade filtermetoden i särklass vanligast och den enda som
behandlas här.

Bild \ref{fig:BildII1-27} illustrerar sidband vid DSB-modulation.
Med filtermetoden blandas HF- och LF-signalerna i en speciell blandare.
Där undertrycks båda dessa signaler medan blandningsprodukterna med deras summa-
och skillnadsfrekvenser blir kvar, dvs. det övre och nedre sidbandet.

Utsignalen från blandaren benämns DSB-signal (Double Side Band).
Till skillnad från i A3E-signalen saknas dock bärvågen i DSB-signalen.
För att även undertrycka det ena sidbandet före sändningen följs blandaren
av ett bandpassfilter med bandbredd och frekvensläge för avsett sidband.

Den signal som sänds ut innehåller därför endast ett sidband (Single Side Band).

\newpage
% \tallfig[0.45]{images/cropped_pdfs/bild_2_1-28.pdf}{Sidbandsval vid SSB}{fig:BildII1-28}
\mediumtopfig{images/cropped_pdfs/bild_2_1-28.pdf}{Sidbandsval vid SSB}{fig:BildII1-28}

\paragraph{Exempel}


Bild \ref{fig:BildII1-28} illustrerar sidbandsval vid SSB-modulering.
Ett SSB-filter har ett passband av \SIrange{9000,3}{9003}{\kilo\hertz}.
Vid bärvågsfrekvensen \SI{9000}{\kilo\hertz} sträcker sig det övre sidbandet
från \SIrange{9000,3}{9003}{\kilo\hertz} och släpps igenom.
Däremot blir bärvågsfrekvensen undertryckt.

Det undre sidbandet \SIrange{8997}{8999,7}{\kilo\hertz} faller utanför filtrets
passband och blir också undertryckt.

Ska däremot det undre sidbandet kunna passera igenom samma filter, så måste
bärvågsfrekvensen höjas med \SI{3}{\kilo\hertz}, alltså till
\SI{9003}{\kilo\hertz}.
Då faller det undre sidbandet, \SIrange{9002,7}{9000,0}{\kilo\hertz} inom
filtrets passband.

Det övre sidbandet \SIrange{9003,3}{9006,0}{\kilo\hertz} faller nu utanför
passbandet och blir undertryckt.

%% k7per: Make this bigger.
\mediumtopfig{images/cropped_pdfs/bild_2_1-29.pdf}{Sidbandslägen vid SSB}{fig:BildII1-29}

Bild \ref{fig:BildII1-29} illustrerar sidbandslägen vid SSB.
LF-signalens amplitud bestämmer amplituden på sidofrekvensen.

LF-signalens frekvens bestämmer sidofrekvensens avstånd från bärvågsfrekvensen
(bärvågen undertryckt).

Bandbredden på den utsända signalen är skillnaden mellan högsta och lägsta
modulerande frekvens i signalen:

till exempel \(b = \SI{3}{\kilo\hertz} - \SI{0,3}{\kilo\hertz} =
\SI{2,7}{\kilo\hertz}\)

\subsubsection{Fördelar med J3E-modulation}
Bra verkningsgrad vid J3E-modulation jämfört med vid A3E-modulation
(traditionell AM).
Effekten i det utsända sidbandet motsvarar den i ett av sidbanden vid A3E.
Hela den utsända effekten finns alltså i ett enda sidband,
som överför hela informationen.

I sändningspauserna sänds ingen effekt ut.
Bandbredden är mindre än hälften av den vid A3E.
Vid mottagning av en J3E-sändning (SSB) är det mindre besvär med
interferenstoner från J3E-sändningar på närliggande frekvenser, eftersom ingen
bärvåg och endast ett sidband sänds ut.

\subsubsection{Nackdelar med J3E-modulation}
J3E-modulation medför mera komplicerade apparater, både för mottagning och
sändning.
En J3E-signal blir förvrängd och hörs i fel tonläge om mottagaren
inte är inställd på exakt rätt frekvens.

\subsection{Vinkelmodulation}
\harecsection{\harec{a}{1.8.3a}{1.8.3a}}
\index{vinkelmodulation}
\label{modulation_vinkel}

Termen vinkelmodulation är samlingsnamnet för frekvensmodulation (FM) och
fasmodulation (PM).
Ofta sägs utrustningar vara för frekvensmodulation när de antingen är för
frekvens- eller fasmodulation.
Det finns alltså skillnader och likheter mellan dessa system, vilka emellertid
inte är oberoende av varandra, eftersom frekvensen i en signal inte kan
varieras utan att fasen också varieras, och vice versa.

Hur effektiv kommunikationen då är beror mest på mottagningsmetoderna.
I båda fallen uppfattas ändringar i den mottagna signalens frekvens och fasläge.
Amplitudändringar uppfattas däremot inte.
De flesta störningar -- särskilt pulserande sådana som från tändningssystem --
kommer därför att skiljas bort.

För att effektivt utnyttja fördelarna med vinkelmodulation, antingen det är
frekvens eller fasmodulation, behövs tillräckligt frekvensutrymme.
Det innebär att främst högre frekvensband kommer i fråga.

\newpage
\subsection{Frekvensmodulation (FM)}
\harecsection{\harec{a}{1.8.3b}{1.8.3b}, \harec{a}{1.8.6d}{1.8.6d}}
\index{frekvensmodulation}
\index{FM|see {frekvensmodulation}}
\label{modulation_fm}

\mediumfig[0.8]{images/cropped_pdfs/bild_2_1-30.pdf}{Frekvensmodulation}{fig:BildII1-30}

Bild \ref{fig:BildII1-30} (överst och i mitten) visar frekvensmodulation.

Vid frekvensmodulation varierar bärvågens frekvens i takt med den modulerande
signalens amplitud och polaritet.
På bilden ökar bärvågens frekvens när den modulerande signalen är positiv
(första halvperioden) och minskar när den modulerande signalen är negativ
(andra halvperioden).
Bilden visar att perioderna i den modulerade bärvågen tar kortare tid (har
högre frekvens), när den modulerande signalen är positiv, och mer tid (har lägre
frekvens) när den modulerande signalen är negativ.
Bärvågen kommer alltså att pendla omkring ett medelvärde, dvs. vara
frekvensmodulerad.

Frekvensavvikelsen \(\Delta f\) (deviationen) från bärvågens vilofrekvens är
vid varje tillfälle proportionell mot den modulerande signalens amplitud.
Sålunda är deviationen liten när den modulerande signalens amplitud är liten
och störst när amplituden når sitt toppvärde, antingen amplituden är positiv
eller negativ.
Vid en modulationsfrekvens av \SI{300}{\hertz} varierar bärvågsfrekvensen 300
gånger per sekund, vid \SI{3}{\kilo\hertz} varierar den 3000 gånger per sekund.

Likspänningsnivåer kan överföras med FM, eftersom en motsvarande
frekvensavvikelse kan framställas.

Bilden visar också vad som oftast sägs, att bärvågsamplituden inte ändras av
modulationen.
Detta är emellertid bara delvis sant, eftersom såväl bärvågsamplitud som
sidbandsamplitud varierar med modulationsindex, vilket förklaras nedan.

\subsubsection{Sidbanden vid vinkelmodulation}

Vid AM produceras endast ett sidbandspar med samma innehåll, ett över och ett
under bärvågsfrekvensen.
Vid vinkelmodulation, både vid FM och PM, produceras däremot flera sidbandspar
över och under bärvågsfrekvensen.
Dessa sidband uppträder på multiplerna av varje modulerande frekvens.
Vid basband med samma frekvensomfång har därför en vinkelmodulerad signal
större bandbredd än en AM-signal.

Vid vinkelmodulation beror antalet sidband på sambandet mellan den modulerande
frekvensen, frekvensdeviationen och modulationsindex.

\mediumtopfig{images/cropped_pdfs/bild_2_1-31.pdf}{Sidbandsspektrum vid FM-modulering med 1 sinuston}{fig:BildII1-31}

\subsubsection{Bandbredden vid vinkelmodulation}

Bild \ref{fig:BildII1-30} (nederst) visar bandbredd på vinkelmodulation.
Vi gör tankeexperimentet att en FM-sändare moduleras med en fyrkantsvåg.
Frekvensen kommer då att hoppa växelvis mellan frekvenserna \(f\) och
\(f + \Delta f\).
Sättet kallas FSK (frekvensskiftnyckling) och används till exempel vid sändning av
radiofjärrskrift (RTTY, AMTOR, Paketradio etc.).

Vi föreställer oss två sändare, som sänder varannan gång, varav den ena sänder
frekvensen \(f\) och den andra sänder \(f + \Delta f\).
Båda sändarnas HF-signaler kommer då att bilda ett frekvensspektrum, som
förutom \(f\) och \(f + \Delta f\) även innehåller sidofrekvenser.

Bredden på detta spektrum beror bland annat på nycklingsfrekvensen.
Eftersom en fyrkantsvåg innehåller summan av dess grundfrekvens och övertoner,
kommer alla dessa toner att modulera vardera sändaren.
De högsta modulerande LF-frekvenserna alstrar sidofrekvenserna längst ut från
vilofrekvensen.
LF-signalens frekvensspektrum påverkar alltså HF-signalens bandbredd.

Spektrum nederst i bilden är en förenklad framställning av
frekvensskiftnyckling.

Vid modulation med en sinussignal istället för med en fyrkantssignal, uppstår
ett frekvensspektrum som på överst i bilden.

%% k7per: Ska inte detts också vara en subsubsection?
\subsubsection{Frekvensdeviation och modulationsindex}
\harecsection{\harec{a}{1.8.4}{1.8.4}}
\index{frekvensdeviation}
\index{modulationsindex}
\index{symbol!\(m\) modulationsindex}

%% k7per: Find a solution for words that already have a hyphen. quote-dash?
Bild \ref{fig:BildII1-31} visar sidbandsspektrum vid FM-moduler\-ing med 1
sinuston.
Vid vinkelmodulation uppstår talrika sidofrekvenser, som beror av den
modulerande frekvensen \(f_{LF}\).
Amplitudfördelningen mellan sidofrekvenserna står i förhållande till
deviationen, varvid deras amplitud blir mindre ju längre bort från bärvågen
de är.

I praktiken anses en sidofrekvens försumbar när dess amplitud är mindre än 1~\%
av amplituden för omodulerad bärvåg.

För beräkning av bandbredden används begreppet modulationsindex \(m\), vilket är
kvoten av maximal deviation \(\Delta f\) och högsta frekvensen \(f_{LF}\).
%%
\[m = \dfrac{\Delta f_{max}}{f_{LFmax}}\]
%%
Inom amatörradion är det vanligt att arbeta med \(\Delta f_{max} =
\SI{3}{\kilo\hertz}\) och \(f_{LFmax} = \SI{3}{\kilo\hertz}\), dvs. \(m = 1\).

Vid modulationsindex \(m = 1\), gäller följande formel för bandbredden \(b\)

% k7per: Make this a formula?
\medskip
\(b = 2 \cdot ( \Delta f_{max} + f_{LFmax}) = 2 \cdot \Delta f_{max}
 + 2 \cdot f_{LFmax}\)
 \medskip
 
Med ovan nämnda värden blir bandbredden \(b = 2 \cdot (\SI{3}{\kilo\hertz} +
\SI{3}{\kilo\hertz}) = \SI{12}{\kilo\hertz}\)

Bandbredden ökar således både med ökande deviation och ökande modulerande
frekvens.
För att inte interferera med trafik på grannkanalerna måste såväl deviation som
frekvensen på den modulerande signalen begränsas.
En deviationsbegränsare begränsar amplituden på denna signal.
Ett lågpassfilter reducerar den distorsion, som uppstår av begränsningen.
Vidare undertrycks modulerande frekvenser högre än \SI{3}{\kilo\hertz}, vilket
är tillräckligt för överföring av tal.

\paragraph{Jämförelse}

En VHF-rundradiosändare är tilldelad ett större frekvensutrymme och kan därför
använda mycket större bandbredd.

Där är \(\Delta f_{max} = \SI{75}{\kilo\hertz}\) och \(f_{LFmax} =
\SI{15}{\kilo\hertz}\), därmed är \(m = \frac{75}{15} = 5\) och \(b = 2 \cdot
(75 + 15) = \SI{180}{\kilo\hertz}\).

Som framgår av tabell~\ref{tab:ampmod} varierar bärvågens liksom
sidofrekvensernas inbördes amplitud med modulationsindex.
Detta ska jämföras med AM där bärvågens amplitud är konstant och endast
sidbandens amplitud varierar.

Vid vinkelmodulation utsläcks bärvågen \(A_0\) vid modulationsindex 2,404.
Den blir sedan ''negativ'' vid högre index, vilket betyder att den återkommer,
men att dess fasläge blir omvänt.
I vinkelmodulation tas energin i sidbanden från bärvågen, vilket innebär att
den totala effekten förblir densamma oavsett modulationsindex.

%\paragraph{Kännetecken för sändningsslaget F3E (FM)}
%\index{F3E}

\paragraph{Fördelar med sändningsslaget F3E (FM)}
F3E-sän\-daren är enkel till sin uppbyggnad och hög överföringskvalitet
uppnås vid stor bandbredd, störningar från amplitudmodulerade signaler såsom
tändgnistor undertrycks i mottagaren.

\paragraph{Nackdelar med sändningsslaget F3E (FM)}
En relativt stor bandbredd behövs för överföring av ett basband med stort
frekvensomfång.
Sändaren måste avge full effekt, även när modulation inte sker.

\begin{table*}[ht]
\begin{center}
  %\begin{tabular}{ll|S|S[table-format=-1.3]|S[table-format=-1.3|S[table-format=-1.3]|S[table-format=-1.3]|S[table-format=-1.3]|S[table-format=-1.3]|S[table-format=-1.3]|}
  \begin{tabular}{ll|S[table-format=-1.3]|S[table-format=-1.3]|S[table-format=-1.3]|S[table-format=-1.3]|S[table-format=-1.3]|S[table-format=-1.3]|S[table-format=-1.3]|l|}
\cline{3-9}
&\multicolumn{1}{l}{}  & \multicolumn{7}{|c|}{Modulationsindex} \\ \cline{3-9}
&\multicolumn{1}{l|}{}  &  \multicolumn{1}{c|}{1}   &   \multicolumn{1}{c|}{2}   &    \multicolumn{1}{c|}{3}   &    \multicolumn{1}{c|}{4}   &    \multicolumn{1}{c|}{5}   &    \multicolumn{1}{c|}{6}   &    \multicolumn{1}{c|}{7}   \\ \hline
\multicolumn{1}{|c|}{\multirow{11}{*}{\rotatebox[origin=c]{90}{Relativ amplitud på}}}&\(A_0\) & 0,765 & 0,224 & \num{-0,260} & \num{-0,397} & \num{-0,178} &  0,151 &  0,300 \\
\multicolumn{1}{|c|}{}&\(A_1\) & 0,440 & 0,577 &  0,334 & \num{-0,066} & \num{-0,328} & \num{-0,277} & -0,005 \\
\multicolumn{1}{|c|}{}&\(A_2\) & 0,115 & 0,353 &  0,486 &  0,364 &  0,047 & \num{-0,243} & -0,301 \\
\multicolumn{1}{|c|}{}&\(A_3\) & 0,020 & 0,129 &  0,309 &  0,430 &  0,365 &  0,115 & -0,168 \\
\multicolumn{1}{|c|}{}&\(A_4\) &       & 0,034 &  0,132 &  0,281 &  0,391 &  0,358 &  0,158 \\
\multicolumn{1}{|c|}{}&\(A_5\) &       & 0,016 &  0,043 &  0,132 &  0,261 &  0,362 &  0,348 \\
\multicolumn{1}{|c|}{}&\(A_6\) & \multicolumn{2}{c|}{} &  0,011 &  0,049 &  0,131 &  0,246 &  0,339 \\
\multicolumn{1}{|c|}{}&\(A_7\) & \multicolumn{3}{c|}{} &  0,015 &  0,053 &  0,130 &  0,234 \\
\multicolumn{1}{|c|}{}&\(A_8\) & \multicolumn{4}{c|}{}           &  0,018 &  0,057 &  0,128 \\
\multicolumn{1}{|c|}{}&\(A_9\) & \multicolumn{4}{c}{} &        &  0,021 &  0,059 \\
\multicolumn{1}{|c|}{}&\(A_{10}\) & \multicolumn{5}{c}{Tomma fält för \(A_n\) under 0,01 (1 \%)} &  &  0,024 \\ \hline
\end{tabular}
\end{center}
\caption{Relativa amplituden på bärvåg $A_0$ och sidofrekvenser $A_1$--$A_{10}$ vid
modulationsindex 1--7. (Vid omodulerad bärvåg är modulationsindex 0. Då är
bärvågens relativa amplitud 1,0.)}
\label{tab:ampmod}
\end{table*}


\subsection{Fasmodulation (PM)}
\index{fasmodulation}
\index{PM}

Vid fasmodulation varierar bärvågens fasläge i förhållande till ett
referensvärde.
Vid PM är frekvensändringen -- deviationen -- direkt proportionell mot hur
snabbt fasläget på den modulerande frekvensen ändras och till den totala
fasändringen.
Hastigheten på fasändringen är direkt proportionell mot frekvensen på den
modulerande frekvensen och till den momentana amplituden på den modulerande
signalen.

Det betyder att deviationen i PM-system ökar både med den momentana amplituden
och frekvensen på den modulerande signalen.
Detta att jämföras med FM-system där deviationen är proportionell mot den
momentana amplituden på den modulerande signalen.

I PM-system uppfattar demodulatorn i mottagaren endast momentana ändringar i
bärvågsfrekvensen.
Till skillnad från vid FM, så kan därför ändringar i likspänningsnivåer
överföras endast om en fasreferens används.

Med konstant amplitud på insignalen till modulatorn är vid PM
modulationsindex konstant oavsett modulerande frekvens, medan vid FM
modulationsindex varierar med den modulerande frekvensen.

\subsection{Frekvens- och fasmodulation jämförs}

\begin{itemize}
\item Frekvensmodulation (FM) alstras genom att sändarens oscillatorfrekvens
  varieras (devieras) i takt med den modulerande signalen (t.ex. tal).
  Det gör man genom att variera resonansfrekvensen i den resonanskrets som
  styr oscillatorfrekvensen.

\item Fasmodulation (PM) alstras vanligen genom att efter sändaroscillatorn
  variera den modulerande signalens fasläge i förhållande till en omodulerad
  bärvåg -- så kallad fasmodulering.
  Det gör man genom att variera resonansfrekvensen i en resonanskrets efter
  oscillatorn, dvs. utan att påverka oscillatorfrekvensen.

\item I båda fallen ändrar man alltså resonansfrekvensen i en resonanskrets i
  takt med frekvensen i den modulerande spänningen, men denna krets har
  olika placering i FM-sändare respektive PM-sändare.

\item I sändaren alstras det i båda fallen utfrekvenser som devierar från
  oscillatorns vilofrekvens.
  Graden av deviation skiljer emellertid vid FM och PM.
  Vid FM är deviationen proportionell mot amplituden på den modulerande
  underbärvågen medan deviationen vid PM är proportionell mot produkten av den
  modulerande underbärvågens amplitud och frekvens.

\item Den hörbara skillnaden mellan FM och PM är därför en annorlunda
  frekvensgång.
  Vid samtidig användning av PM-sändare och FM-mottagare är det alltså lämpligt
  att justera frekvensgången i PM-sändarens modulator, lämpligen
  med \SI{6}{\decibel} dämpning per oktav ökad frekvens.
\end{itemize}

\subsection{Pulsmodulation}
\index{pulsmodulation}
\index{PWM}
\index{PAM}
\index{PPM}

Pulsmodulation används mest i mikrovågsområdet.
Pulsmodulerade signaler sänds vanligen som en serie korta pulser åtskilda av
relativt långa pauser utan modulering.

En typisk sändning kan bestå av pulser med en längd av \SI{1}{\micro\second} och
en frekvens av \SI{1000}{\hertz}.
Toppeffekten på en pulssändning är därför mycket högre än dess medeleffekt

Före WARC~79 var symbolen för all pulssändning P.
Därefter används P endast för omodulerade pulståg.
Annan pulsmodulation har följande symboler

\begin{description}
\item[K] -- puls-/amplitudmodulation (PAM)
\item[L] -- pulsviddmodulation (PWM)
\item[M] -- pulsposition/fasmodulation (PPM)
\item[Q] -- vinkelmodulation under pulsen
\item[V] -- kombination av dessa eller annat sätt.
\end{description}

\begin{table*}[ht]
\begin{center}
\begin{tabular}{|L{.12\textwidth}|L{.18\textwidth}|L{.18\textwidth}|L{.18\textwidth}|L{.19\textwidth}|}
\hline
  Sändningsslag &
    Amplituden på LF-signalen &
    Tonhöjden på LF- signalen påverkar & 
    Bandbredden b förhåller sig till &
    För stor amplitud på LF-signalen medför \\
  \hline % =======================================================
  A3E (AM) & 
    amplituden i båda sidbanden &
    sidofrekvensernas avstånd från bärvågen &
    LF-signalens högsta frekvens & 
    övermodulering och för stor bandbredd \\
\hline
  J3E (SSB) & 
    amplituden på utsänt sidband & 
    sidofrekvensernas avstånd från bärvågen  & 
    skillnaden mellan LF-signalens högsta och lägsta frekvens & 
    för stor bandbredd, överstyrning av förstärkarsteg \\
\hline
  F3E (FM) &
    deviationen & 
    hastigheten på bärvågens frekvensändring &
    dubbla summan av största deviation och högsta LF-frekvens & 
    för stor deviation, för stor bandbredd \\
  \hline % =======================================================
\end{tabular}
\end{center}
\caption{Jämförelse mellan några vanliga sändningsslag inom amatörradio}
\end{table*}

% \newpage % layout

\subsection{Digital modulation}
\harecsection{\harec{a}{1.8.8}{1.8.8}}
\index{digital modulation}
\label{modulation_digital}

Utöver de klassiska analoga modulationsmetoderna finns ett antal digitala
modulationsformer.
De är anpassade för transmission av binära data.
I viss mån kan CW ses som digital modulation där 0 moduleras utan bärvåg
och 1 moduleras med bärvåg.
Det finns dock flera andra modulationsmetoder som FSK, 2-PSK/BPSK, 4-PSK och
QAM vilka presenteras i följande delavsnitt.

\subsubsection{Frekvensskiftsmodulation -- FSK}
\harecsection{\harec{a}{1.8.8a}{1.8.8a}}
\index{frekvensskiftsmodulation}
\index{Frequency Shift Keying (FSK)}
\index{FSK}
\index{frekvensmodulation}
\index{GFSK}
\index{Gaussian Frequency Shift Keying (GFSK)}
\index{Gaussiskt filter}
\index{C4FM}
\index{JT65}
\index{JT9}

\emph{Frekvensskiftsmodulation} (eng. \emph{Frequency Shift Keying (FSK)})
skiljer sig från CW-modulationen genom att den ändrar frekvensen, dvs. är en
variant av frekvensmodulation. I den enklaste formen, binär FSK växlar man
mellan två frekvenser, där en frekvens får representera 0 och den andra får
representera 1. Denna metod har används för modem på telefonförbindelser,
såsom Bell 103.

Eftersom varje växling mellan frekvenser ger avbrott i bägge signalerna,
likt nycklingen i CW, så kommer de att skapa sidband. Av det skälet filtrerar man
gärna signalen, och använder man ett Gaussiskt filter får man
\emph{Gaussian Frequency Shift Keying (GFSK)} som används av till exempel GSM-telefoni.

Man kan använda fler än två frekvenser, till exempel används fyra frekvenser i Continuous
4 level FM (C4FM), i Phase 1 radios, i Project 25 samt Fusion.

Frekvensskift används även för att sända långsamma meddelanden där JT65
använder 65 frekvenser som den skiftar mellan, medan JT9 använder 9~frekvenser.

\subsubsection{Binär fasskiftsmodulation -- 2-PSK \& BPSK}
\harecsection{\harec{a}{1.8.8b}{1.8.8b}}
\index{binär fasskift modulation}
\index{fasskift modulation!binär}
\index{2-PSK}
\index{fasskift modulation!2-PSK}
\index{BPSK}
\index{fasskift modulation!BPSK}
\index{Costas loop}

Istället för att modulera frekvensen kan man modulera polariteten eller fasen.
En sådan modulationsform är \emph{binär fasskift modulation} (eng.
\emph{Binary Phase Shift Keying (BPSK)} eller \emph{2-state Phase Shift Keying
(2-PSK)}.
Förenklat kan man säga att bärvågen moduleras med \num{+1} eller \num{-1}, ofta
med \num{+1} representerande \num{0} och \num{-1} representerande \num{1}.

En nackdel med BPSK är att om polariteten blir förväxlad kommer meddelandet
att bli inverterat, dvs. 0 blir 1 och 1 blir 0. BPSK behöver därför också
kompletteras med annan digital modulation för att hantera polariteten, något
som i allmänhet kan åstadkommas enkelt.

BPSK används av satellitnavigationssystem som GPS, GLONASS och Galileo.
För att återvinna BPSK behöver man ofta en speciell variant av PLL-loop känd
som \emph{Costas loop}, eftersom en normal PLL-loop inte klarar av
teckenändringarna på signalen.

\subsubsection{Fyrnivå fasskiftmodulation -- 4-PSK}
\harecsection{\harec{a}{1.8.8c}{1.8.8c}}
\index{4-PSK}
\index{fasskift modulation!4-PSK}
\index{kvadratur-modulering}
\index{quadrature modulation}
\index{In phase (I)}
\index{Quadrature (Q)}
\index{I/Q modulation}

Fasskiftmodulation kan även göras med flera nivåer. När fyra olika faslägen
används kallas det för \emph{fyrnivå fasskiftmodulation} (eng.
\emph{4-state Phase Shift Keying (4-PSK)}).

Istället för 180~graders fasskift (0 och 180 grader) som används vid 2-PSK/BPSK
så använder man 360/4 det vill säga 90 graders fasskift mellan symbolerna.
Ett effektivt sätt att avkoda det är att göra \emph{kvadraturmodulering}
(eng. \emph{quadrature modulation}) där man modulerar en signal till två
komponenter, i \emph{fas} (eng. \emph{In Phase (I)}) och förskjuten 90 grader \emph{kvadratur} (eng.
\emph{Quadrature (Q)}), ofta kallat I/Q modulering.

De fyra faslägena kan nu enkelt förklaras som amplituder i de olika faslägena
som anges av tabell \ref{tab:4-PSK}.

\begin{table}[t]
\begin{center}
\begin{tabular}{|r|r|r|r|}
\hline
Symbol & Vinkel & I & Q \\ \hline
0 &   0 & +1       &  0 \\
1 &  90 &  0       & +1 \\
2 & 180 & \num{-1} &  0 \\
3 & 270 &  0       & \num{-1} \\ \hline
\end{tabular}
\end{center}
\caption{4-PSK i kvadratur-modulering}
\label{tab:4-PSK}
\end{table}

Amplituden är densamma för alla fyra symbolerna, men med olika vinkel.
I likhet med 2-PSK/BPSK behöver man återvinna fasen och sedan kunna avgöra
vad som är 0 grader, men givet att det görs i den övriga modulationen så
kan informationen avkodas korrekt.

\subsubsection{Kvadratur-amplitudmodulation -- QAM}
\harecsection{\harec{a}{1.8.8d}{1.8.8d}}
\label{QAM}
\index{kvadratur-amplitudmodulation}
\index{QAM}
\index{16QAM}
\index{DAB}
\index{DVB-T}
\index{DVB-T2}
\index{Wi-Fi}

Medan fasskiftning kan göras för fler fassteg har man funnit att det inte
är lika enkelt för högre upplösningar. Redan vid åtta steg behöver man ha
I- och Q-värden som är \(\sqrt{1/2}\), vilket i och för sig går att approximera.
En smidigare modulationsform är istället att låta även amplituden variera,
och genom att låta några bitar modulera I och några bitar modulera Q kan
man enkelt få ett symbolmönster som är effektivt att implementera.
Denna modulationsform kallar man \emph{kvadratur-amplitudmodulation}
(eng. \emph{Quadrature Amplitude Modulation (QAM)}).

Ofta benämner man olika varianter med antalet olika positioner, så att 16QAM
har 16 olika lägen i fas och amplitud tillsammans.
Ett exempel på hur 16QAM kan moduleras finns i tabell \ref{tab:16QAM}.

\begin{table*}[ht]
\begin{center}
\begin{tabular}{|r|r|r|r|r|r|r|}
\hline
Symbol & Isym & Qsym & Amplitud      & Vinkel &  I &   Q \\ \hline
     0 &    0 &    0 & \(3\sqrt{2}\) &    +45 & +3 &  +3 \\
     1 &    0 &    1 & \(\sqrt{10}\) &    +72 & +3 &  +1 \\
     2 &    0 &    2 & \(\sqrt{10}\) &   +108 & +3 &  \num{-1} \\
     3 &    0 &    3 & \(3\sqrt{2}\) &   +135 & +3 &  \num{-3} \\
     4 &    1 &    0 & \(\sqrt{10}\) &    +18 & +1 &  +3 \\
     5 &    1 &    1 &  \(\sqrt{2}\) &    +45 & +1 &  +1 \\
     6 &    1 &    2 &  \(\sqrt{2}\) &   +135 & +1 &  \num{-1} \\
     7 &    1 &    3 & \(\sqrt{10}\) &   +162 & +1 &  \num{-3} \\
     8 &    2 &    0 & \(\sqrt{10}\) &   +342 & \num{-1} &  +3 \\
     9 &    2 &    1 &  \(\sqrt{2}\) &   +315 & \num{-1} &  +1 \\
    10 &    2 &    2 &  \(\sqrt{2}\) &   +225 & \num{-1} &  \num{-1} \\
    11 &    2 &    3 & \(\sqrt{10}\) &   +198 & \num{-1} &  \num{-3} \\
    12 &    3 &    0 & \(3\sqrt{2}\) &   +225 & \num{-3} &  +3 \\
    13 &    3 &    1 & \(\sqrt{10}\) &   +252 & \num{-3} &  +1 \\
    14 &    3 &    2 & \(\sqrt{10}\) &   +288 & \num{-3} &  \num{-1} \\
    15 &    3 &    3 & \(3\sqrt{2}\) &   +315 & \num{-3} &  \num{-3} \\ \hline
\end{tabular}
\end{center}
\caption{Exempel på 16QAM i kvadraturmodulering}
\label{tab:16QAM}
\end{table*}

Medan både amplituder och vinklar kan kännas udda, så är det enkelt
att mappa bitarna över till I- och Q-amplituder och faslägen via Isym- och
Qsym-delarna av symboler.

QAM-modulering används av DAB, DVB-T, DVB-T2, IEEE~802.11 (Wi-Fi),
mikrovågslänkar och många andra moderna system såsom EDGE
(efterföljaren till GSM med högre datatakt), UMTS när man kör
höghastighet (HSPA) liksom i LTE där man kör relativt långsamma
symboler men i stället väldigt många parallellt fördelat över ett
större frekvensband.
I mobiltelefonisystem använder man bland annat 64QAM och 256QAM.

Mikrovågslänkar använder upp till 2048QAM. En fördel med
QAM-moduleringen är att det är enkelt att få samma avstånd mellan de
olika symbolpositionerna, och därmed kan också modulationen anpassas
till störningen. Detta nyttjas av många moderna modulationssystem så
att QAM-modulationen anpassas utifrån mottagarens rapportering om
störning.  Denna dynamiska anpassning gör att kommunikationen kan
upprätthållas även om kapaciteten tillåts variera.

\subsection{Begrepp vid digital modulation}
\harecsection{\harec{a}{1.8.9}{1.8.9}}
\index{digital modulation}

Digital modulation innebär också att signalerna som sänds har lite
andra egenskaper än de analoga. Istället för varierande
spänningsnivåer som för till exempel tal skickar vi diskreta fixa
nivåer, ofta i form av bitar. Det är därför lämpligt att diskutera
några grundläggande begrepp kring digital modulation.

% k7per
% \newpage % layout

\subsubsection{Bit rate}
\harecsection{\harec{a}{1.8.9a}{1.8.9a}}
\index{bit}
\index{byte}
\index{informationsmängd}
\index{informationsöverföringskapacitet}
\index{bit rate}

Informationen som vi skickar har vi kodat i bitar (eng. \emph{bit (b)}),
\emph{informationsmängden} vi har är därför ett visst antal bitar och takten på
denna informationsmängd blir därmed \emph{informationsöverföringskapaciteten}
(eng. \emph{bit rate}) i bitar per sekund.

Ofta brukar vi referera till informationsmängden som mängden \emph{byte (B)}
som till exempel att en fil är 2\,kB eller en bild är 1,25\,MB.
Då en byte innehåller åtta bitar motsvarar det 16\,kb respektive 10\,Mb.
I dagligt tal talar vi då om storleken på en fil.

Överföringskapaciteten, eller i dagligt tal hastigheten, brukar vi ofta prata
om i termer av \emph{bit rate} som 10\,Mb/s (ofta skrivet \emph{bps -- bits per
second}), dvs. man klarar av att överföra upp till 10 miljoner bitar per sekund.

Det är ofta som man talar om den råa överföringskapaciteten, medan den
verkliga överföringskapaciteten för nyttotrafik är något lägre på grund av
olika former av packningsformat och protokollbehov, så kallad \emph{overhead}.
Man ska därför vara noga med att skilja dessa åt.

\subsubsection{Symboltakt -- Baud rate}
\harecsection{\harec{a}{1.8.9b}{1.8.9b}}
\index{symbol}
\index{symboltakt}
\index{symbol rate}
\index{Baud rate}
\index{Baudot, Emile}
\index{enheter!baud (Bd)}

Som vi redan sett exempel på kan ibland bitar skickas en och en, eller
ihopklumpade. Varje sådan ihopklumpning kallas \emph{symbol}, och en symbol
kan bära en eller flera bitar, ibland inte ens ett jämnt antal.

Om man kan artikulera något i två olika \emph{nivåer} (av amplitud, fas, frekvens
eller kombination), så kan man representera en bit. Om man kan artikulera något
i fyra olika nivåer, kan man representera två bitar. På samma sätt ger åtta nivåer
support för tre bitar. Varje representation kallar man en symbol och varje
symbol bär alltså en, två eller tre bitar information. Strikt räknat
är det logaritmen med bas två (2-logaritm eller $\log_{2}$) av antalet nivåer som
anger antalet bitar som en symbol kan bära.
Tre nivåer brukar sägas kunna bära 1,5~bitar, vilket är en slarvig approximation
men visar principen.

Den takt varmed symboler överförs \emph{symboltakten} (eng. \emph{symbol rate}),
benämns även \emph{Baud rate}, efter Emile Baudot, med enheten \emph{baud (Bd)}.
Enheten baud (förkortat Bd) anger antalet symboler per sekund. Genom att
multiplicera antalet symboler per sekund med antalet bitar per symbol fås
överföringskapaciteten bitar per sekund.

\subsubsection{Bandbredd}
\harecsection{\harec{a}{1.8.9c}{1.8.9c}}
\index{bandbredd}
\index{Nyquist-Shannons samplingsteorem}

Genom att justera antalet bitar per symbol kan man ändra antalet symboler
per sekund utan att ändra överföringskapaciteten. En anledning till att man
vill göra det är att bandbredden som används av en överföring är ungefär
proportionerlig mot symboltakten, det vill säga hur många baud man överför.
Detta påverkar hur stor del av radiospektrat man upptar, och därmed också hur
nära en annan signal man kan ligga i spektrat utan att störa varandra, dvs.
det påverkar frekvensplaneringen av bandet ifråga.

Ofta används begreppet bandbredd synonymt med överföringskapaciteten, eftersom
det finns en proportionell relation dem emellan, men bandbredden är inte den
enda parametern som krävs, så i mer strikta sammanhang ska dessa begrepp
hanteras som separata för att undvika missförstånd.

Bandbredden för en digital ström är relaterad till nyquistteoremet, som säger
att samplingstakten måste vara minst dubbelt så hög som den högsta frekvens
som överförs.

\subsection{Bitfel -- detektion och korrigering}

Hittills har vi diskuterat digital modulation utan att ta hänsyn till
störningar och hur dessa påverkar våra överförda data. Precis som
vår CW eller SSB kan vara störd av atmosfäriska störningar, andra sändare
eller helt enkelt vara svaga signaler så att det interna bruset blir en
begränsning, så kommer mottagningen av digitala signaler att bli störd.
Vi ska titta på dessa grundläggande begrepp såsom bitfel, bitfelssannolikhet,
felupptäckt samt korrigering med återsändning eller korrigeringskoder.

\subsubsection{Bitfel}
\index{bitfel}
\index{bit error}

Av olika orsaker kommer en eller flera bitar ofta att bli fel.
Vi kallar varje sådant fel för att \emph{bitfel} (eng. \emph{bit error}).
Störningar kan göra att vi tolkar en symbol fel, vilket kan resultera i en eller
flera felaktiga bitar.

Om vi i till exempel 16QAM-koden i kapitel \ref{QAM} får in +0.2 i I och +1.1 i Q,
ser vi i tabell \ref{tab:16QAM} att närmaste symbolen är symbol 5 med +1 i I
och +1 i Q. Vi skulle kunna anta att om I är större än 0 och mindre än 2, samt
Q är större än 0 och mindre än 2 så är symbol 5 den enda vettiga symbolen, och
det är precis den tolkning vi i allmänhet gör, för det är den symbolen vars
avstånd är lägst och därmed rimligast.
Det kan dock vara så att man egentligen sände symbol 9 med \num{-1} i I och +1 i
Q, och därmed fick för stor störning på I för att man ska tolka det som rätt
symbol.
Vi kommer då lägga ut 9 istället för 5, vilket innebär att två bitar har
ändrats.

Genom att granska tabell \ref{tab:16QAM} vidare ser man att värdena för
I och Q för de olika symbolerna är gjorda så att minsta avstånd är 2 mellan
alla närliggande symboler, i respektive I- och Q-riktning. Det förenklar
tolkning av symbolerna. Är dock störningen större än 1 i någon riktning
kommer man tolka den symbolen fel, och det kan då leda till 1 eller fler
bitfel.

\subsubsection{Bitfelssannolikhet}
\index{bitfelssannolikhet}
\index{bit error rate}
\index{BER|see {bit error rate}}
\index{gaussiskt brus}
\index{brus!gaussiskt}
\index{gaussian noise}
\index{effektiv-värde}
\index{Root Mean Square}
\index{RMS|see {Root Mean Square}}
\index{Error Function (erf)}
\index{erf}

Om vi antar att vi inte har störning från några andra signaler, utan enbart har
brus som störning, så kan vi estimera \emph{bitfelssannolikheten} (eng.
\emph{bit error rate (BER)}) ur hur starkt bruset är i förhållande till vårt
steg. Eftersom bruset antas vara vitt brus, så har det egenskaperna av
\emph{Gaussiskt brus} (eng. \emph{Gaussian noise}).

Gaussiskt brus har en statistisk fördelning med hög sannolikhet nära
medelvärdet och avtar sedan med avståndet. Sannolikheten att man tolkar en
signal som vara på ena eller andra sidan av en gräns beror på hur långt bort
från medelvärdet den gränsen, ofta benämnd kvantiseringsgränsen, är i
förhållande till den effektiva värdet (eng. \emph{Root Mean Square (RMS)}) i
amplitud hos bruset. Detta kan uttryckas i form av den matematiska funktionen
\emph{error function (erf)}.

När gränsen är 1~sigma, det vill säga 1 gånger RMS-värdet för brusamplituden,
från medelvärdet så är det 67~\% sannolikhet att värdet ligger inom
gränsvärdet, det vill säga en bitfelssannolikhet på 33~\%.
Ligger det inom 2 sigma har sannolikheten ökat till 97~\%, en
bitfelssannolikhet på 3~\%, och vid 3 sigma är den 99,7~\% med en
bitfelssannolikhet på ringa 0,3~\%, vilket ofta används för många
ingenjörsapplikationer. Dock, för överföring av information har vi högre krav.
För en bitfelssannolikhet på \(10^{-12}\), ofta benämnt BER på 1E-12, behövs
det 14~sigma bort till gränsen, dvs. brusmängden får max vara 1/14 av
kvantiseringsgränsen. Den råa radiokanalen uppvisar dock sällan så bra
egenskaper, men det kan uppnås i kabel och fiber.

\subsubsection{Detektion}
\harecsection{\harec{a}{1.8.10a}{1.8.10a}}
\index{bitfelsdetektion}
\index{paritet}
\index{CRC}
\label{bitfel_detektion}

Eftersom störningar förekommer och man har behov av lägre bitfelssannolikhet
än vad den råa kanalen medger är det lämpligt att identifiera när det har
blivit bitfel. Detta kan utföras på många sätt, men ett sätt är att räkna fram
checksummor som skickas med datat. Det kräver visserligen en del av
informationsöverföringskapaciteten, men tjänsten det medger är att försäkra sig
om att informationen är rimligt korrekt.

En enkel form av checksumma är paritet, där bitarna i ett ord har summerats ihop
binärt (med XOR) för att bilda en checksumma. I mottagaränden görs samma
kombination och sedan jämförs det med paritetsbiten, och om de överensstämmer så
har inget bitfel upptäckts. Denna enkla metod har en svaghet i att ett jämnt
antal bitfel kommer att kompensera varandra, varvid det döljer bitfel från
upptäckt.
Det är med andra ord inte en särdeles stark checksumma.
Paritet används till exempel i seriekommunikation så som RS-232.

Ett flertal checksummor finns, för olika ändamål, olika mängd fel och olika
typer av fel. För lite större meddelanden är det vanligt att summera bytes
till en checksumma antingen additivt eller med XOR. För större meddelanden
används en lite mer intrikat metod som heter Cyclic Redundancy Check (CRC)
där man återmatar överskjutande del på checksumman till sig själv och får en
starkare kod den vägen. CRC används till exempel i Ethernet.

\subsubsection{Omsändning}
\harecsection{\harec{a}{1.8.10b}{1.8.10b}}
\index{omsändning}
\index{ARQ}
\index{TCP}

En enkel åtgärd att vidta när man konstaterat att ett block data man
tagit emot har fel, är att begära omsändning. Genom att sändaren håller en
buffert med meddelanden som den skickat, och mottagaren meddelar sändare om
den mottagit meddelandet eller behöver ha det omskickat, så kan omsändning
realiseras. Automatisk omsändningsbegäran (eng. \emph{Automatic Repeat reQuest
(ARQ)}) är en typ av protokoll som gör automatisk omsändningsbegäran om ett
enskilt datablock, även kallat paket, inte kommit fram rätt eller helt
försvunnit.
Ett sådant protokoll är TCP, som ingår i internetsviten av TCP/IP-protokollet.

\subsubsection{Korrigeringskod -- FEC}
\harecsection{\harec{a}{1.8.10c}{1.8.10c}}
\index{korrigeringskod}
\index{felrättandekod}
\index{FEC}
\index{AMTOR}
\index{Hamming-koder}
\index{paritet}
\index{Reed-Solomon (RS)}

En annan form av korrigering är att helt enkelt skicka för mycket data redan
från början, som mottagaren kan använda för att korrigera meddelandet utan att
skicka någon begäran till sändaren. Detta är praktiskt antingen om det skulle
ta för mycket tid eller om det helt enkelt inte finns någon kommunikation från
mottagaren till sändaren, till exempel för satellitmottagare.

En enkel form av felrättande kod används i AMTOR FEC, där man helt enkelt
sänder samma tecken två gånger.
Liknande används i Bluetooth där meddelandet sänds tre gånger, varvid man kan
göra majoritetsröstning.

Andra system för FEC är Hamming-koder, paritets-paket och Reed-Solomon (RS).

%% k7per
%% \newpage % layout

\subsection{Digitala sändningsslag}

Här ges exempel på digitala modulationstekniker för kortvågstillämpningar inom
amatörradio.

De flesta digitala sändningsslagen för kortvåg är smalbandiga och bandbredden
kan i vissa fall endast vara några hertz.

Signalbehandlingen sker i den dator som programmet körs på och där datorns
in- och utgång för dess ljudkort är kopplade till amatöradioutrustningen.
Oftast är programmets styrning av sändning och mottagning också kopplad till
lämplig serieport, till exempel via dess USB-anslutning.

\subsubsection{RTTY}
\index{RTTY}
\index{FSK}
\index{Frequency Shift Keying (FSK)}
\index{Audio Frequency Shift Keying (AFSK)}
\index{AFSK}

\paragraph{Historia}

Ett av de första digitala trafiksätten som användes av radioamatörer var \emph{RTTY},
uttytt ''RadioTeleTYpe'', där man använde sig av så kallade teleprintrar,
automatiska skrivmaskiner som skrev ut text.

Emile Baudot konstruerade år 1874 ett system baserat på fem bitar,
som fortfarande används idag.
I augusti 1922 testade The US Department of Navy ''skriven telegrafi'' mellan
ett flygplan och en markstation.
Amerikanska kommersiella RTTY-system fanns aktiva så tidigt som 1932.
Under 50-talet började surplusutrustning komma ut på den amerikanska
marknaden och radioamatörerna var inte sena att prova den nya tekniken på
kortvågsbanden.
De kommersiella systemen körde med 50~baud, 75~baud eller 100~baud.

Amatörerna i USA körde med 45,45~baud, vilket motsvarar 300 tecken per minut.
De europeiska utrustningarna, bland annat framtagna av Siemens, arbetade med
50~baud men gick att justera ner till 45,45~baud.
45~baud är idag den vedertagna standarden över världen.

\paragraph{Teknik}

RTTY använder FSK-modulering.
För att åstadkomma detta behöver man styra frekvensen så att den hoppar mellan
två frekvenser med en skillnad, ett så kallat ''skift'', på \SI{170}{\hertz}.

Äldre sändare behövde modifieras för att åstadkomma detta frekvensskift, men
med en SSB-sändare kunde man istället mata sändaren med två toner, som gav
samma resultat -- så kallad Audio Frequency Shift Keying (AFSK).

Med nyare amatörradiotransceivrar blev det senare den vanligaste förekommande
metoden att modulera sändaren.
Det innebar att man slapp modifiera utrustningen.

Idag kör de flesta radioamatörer RTTY med en dator och använder sig ofta av
AFSK-tekniken med hjälp av programvaror, med samma uppkoppling som man använder
för andra digitala trafiksätt.

\subsubsection{SSTV}
\index{SSTV}

\emph{Slow Scan Television (SSTV)} är en blandning av analog och digital teknik.
En SSTV sändning sker långsamt jämfört med traditionell TV, men är i grunden
rätt lik.
Varje linje sänds en efter en, men modulerad så att den kan sändas över en
SSB-radiolänk.
Intensiteten för varje pixel anger tonhöjden som moduleras, vilket därmed
innebär en FM-modulation.
Denna FM-modulerade ton skickas sedan över SSB.
I början på varje linje skickas ett 7-bitars tal med jämn paritet som anger
vilken modulationsform som används.
De olika modulationsformerna kan sedan hantera olika upplösningar samt
variera med avseende på svart-vitt eller färg.

\subsubsection{APRS}
\index{APRS}
\index{AX.25}
\index{TNC}
\label{modulation_aprs}

\emph{Automatic Packet Reporting System (APRS)} är en teknik för att
huvudsakligen över VHF och UHF förmedla GPS-position, väderdata, enkla
meddelanden och annat.
Den bygger på en teknik som heter \emph{AX.25}, som är en amatörradiospecifik
version av telekomstandarden X.25.
AX.25 moduleras över 1200 baud Bell 202 AFSK teknik på vanlig talkanal.
Ofta används en Terminal Node Controller (TNC) som gränssnitt mellan dator och
radio.

\subsubsection{PSK31}
\index{PSK31}
\label{modulation_psk31}

\paragraph{Historia}

Namnet beskriver modulationstypen och överföringshastigheten i baud.
Det första programmet utvecklades specifikt för windowsbaserade datorer med
ljudkort av den engelska radioamatören Peter Martinez, G3PLX, och
introducerades i amatörradiovärlden 1998.

\paragraph{Teknik}

Modulationen som används i PSK31 utvecklades från en idé av den polske
radioamatören Pawel Jalocha, SP9VRC, som hade tagit fram en mjukvara
''SLOWBPSK'' för Motorolas EVM-radio, vilket var ett radiosystem för att
utvärdera olika modulationsformer.
Istället för att använda den gängse metoden med frekvensskift baserades
''SLOWBPSK'' på polaritetsskiftning av fasläget.
Ett bra utformat PSK-baserat system kan ge bättre resultat än FSK, och kan
arbeta med smalare bandbredd än FSK.
Överföringshastigeten 31~baud valdes för att passa en genomsnittlig
skrivhastighet hos den gemene amatören.

\subsubsection{WSPR}
\index{WSPR}

\paragraph{Historia}

WSPR släpptes i sin första version 2008.
Programmet skrevs initialt av Joe Taylor, K1JT, men är nu ett
open source-program och utvecklas av ett litet team.
Joe Taylor fick sin utbildning i astronomi vid Harvard University.
Han var sedan verksam inom området astrofysik vid Princeton University,
varifrån han pensionerades 2006.
Joe Taylor tilldelades Nobelpriset i fysik år 1993.

Programmet är i huvudsak tänkt för vågutbredningstester inom kortvågsområdet.

\paragraph{Teknik}

WSPR står för Weak Signaling Propagation Reporter och uttalas ''Whisper''.
WSPR är ett sändningsslag som använder amatörradiostationen som en radiofyr, en
så kallad beacon. Sändning och mottagning sker i tvåminuterspass och efter varje
sändningspass rapporterar de stationer som mottagit signalen in sitt resultat till en
databas över internet.
Den sändande stationen kan därefter studera resultatet.
WSPR använder låga effekter, det går att nå europeiska stationer med effekter
under \SI{100}{\milli\watt}, och andra kontinenter med effekter under några
watt, även med modesta antenner.

\subsubsection{WSJT}
\index{WSJT}
\index{FSK441}
\index{JT6M}
\index{JT65}
\index{JT9}
\index{FT8}
\index{FSK}
\index{Frequency Shift Keying (FSK)}
\index{meteorer}
\index{troposcatter}
\index{EME}
\index{månstuds}
\index{8FSK}

WSJT är liksom WSPR ett program som används inom amatörradiohobbyn för så
kommunikation med svaga signaler.
Även detta program är utvecklat av Joe Taylor, K1JT.
De flesta av dessa sändningsslag (se nedan) är så smalbandiga, att de inte upptar
större bandbredd än några hertz.

\paragraph{Historia}

WSJT presenterades för amatörradiovärlden år 2001 och har undergått ett flertal
revisioner.
Olika sändningsslag har under åren lagts till och tagits bort.
Sedan 2005 har programmet öppen källkod och utvecklas av ett litet team.

\paragraph{Teknik}

WSJT erbjuder en plattform för ett flertal olika tillämpningar där olika
varianter av i huvudsak FSK-modulering används.

FSK441 används för att utvärdera överföringar via radiovågsreflekterande skikt
av laddade joner, som uppkommer från de spår som meteorer lämnar efter sig.

JT6M introducerades år 2002 och är avsett för kommunikation via bland annat
meteorreflektioner på \SI{6}{\metre}-bandet.

JT65, utvecklat och släppt år 2003, används för kommunikation via troposfären,
så kallat troposcatter, men också för kommunikation via reflektion mot månen
så kallad EME-trafik.

JT9 används för kortvågstrafik och är snarlikt JT65, men använder sig av en
FSK-signal med nio toner.
JT9 använder sig av mindre än \SI{16}{\hertz} bandbredd.

FT8 utvecklades och släpptes år 2017 och använder sig av en 8FSK-signal.

FT8 är att föredra vid så kallat multi-hop via E-skikt, där signalerna utsätts för
fädning och där öppningarna mot andra stationer är korta så att man behöver
slutföra kommunikationen inom en kort tid.

\subsubsection{FreeDV}
\index{FreeDV}

FreeDV skiljer sig mot de sändningsslag som nämnts ovan genom att detta är tänkt
för digitalt tal på kortvåg.

\paragraph{Historia}

FreeDV skapades av en grupp radioamatörer från olika länder som arbetade
med kodning, utformning, användargränssnitt och testning.
FreeDV släpptes år 2015.

\paragraph{Teknik}

FreeDV är tänkt att användas på kortvåg med SSB-modulerade radiostationer,
men kan också användas med AM- eller FM -modulering.
Fördelen ska vara att överföringen blir ner robust samt att signaleringen är
utformad för att motverka påverkan av fädning.

FreeDV använder en något mer komplex modulering.
Man använder sig av ett flertal bärvågor inom dess bandbredd på
\SI{1,25}{\kilo\hertz}.
Bärvägorna ligger med \SI{75}{\hertz} mellanrum och varje bärvåg moduleras med
varianter av PSK-modulering.
Bandbredden är hälften (\SI{1,25}{\kilo\hertz}) av en normal SSB-bandbredd
(\SI{2,4}{\kilo\hertz}).
