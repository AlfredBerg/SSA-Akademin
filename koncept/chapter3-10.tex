\section{Digital signalbehandling}
\harec{a}{3.8}{3.8}
\index{digital signalbehandling}
\index{digital signal processing (DSP)}

I takt med att utvecklingen gjort avancerade kretsar allt billigare har det
blivit allt vanligare med olika former av digital signalbehandling, och dessa
används i varierande grad även i radiodesign.

Ofta sammanfattas det med termen \emph{digital signalbehandling} (eng.
\emph{Digital Signal Processing (DSP)}).

Ofta förväxlas det begreppet med Digital Signal Processor (DSP), som kommit att
representera en typ av processorer anpassade för signalbearbetning.
Dock är begreppet vidare än så, och vilken annan form av digital processing är
också en digital signalprocessing.

\subsection{Digitala filter}
\harec{a}{3.8.1}{3.8.1}
\label{digitala filter}
\index{digitala filter}
\index{filter!digitala}
\index{FIR}
\index{filter!FIR}
\index{IIR}
\index{filter!IIR}

Eftersom en signal så som den representeras för digitala kretsar måste vara
samplad och kvantiserad, så kommer signalen att ofrånkomligen bestå av ett
antal sampel med ett visst antal bitar för dess PCM värde.

Att ändra nivån på en sådan signal görs genom att multiplicera den med något
värde, det vill säga låta varje enskilt sampel i tur och ordning multipliceras
med samma värde, men det ändrar inga egenskaper i frekvensen.
För att få en påverkan med avseende på frekvens behöver man kombinera värdena
från flera olika tidpunkter i signalen, och ofta väljer man att låta de
vägas samman med olika vikt.
Detta görs genom att helt enkelt fördröja samplen i flera steg,
multiplicera varje fördröjning med sin vikt-konstant och sedan summera
resultatet.

Det filter som man då skapat kallas för ett Finite Impulse Response (FIR)
filter, för skickar man in en puls på ingången så kommer den att fördröjas
stegvis och ge svaret från var och en av multiplikatorerna, i var sitt sampel,
till dess att fördröjningskedjan är slut, varvid den impulsen inte ger något
mer bidrag till utgången.
Den räcka med sampel som kom från impulsen kallas för impulsresponsen, och
eftersom den tar slut är den finit, därav namnet.

Man kan göra en variant av det här där man helt enkelt låter en annan
uppsättning med multiplikatorer väga samma fördröjda sampel, men där det
summerade svaret återmatas till ingången och adderas där innan
fördröjningskedjan.
Detta kallas för Infinite Impulse Response (IIR) filter, för att det i likhet
med FIR-filter har en impulsrespons, men eftersom den återmatar så kan denna
rent teoretiskt pågå i all oändlighet, det vill säga engelskans infinite.
I praktiken designas filter så att de inte pågår i evinnerlig tid utan, så att
säga ringer ut.
Själva arkitekturen är dock väldigt lämplig att använda för många ändamål.

Utöver själva filterstrukturen, det vill säga IIR och FIR, så karakteriseras de
av hur många fördröjningssteg man har, då det representerar hur komplext
filtret är, samt av koefficienterna som ger responsen hos filtret.
Design av filterkoefficienter skiljer markant för IIR och FIR, och det finns
både enkla och avancerade verktyg för det.

Ett specialfall på FIR-filter är när koefficienterna är speglade runt mitten.
Då kan man matematiskt visa att de har egenskapen av linjär fas (eng. linear
phase filter), och de har enbart påverkan på amplituden.
En fördel med sådana filter, som är fas-linjära, är att olika frekvensers
signal upplever samma grupp-fördröjning och därmed inte förskjuts i förhållande
till varandra.
Detta brukar bland annat öka taltydligheten.

\subsection{Fouriertransform (FFT)}
\harec{a}{3.8.2}{3.8.2}
\index{Fouriertransform}
\index{Fourier!DFT}
\index{Fourier!FFT}
\index{diskret fouriertransform}
\index{Discrete Fourier Transform (DFT)}
\index{DFT}
\index{FFT}
\index{Fast Fourier Transform (FFT)}

En specifik form av processing som blivit tillgänglig är fouriertransform,
det vill säga förmågan att omvandla från signalstyrka över tid till
signalstyrka över frekvens.
Eftersom processingen sker i diskret tid, det vill säga värden med en viss tid
emellan, så som ofrånkomligt med samplade värden, så är det ett specialfall av
fouriertransform, som därför heter \emph{diskret fouriertransform} (eng.
\emph{Discrete Fourier Transform (DFT)}).

DFT kan göras på alla möjliga längder av sekvenser, men är beräkningstungt
om man vill ha alla möjliga frekvenser.
För att reducera beräkningsmängden kan man givetvis beräkna DFT bara för ett
fåtal frekvenser, men när det inte är applicerbart behöver man agera lite
smartare.
Så som DFT är formulerat, så ger matematiken flera genvägar, som gör att man
på flera olika sätt kan slå samman beräkningarna och göra delberäkningar som
kan användas av flera andra steg, och på så sätt minska beräkningsbördan.
Detta kan sedan göras hierarkiskt, så att en rekursiv form kan göras.
Det finns flera metoder att göra detta på, men de sammanfattas med som en snabb
DFT, det vill säga \emph{Fast Fourier Transform (FFT)}, som även den är diskret.
En nackdel med FFT är man ofta hamnar på jämna tvåpotenser i antalet sampel,
till exempel 512, 1024, 2048, 4096 sampel och frekvenser.
Man har därmed offrat lite av DFT:ns generalitet.

Det finns mer avancerade formuleringar av FFT som utnyttjar ett eller annat
trick för att jämna ut till fler storlekar, genom att inte bara göra
kombination om 2 sampel, utan även 3, 5 och så vidare, som sedan kan kombineras
till flera storlekar.
Ett annat trick är att helt enkelt fylla på med bara nollor efter, och köra
med en för stor FFT.

Oavsett hur fourieranalysen görs, medger den att man fort kan få upp ett
spektrum.
Detta används nu mer allt oftare för att få en spektrumplot och genom att
lägga flera av dessa efter varandra kan man få de nu mer allt vanligare
spektrumhistogrammen även kända som vattenfallsplottar då de påminner om ett
vattenfall med sina vertikala streck.

\subsection{Direct Digital Synthesis (DDS)}
\harec{a}{3.8.3}{3.8.3}
\index{Direct Digital Synthesis (DDS)}
\index{DDS}
\index{fasackumulator}
\index{phase accumulator (PA)}
\index{PA}
\index{nyquistfrekvens}

En term som kommit starkt på senare år är \emph{Direct Digital Synthesis (DDS)}.
Detta syftar på att man kan istället för som med en PLL indirekt styra en
oscillator direkt syntetisera en vågform, och man kan göra det med väldigt
hög upplösning och ändra den väldigt fort.
Medan det kan göras på många sätt, så är den dominerande principen den att
man gör en oscillator med en så kallad \emph{fasackumulator} (eng. \emph{phase
accumulator (PA)}).
En fasackumulator är inget annat än ett adderingssteg följt av en delay-steg.
Det är ett extremfall av ett IIR filter, med enbart en pol, som integrerar,
det vill säga ackumulerande effekt.
Värdet ut från denna representerar oscillatorns fas, där av phase accumulator.
Frekvensen styrs helt enkelt med ett värde som anger hur mycket fasen ska
ökas för varje sampel.
Frekvensen blir därför helt linjär, så när som på steg-upplösningen, och kan
varieras fort och fritt.
Upplösningen avgörs därför av hur många bitar bred som hela ackumulatorn har.
Högsta frekvensen blir \emph{Nyquistfrekvensen}, det vill säga halva
samplingsfrekvensen och lägsta blir den som minst signifikanta biten ger.

Den utgående fasen ur själva fasackumulatorn vågformas sedan om till sinus,
cosinus eller vad man nu önskar.
Det går även att använda en uppslagstabell för att kunna syntetisera godtyckliga
vågformer.

Idag finns det färdiga kretsar som ger väldigt stort frekvensområde med 32, 48,
eller fler bitars upplösning.
Inte helt sällan används DDS i kombination med mer klassiska PLL lösningar
för att få bra egenskaper.

DDS har skapat en enorm frihet i hur radioapparater kan designas, och det har
bidragit enormt till både prestanda och miniatyrisering.
