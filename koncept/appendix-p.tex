\onecolumn

\chapter{Bandplaner}
\label{bandplaner2}
\index{bandplan}

\noindent Dessa bandplaner presenteras efter svenska förhållanden så långt de
är kända vid tillfället. Kontrollera alltid de senaste bandplanerna på SSA för
att säkerställa du använder de senaste.

Effektbegränsningen är generellt på alla amatörband i Sverige \SI{200}{W} PEP
tillfört antennen om inte annat anges. På vissa band finns det ytterligare
restriktioner. Tillstånd kan sökas från PTS för högre effekt, i nuläget har
tillstånd på upp till \SI{1}{kW} delats ut från PTS.

\section{Bandplan LF/MF/HF}
\index{bandplan!hf}

Alla frekvenser i kHz, bandbredder i Hz.

\subsubsection{Bandplan 2.2\,km, 135,7--137,8\,kHz}
\begin{tabular}{rrrll}
\multicolumn{2}{c}{\textbf{Frekvens}} & \textbf{BW} & \textbf{Trafik} & \textbf{Noteringar} \\ \hline
135,7 & 135,8 & 200 & CQ, QRSS, Digi & OBS! Högsta effekt \SI{1}{W} ERP. \\ \hline
\end{tabular}

\subsubsection{Bandplan 600\,m, 472--479\,kHz}
\begin{tabular}{rrrll}
\multicolumn{2}{c}{\textbf{Frekvens}} & \textbf{BW} & \textbf{Trafik} & \textbf{Noteringar} \\ \hline
472 & 479 & 200 & CW, QRSS, Digi & OBS! Högsta utstrålad effekt \SI{1}{W} EIRP \\ \hline
\end{tabular}

\subsubsection{Bandplan 160\,m, 1810--2000\,kHz}
\begin{tabular}{rrrll}
\multicolumn{2}{c}{\textbf{Frekvens}} & \textbf{BW} & extbf{Trafik} & \textbf{Noteringar} \\ \hline

1810 & 1838 & 200  & CW         & Exklusivt för CW. Interkontinental trafik har prio. \\ \hline
1838 & 1840 & 500  & Smalband   & Ej packet på \SI{160}{\meter}, PSK \num{1838,15}         \\ \hline
1840 & 1850 & 2700 & Alla moder & Även digimode. SSB QRP \num{1843}                  \\ \hline
1850 & 1900 & 2700 & Alla moder & OBS! Max \SI{10}{\watt} till ant.                       \\ \hline
1900 & 1950 & 2700 & Alla moder & OBS! Max \SI{100}{\watt} till ant.                      \\ \hline
1950 & 2000 & 2700 & Alla moder & OBS! Max \SI{10}{\watt} till ant.                       \\ \hline
\end{tabular}

\subsubsection{Bandplan 80\,m, 3500--3800\,kHz}
\begin{tabular}{rrrll}
\multicolumn{2}{c}{\textbf{Frekvens}} & \textbf{BW} & \textbf{Trafik}
& \textbf{Noteringar} \\ \hline

3500 & 3510 & 200  & CW             & Exklusivt CW                                    \\ 
     &      &      &                & Interkontinental DX-trafik har prio             \\ \hline
3510 & 3580 & 200  & CW             & Exklusivt CW contest \SIrange{3510}{3560}{\kilo\hertz}  \\ 
     &      &      &                & CW QRS \num{3555}, CW QRP \num{3560}            \\ \hline
3580 & 3600 & 500  & Smalband, Digi & PSK \num{3580,15}                               \\
     &      &      &                & Automatiska Digimoder \SIrange{3590}{3600}{\kilo\hertz} \\ \hline
3600 & 3620 & 2700 & Alla moder     & Digimoder Automatiska Digimoder                 \\ \hline
3600 & 3650 & 2700 & Alla moder     & SSB contest \SIrange{3600}{3650}{\kilo\hertz}           \\
     &      &      &                & DV \num{3630}                                   \\ \hline
3650 & 3700 & 2700 & Alla moder     & SSB QRP \num{3690}                              \\ \hline
3700 & 3800 & 2700 & Alla moder     & Contest \SIrange{3700}{3800}{\kilo\hertz}               \\
     &      &      &                & Image \num{3775}                                \\
     &      &      &                & Region 1 hjälpinsatsfrekvens \num{3760}         \\ \hline
3775 & 3800 & 2700 & Alla moder     & Interkontinental DX-trafik prioritet            \\ \hline
\end{tabular}

\subsubsection{Bandplan 40\,m, 7000--7200\,kHz}
\begin{tabular}{rrrll}
\multicolumn{2}{c}{\textbf{Frekvens}} & \textbf{BW} & \textbf{Trafik} & \textbf{Noteringar} \\ \hline

7000 & 7040 & 200  & CW         & Exklusivt CW.                                        \\
     &      &      &            & QRP aktivitetscentrum \SI{7030}{}                    \\ \hline
7040 & 7050 & 500  & Smalband   & Digimoder Automatiska inom \SIrange{7047}{7050}{\kilo\hertz} \\ \hline
7050 & 7060 & 2700 & Alla moder & Digimoder Automatiska inom \SIrange{7050}{7053}{\kilo\hertz} \\ \hline
7060 & 7100 & 2700 & Alla moder & SSB contest i segmentet                              \\
     &      &      &            & DV \num{7070}, SSB QRP \num{7090}                    \\ \hline
7100 & 7130 & 2700 & Alla moder & Region 1 hjälpinsatsfrekvens \num{7110}           \\ \hline
7130 & 7200 & 2700 & Alla moder & SSB contest i segmentet                              \\
     &      &      &            & Image \num{7165}                                     \\ \hline
7175 & 7200 & 2700 & Alla moder & Interkontinental DX-trafik prio                      \\ \hline

\end{tabular}
\subsubsection{Bandplan 30\,m, 10\,100--10\,150 kHz}
\begin{tabular}{rrrll}
\multicolumn{2}{c}{\textbf{Frekvens}} & \textbf{BW} & \textbf{Trafik} & \textbf{Noteringar} \\ \hline

10\,100 & 10\,140 & 200 & CW       & CW exkl. Max \SI{150}{\watt} på \SI{30}{m} \\
        &         &     &          & CW QRP \num{10116}                     \\ \hline
10\,140 & 10\,150 & 500 & Smalband & Digimoder PSK \num{10142,15} Ej Packet \\ \hline
\end{tabular}

\subsubsection{Bandplan 20\,m, 14\,000--14\,350 kHz}
\begin{tabular}{rrrll}
\multicolumn{2}{c}{\textbf{Frekvens}} & \textbf{BW} & \textbf{Trafik}
        & \textbf{Noteringar} \\ \hline
        
14\,000 & 14\,070 & 200  & CW         & Exklusivt CW                                        \\
        &         &      &            & Contest \SIrange{14000}{14060}{\kilo\hertz}                 \\
        &         &      &            & CW QRS \num{14055}, CW QRP \num{14060}              \\ \hline
14\,070 & 14\,099 & 500  & Smalband   & PSK \num{14070,15}                                  \\
        &         &      &            & Auto Digimoder \SIrange{14089}{14099}{\kilo\hertz}          \\ \hline
14\,099 & 14\,101 & 200  & Fyrar      & Exklusivt IBP, endast fyrar                         \\ \hline
14\,101 & 14 \,12 & 2700 & Alla moder & Digitala moder och obevakade Digimoder              \\ \hline
14\,112 & 14\,350 & 2700 & Alla moder & SSB Contest \SIrange{14125}{14300}{\kilo\hertz}             \\
        &         &      &            & DV \SI{14130}{\kilo\hertz}, DXpedition prio \SI{14195(5)}{} \\ \hline
14\,300 & 14\,350 & 2700 & Alla moder & Image \SI{14230}{\kilo\hertz}, SSB QRP \num{14285}          \\
        &         &      &            & Global hjälpinsatsfrekvens \num{14300}              \\ \hline
\end{tabular}

\subsubsection{Bandplan 17\,m, 18\,068--18\,168 kHz}

\begin{tabular}{rrrll}
\multicolumn{2}{c}{\textbf{Frekvens}} & \textbf{BW} & \textbf{Trafik}
        & \textbf{Noteringar} \\ \hline
        
 18\,068 & 18\,095 & 200  & CW         & CW exklusivt. QRP \num{18086}                             \\ \hline
18\,095  & 18\,109 & 500  & Smalband   & Digimoder PSK \SI{18100,15}{}                             \\
         &         &      &            & Automatiska Digimoder \SIrange{18105}{18109}{\kilo\hertz} \\ \hline
18\,109  & 18\,111 & 200  & Fyrar      & Exklusivt fyrar, IBP fyrnät                               \\ \hline
18\,111  & 18\,168 & 2700 & Alla moder & Digi \SIrange{18111}{18120}{\kilo\hertz}                  \\
         &         &      &            & SSB QRP \num{18130}, DV \num{18150}                       \\
         &         &      &            & Global hjälpinsatsfrekvens \num{18160}                    \\ \hline
\end{tabular}

\subsubsection{Bandplan 15\,m, 21\,000--21\,450 kHz}

\begin{tabular}{rrrll}
\multicolumn{2}{c}{\textbf{Frekvens}} & \textbf{BW} & \textbf{Trafik} & \textbf{Noteringar} \\ \hline

21\,000 & 21\,070 & 200  & CW         & Exklusivt CW, QRS \num{21055}, CW QRP \num{21060}                                          \\ \hline
21\,070 & 21\,110 & 500  & Smalband   & PSK \num{21080,15}, Automatiska Digimoder \SIrange{21090}{21110}{\kilo\hertz}              \\
21\,110 & 21\,120 & 2700 & Alla moder & Alla moder utom SSB!                                                                       \\
        &         &      &            & Digimoder, och Automatiska Digimoder                                                       \\ \hline
21\,120 & 21\,149 & 500  & Smalband   &                                                                                            \\ \hline
21\,149 & 21\,151 & 200  & Fyrar      & Exklusivt fyrar. IBP fyrnät                                                                \\ \hline
21\,151 & 21\,450 & 2700 & Alla moder & DV \SI{21180}{\kilo\hertz}, SSB QRP \SI{21285}{\kilo\hertz}, Image \SI{21340}{\kilo\hertz} \\
        &         &      &            & Global hjälpinsatsfrekvens \SI{21360}{\kilo\hertz}                                         \\ \hline
\end{tabular}

\subsubsection{Bandplan 12\,m, 24\,890--24\,990 kHz}
\begin{tabular}{rrrll}
\multicolumn{2}{c}{\textbf{Frekvens}} & \textbf{BW} & \textbf{Trafik} & \textbf{Noteringar} \\ \hline

24\,890 & 24\,915 & 200  & CW         & Exklusivt CW, QRP \num{24906}                      \\ \hline
24\,915 & 24\,929 & 500  & Smalband   & PSK \num{24920,15}, Automatiska
                                        Digimoder \SIrange{24925}{24929}{\kilo\hertz}      \\ \hline
24\,929 & 24\,931 & 200  & Fyrar      & Fyrar, IBP fyrnät                                  \\ \hline
24\,931 & 24\,990 & 2700 & Alla moder & Auto Digimoder \SIrange{24931}{24940}{\kilo\hertz} \\
        &         &      &            & SSB QRP \num{24950}, DV \num{24960}{}              \\ \hline
\end{tabular}

\subsubsection{Bandplan 10\,m, 28\,000-29\,700 kHz}
\begin{tabular}{rrrll}
\multicolumn{2}{c}{\textbf{Frekvens}} & \textbf{BW} & \textbf{Trafik} & \textbf{Noteringar} \\ \hline

28\,000 & 28\,070 & 200  & CW         & Exklusivt CW, QRS \num{28055}, CW QRP \num{28060}                  \\ \hline
28\,070 & 28\,190 & 500  & Smalband   & PSK \num{28120,15} Auto Digimoder inom \SIrange{28120}{28150}{\kilo\hertz} \\ \hline
28\,190 & 28\,199 & 200  & Fyrar IBP  & Regionala fyrar med tidsdelning                                    \\ \hline
28\,199 & 28\,201 & 200  & Fyrar IBP  & IBP fyrnät                                                         \\ \hline
28\,201 & 28\,225 & 200  & Fyrar IBP  & kontinuerligt sändande fyrar                                       \\ \hline
28\,225 & 28\,300 & 2700 & Alla moder & Övriga fyrar                                                       \\ \hline
28\,300 & 28\,320 & 2700 & Alla moder & Digimoder och Automatiska Digimoder                                \\ \hline
28\,320 & 29\,100 & 2700 & Alla moder & DV \num{28330}, SSB QRP \num{28360}, Image \num{28680}{}           \\
29\,100 & 29\,200 & 6000 & Alla moder & FM simplex, \SI{10}{\kilo\hertz} kanaler                                   \\
        &         &      &            & Maximalt $\pm$2,5\,kHz dev., max 2,5\,kHz mod.frek.                \\ \hline
29\,200 & 29\,300 & 6000 & Alla moder & Digimoder och Automatiska Digimoder                                \\ \hline
29\,300 & 29\,510 & 6000 & Satellit   & Nerlänk fr. satellit. EJ SÄNDNING I SEGMENTET                      \\ \hline
29\,510 & 29\,520 & 6000 & Skydd      & Skyddsfrekvens för satelliter. EJ SÄNDNING I SEGMENTET             \\ \hline
29\,520 & 29\,590 & 6000 & Alla moder & FM Repeater in RH1--8, \SI{100}{\kilo\hertz} duplex, \SI{2,5}{\kilo\hertz} NBFM    \\ \hline
29\,600 & 29\,620 & 6000 & Alla moder & FM simplex, anrop \num{29600}                                      \\
        &         &      &            & FM simplex repeater \num{29610}                                    \\ \hline
29\,620 & 29\,700 & 6000 & Alla moder & FM Repeater ut RH1--8, \SI{100}{\kilo\hertz} duplex                       \\ \hline
\end{tabular}
\newpage

\section{Bandplan VHF--UHF}
\index{bandplan!vhf}
\index{bandplan!uhf}

\begin{tabular}{rrrll}
\textbf{Frekvens} &        & \textbf{BW} & \textbf{Trafik}
& \textbf{Noteringar} \\ \hline

50.000 & 50.100 & 500 Hz  & CW          & \textbf{CW anrp. 50.050 och 50.090 (interkont.)}                                    \\ \hline
50.100 & 50.130 & 2.7 kHz & CW, SSB     & Interkontinental DX-trafik. Ej QSO inom Europa                                      \\ \hline
50.100 & 50.200 & 2.7 kHz & CW,SSB      &
DX \SIrange{50,110}{50,130}{\mega\hertz}, \textbf{\num{50,150}} anrop (interkont.)  \\ \hline
50.200 & 50.300 & 2.7 kHz & CW,SSB      & Generell användning. \num{50,285} för crossband                                     \\ \hline
50.300 & 50.400 & 2.7 kHz & CW, MGM     & PSK \num{50,305}, EME \SIrange{50,310}{50,320}{\mega\hertz}                         \\
       &        &         &             & MS \SIrange{50,350}{50,380}{\mega\hertz}                                            \\ \hline
50.400 & 50.500 & 1 kHz   & CW, MGM     & Endast fyrar, \SI{50401(500)}{Hz} WSPR-fyrar                                        \\ \hline
51.210 & 51.390 & 12 kHz  & FM          & Repeater in, 20/10 kHz kanalavstånd                                                 \\
       &        &         &             & RF81 – RF99                                                                         \\ \hline
50.500 & 52.000 & 12 kHz  & Alla moder  & SSTV \num{50,510}, RTTY \num{50,600}, FM \num{51,510}                               \\ \hline
51.810 & 51.990 & 12 kHz  & FM Repeater & Repeater ut, 20/10 kHz kanalavstånd                                                 \\
       &        &         &             & RF81 – RF99                                                                         \\ \hline
\end{tabular}

\subsubsection{Bandplan 2\,m 144--146\,MHz}
\begin{tabular}{rrrll}
\textbf{Frekvens} &           & \textbf{BW} & \textbf{Trafik}
& \textbf{Noteringar}                                             \\ \hline

144.0000 & 144.1100  & 500 Hz  & CW, EME      & \textbf{CW anrop \num{144,050}}                                 \\
         &           &         &              & MS random \num{144,100}                                         \\ \hline
144.1100 & 144.1500  & 500 Hz  & CW, MGM      & EME MGM \numrange{144,120}{144,160}                             \\
         &           &         &              & PSK31 cent. \num{144,138}                                       \\ \hline
144.1500 & 144.1800  & 2.7 kHz & CW, SSB, MGM & EME \numrange{144,150}{144,160}                                 \\
         &           &         &              & MGM \numrange{144,160}{144,180} anrop \num{144,170}             \\ \hline
144.1800 & 144.3600  & 2.7 kHz & CW, SSB, MGM & MS SSB random \numrange{144,195}{144,205}                       \\
         &           &         &              & \textbf{SSB anrop \num{144,300}}                                \\ \hline
144.3600 & 144.3990  & 2.7 kHz & CW, SSB, MGM & MS MGM random anrop \num{144,370}                               \\ \hline
144.4000 & 144.4900  & 500 Hz  & Fyr          & Exklusivt segment fyrar, ej QSO                                 \\ \hline
144.5000 & 144.7940  & 20 kHz  & All mode     & SSTV, RTTY, FAX, ATV                                            \\
         &           &         &              & Linjära transpondrar                                            \\ \hline
144.7940 & 144.9625  & 12 kHz  & MGM          & APRS \num{144,800}                                              \\ \hline
144.9750 & 145.19350 & 12 kHz  & FM, DV       & Rpt in \numrange{144,9750}{145,1935}                            \\
         &           &         &              & RV46–-RV63, \SI{12,5}{\kilo\hertz}, \SI{600}{\kilo\hertz} skift \\ \hline
145.1940 & 145.2060  & 12 kHz  & FM rymd      & \num{145,200} för kom. m. bem. rymdfark.                        \\ \hline
145.2060 & 145.5625  & 12 kHz  & FM, DV       & FM \numrange{145,2125}{145,5875}  V17–V47                       \\
         &           &         &              & \textbf{FM anrop 145,500}, RTTY 145.300                         \\
         &           &         &              & FM Inet-GW \num{145,2375}, \num{145,2875}, \num{145,3375}       \\
         &           &         &              & DV anrop \num{145,375}                                          \\ \hline
145.5750 & 145.7935  & 12 kHz  & FM, DV       & Rpt ut \numrange{145,575}{145,7875}                             \\
         &           &         &              & RV46–RV63, \SI{12,5}{\kilo\hertz} kanalavstånd                  \\ \hline
145.794  & 145.806   & 12 kHz  & FM Rymd      & \num{145,800}, \SI{145,200} dplx m. bem. rymdfark.              \\ \hline
145.806  & 146.000   & 12 kHz  & All mode     & Exklusivt satellit                                              \\ \hline
\end{tabular}

\subsubsection{Bandplan 70\,cm 432--438\,MHz}
\begin{tabular}{rrrll}
	\textbf{Frekvens} &          & \textbf{BW} & \textbf{Trafik} & \textbf{Anmärkning}   \\ \hline
	
	
432.0000 & 432.0250 & 500 Hz  & CW           & EME exklusivt.                                               \\ \hline
432.0250 & 432.1000 & 500 Hz  & CW, PSK31    & CW mellan \num{432,000}{432,085}, \textbf{CW anrop 432,050}  \\
         &          &         &              & PSK31 \num{432,088}                                          \\ \hline
432.1000 & 432.3990 & 2.7 kHz & CW, SSB, MGM & \textbf{SSB anrop \num{432,200}}                             \\
         &          &         &              & Mikrovåg talkback \num{432,350}, FSK441 \num{432,370}        \\ \hline
432.4000 & 432.4900 & 500 Hz  & Fyr          & Exklusivt segment för fyrar                                  \\ \hline
432.5000 & 432.5940 & 12 kHz  & All mode     & Linjära transpondrar IN \numrange{432,500}{432,600}          \\ \hline
432.5000 & 432.5750 & 12 kHz  & All mode     & NRAU Digital rep. in \numrange{432,500}{432,575} 2 MHz skift \\ \hline
432.5940 & 432.9940 & 12 kHz  & All mode     & Linjära transpondrar ut \numrange{432,600}{432,800}          \\ \hline
432.5940 & 432.9940 & 12 kHz  & FM           & Rep. in \numrange{432,600}{432,975} RU368--398 2,0 MHz skift \\ \hline
432.9940 & 433.3810 & 12 kHz  & FM           & Rep. in \numrange{433,000}{433,375} RU368--398 1,6 MHz skift \\ \hline
433.3940 & 433.5810 & 12 kHz  & FM           & SSTV (FM/AFSK) \num{433,400}                                 \\
         &          &         &              & FM simplex U272--286 \textbf{anrop \num{433,500}}            \\ \hline
433.6000 & 434.0000 & 20 kHz  & All mode     & RTTY (FM/AFSK) \num{433,600}                                 \\
         &          &         &              & FAX \num{433,700}, APRS \num{433,800}                        \\ \hline
434.0000 & 434.4940 & 20 kHz  & All mode     & NRAU Dig. kanaler \num{433,450}, \num{434,475}               \\ \hline
434.5000 & 434.5940 & 20 kHz  & All mode     & NRAU Dig. rep. ut \numrange{434,500}{434,575} 2,0 MHz skift  \\ \hline
434.5940 & 434.9810 & 12 kHz  & FM           & NRAU Rep. ut \numrange{434,600}{434,975} RU 368--RU398       \\
         &          &         &              & \SI{12,5}{\kilo\hertz} med \SI{2,0}{\mega\hertz} skift       \\ \hline
435.0000 & 438.0000 & 20 kHz  & All mode     & Exklusivt satellit
\end{tabular}

\subsubsection{Bandplan 23\,cm 1240--1300\,MHz}
\begin{tabular}{rrrll}
	\textbf{Frekvens} &          & \textbf{BW} & \textbf{Trafik} & \textbf{Anmärkning}          \\ \hline
        
1240.000 & 1243.250 & 20 kHz  & Alla moder   & \numrange{1240,000}{1241,000} Digital kommunikation           \\ \hline
1243.250 & 1260.000 & 20 kHz  & ATV och Data & Repeater ut \numrange{1258,150}{1259,350}  R20--68            \\ \hline
1260.000 & 1270.000 & 12 kHz  & Satellit     & Endast för satelliter alla moder                              \\ \hline
1270.000 & 1272.000 & 20 kHz  & Alla moder   & Repeater in, \numrange{1270,025}{1270,700} RS1--28            \\
	 &          &         &              & Packet RS29--RS50                                             \\ \hline
1272.000 & 1290.994 & 20 kHz  & ATV och Data & Amatörtelevision ATV                                          \\ \hline
1290.994 & 1291.481 & 20 kHz  & FM och DV    & Repeater in Repeat. in \numrange{1291,000}{1291,475}          \\
	 &          &         &              & RM0 – RM19, \SI{25}{\kilo\hertz} \SI{6}{\mega\hertz} skift    \\ \hline
1291.494 & 1296.000 & 12 kHz  & Alla moder   &                                                               \\ \hline
1296.000 & 1296.150 & 500 Hz  & CW,  MGM     & EME \numrange{1296,000}{1296,025} \textbf{CW anrop 1296.050}  \\
	 &          &         &              & PSK31 \SI{1296,138}{\mega\hertz}                              \\ \hline
1296.150 & 1296.400 & 2.7 kHz & CW, SSB, MGM & \textbf{SSB anrop \num{1296,200}}                             \\
	 &          &         &              & \textbf{FSK441 MS anrop \num{1296,370}}                       \\ \hline
1296.400 & 1296.600 & 2.7 kHz & CW, SSB, MGM & Linjära transpondrar infrekvens                               \\ \hline
1296.600 & 1296.800 & 2.7 kHz & CW, SSB, MGM & SSTV/FAX \num{1296,500} MGM/RTTY \num{1296,600}               \\ \hline
1296.600 & 1296.800 & 2.7 kHz & CW, SSB, MGM & Linjära transpondrar utfrekvens                               \\
	 &          &         &              & \numrange{1296,750}{1296,800} lokala fyrar max \SI{10}{\watt} \\ \hline
1296.800 & 1296.994 & 500 Hz  & Fyrar        & Exklusivt segment för fyrar                                   \\ \hline
1296.994 & 1297.481 & 20 kHz  & FM           & Repeater ut Repeater ut \numrange{1297,000}{1297,475}         \\
	 &          &         &              & RM0--RM19, \SI{25}{\kilo\hertz}, \SI{6}{\mega\hertz} skift    \\ \hline
1297.494 & 1297.981 & 20 kHz  & FM simplex   & Simplex \SI{25}{\kilo\hertz} kanaler SM20--SM39               \\
	 &          &         &              & \textbf{FM anrop \num{1297,500} SM20}                         \\ \hline
1299.000 & 1299.750 & 150 kHz & Alla moder   & 5 st \SI{150}{\kilo\hertz} kanaler för DD,                    \\
	 &          &         &              & 1299,075 1296,225 1296,375  1296,525 1296,675                 \\
	 &          &         &              & Kanalbandbredd \SI{150}{\kilo\hertz} dvs center $\pm$75\,kHz  \\ \hline
1299.750 & 1300.000 & 20 kHz  & Alla moder   & 8 st FM/DV \SI{2}{\kilo\hertz} kan. \numrange{1299,775}{1299.975}
\end{tabular}

\clearpage

\twocolumn
