\onecolumn

\chapter{Bandplaner}
\label{bandplaner2}
\index{bandplan}

\noindent Dessa bandplaner presenteras efter svenska förhållanden så långt de är kända vid tillfället. Kontrollera alltid de senaste bandplanerna på SSA för att säkerställa du använder de senaste.

\section{Bandplan HF}
\index{bandplan!hf}

Alla frekvenser i kHz, bandbredder i Hz.

\subsubsection{Bandplan 2.2\,km, 135,7--137,8\,kHz}
\begin{tabular}{rrrll}
\textbf{Frekvens} &  & \textbf{BW} & \textbf{Trafik} & \textbf{Noteringar} \\ \hline
135,7 & 135,8 & 200 & CQ, QRSS, Digi & OBS! Högsta effekt 1W ERP. \\ \hline
\end{tabular}

\subsubsection{Bandplan 600\,m, 472--479\,kHz}
\begin{tabular}{rrrll}
\multicolumn{2}{c}{\textbf{Frekvens}} & \textbf{BW} & \textbf{Trafik} & \textbf{Noteringar} \\ \hline
472 & 479 & 200 & CW, QRSS, Digi & OBS! Högsta utstrålad effekt 1W EIRP \\ \hline
\end{tabular}

\subsubsection{Bandplan 160\,m, 1810--2000\,kHz}
\begin{tabular}{rrrll}
\multicolumn{2}{c}{\textbf{Frekvens}} & \textbf{BW} & \textbf{Trafik} & \textbf{Noteringar} \\ \hline
1810 & 1838 & 200  & CW         & Exklusivt för CW. Interkontinental trafik har prio. \\ \hline
1838 & 1840 & 500  & Smalband   & Ej packet på 160m, PSK 1838,150                    \\ \hline
1840 & 1850 & 2700 & Alla moder & Även digimode. SSB QRP 1843 kHz                    \\ \hline
1850 & 1900 & 2700 & Alla moder & OBS! Max 10 W till ant.                             \\ \hline
1900 & 1950 & 2700 & Alla moder & OBS! Max 100 W till ant.                            \\ \hline
1950 & 2000 & 2700 & Alla moder & OBS! Max 10 W till ant.                             \\ \hline
\end{tabular}

\subsubsection{Bandplan 80\,m, 3500--3800\,kHz}
\begin{tabular}{rrrll}
\multicolumn{2}{c}{\textbf{Frekvens}} & \textbf{BW} & \textbf{Trafik} & \textbf{Noteringar} \\ \hline
3500 & 3510 & 200  & CW             & Exklusivt CW                         \\ 
      &       &      &                & Interkontinental DX-trafik har prio  \\ \hline
3510 & 3580 & 200  & CW             & Exklusivt CW contest 3510-–560       \\ 
      &       &      &                & CW QRS 3 555 kHz, CW QRP 3 560       \\ \hline
3580 & 3600 & 500  & Smalband, Digi & PSK 3580,150                        \\
      &       &      &                & Automatiska Digimoder 3590--600     \\ \hline
3600 & 3620 & 2700 & Alla moder     & Digimoder Automatiska Digimoder      \\ \hline
3600 & 3650 & 2700 & Alla moder     & SSB contest 3600--650               \\
      &       &      &                & DV 3630                             \\ \hline
3650 & 3700 & 2700 & Alla moder     & SSB QRP 3690                        \\ \hline
3700 & 3800 & 2700 & Alla moder     & Contest 3700-–800                   \\
      &       &      &                & Image 3775                          \\
      &       &      &                & Region 1 nödfrekvens 3760           \\ \hline
3775 & 3800 & 2700 & Alla moder     & Interkontinental DX-trafik prioritet \\ \hline
\end{tabular}

\subsubsection{Bandplan 40\,m, 7000--7200\,kHz}
\begin{tabular}{rrrll}
\multicolumn{2}{c}{\textbf{Frekvens}} & \textbf{BW} & \textbf{Trafik} & \textbf{Noteringar} \\ \hline
7000 & 7040 & 200  & CW         & Exklusivt CW.                             \\
     &      &      &            & QRP aktivitetscentrum 7030\,kHz           \\ \hline
7040 & 7050 & 500  & Smalband   & Digimoder Automatiska inom 7047–-050\,kHz \\ \hline
7050 & 7060 & 2700 & Alla moder & Digimoder Automatiska inom 7050–-053\,kHz \\ \hline
7060 & 7100 & 2700 & Alla moder & SSB contest i segmentet                   \\
     &      &      &            & DV 7 070 kHz, SSB QRP 7090 kHz            \\ \hline
7100 & 7130 & 2700 & Alla moder & Region 1 nödfrekvens 7110 kHz             \\ \hline
7130 & 7200 & 2700 & Alla moder & SSB contest i segmentet                   \\
     &      &      &            & Image 7165\,kHz                           \\ \hline
7175 & 7200 & 2700 & Alla moder & Interkontinental DX-trafik prio           \\ \hline
\end{tabular}

\subsubsection{Bandplan 30 m, 10\,100--10\,150 kHz}
\begin{tabular}{rrrll}
\multicolumn{2}{c}{\textbf{Frekvens}} & \textbf{BW} & \textbf{Trafik} & \textbf{Noteringar} \\ \hline
10\,100 & 10\,140 & 200 & CW       & CW exkl. Max 150 Watt på 30 m           \\
        &         &     &          & CW QRP 10\,116\,kHz                     \\ \hline
10\,140 & 10\,150 & 500 & Smalband & Digimoder PSK 10142,150\,kHz. Ej Packet \\ \hline
\end{tabular}

\subsubsection{Bandplan 20 m, 14\,000--14\,350 kHz}
\begin{tabular}{rrrll}
\multicolumn{2}{c}{\textbf{Frekvens}} & \textbf{BW} & \textbf{Trafik} & \textbf{Noteringar} \\ \hline
14\,000 & 14\,070 & 200  & CW         & Exklusivt CW                             \\
        &         &      &            & Conctest 14\,000-–060                    \\
        &         &      &            & CW QRS 14 055, CW QRP 14\,060            \\ \hline
14\,070 & 14\,099 & 500  & Smalband   & PSK 14 070,150                           \\
        &         &      &            & Auto Digimoder 14 089-–099               \\ \hline
14\,099 & 14\,101 & 200  & Fyrar      & Exklusivt IBP, endast fyrar              \\ \hline
14\,101 & 14 \,12 & 2700 & Alla moder & Digitala moder och obevakade Digimoder   \\ \hline
14\,112 & 14\,350 & 2700 & Alla moder & SSB Contest 14 125--300                  \\
        &         &      &            & DV 14 130, DXpedition prio 14\,195$\pm$5 \\ \hline
14\,300 & 14\,350 & 2700 & Alla moder & Image 14\,230, SSB QRP 14\,285           \\
        &         &      &            & Global nödfrekvens 14 300                \\ \hline
\end{tabular}

\subsubsection{Bandplan 17 m, 18\,068--18\,168 kHz}
\begin{tabular}{rrrll}
\multicolumn{2}{c}{\textbf{Frekvens}} & \textbf{BW} & \textbf{Trafik} & \textbf{Noteringar} \\ \hline

18\,068 & 18\,095 & 200  & CW         & CW exklusivt. QRP 18\,086             \\ \hline
18\,095 & 18\,109 & 500  & Smalband   & Digimoder PSK 18\,100,150             \\
        &         &      &            & Automatiska Digimoder 18\,105-–109 \\ \hline
18\,109 & 18\,111 & 200  & Fyrar      & Exklusivt fyrar, IBP fyrnät           \\ \hline
18\,111 & 18\,168 & 2700 & Alla moder & Digi 18\,111–-120                   \\
        &         &      &            & SSB QRP 18\,130, DV 18\,150           \\
        &         &      &            & Global nödfrekv. 18\,160              \\ \hline
\end{tabular}

\subsubsection{Bandplan 15 m, 21\,000--21\,450 kHz}
\begin{tabular}{rrrll}
\multicolumn{2}{c}{\textbf{Frekvens}} & \textbf{BW} & \textbf{Trafik} & \textbf{Noteringar} \\ \hline

21\,000 & 21\,070 & 200  & CW         & Exklusivt CW, QRS 21\,055, CW QRP 21\,060           \\ \hline
21\,070 & 21\,110 & 500  & Smalband   & PSK 21\,080.150, Automatiska Digimoder 21\,090–-110 \\
21\,110 & 21\,120 & 2700 & Alla moder & Alla moder utom SSB!                                \\
        &         &      &            & Digimoder, och Automatiska Digimoder                \\ \hline
21\,120 & 21\,149 & 500  & Smalband   &                                                     \\ \hline
21\,149 & 21\,151 & 200  & Fyrar      & Exklusivt fyrar. IBP fyrnät                         \\ \hline
21\,151 & 21\,450 & 2700 & Alla moder & DV 21\,180, SSB QRP 21\,285, Image 21\,340          \\
        &         &      &            & Global nödfrekv. 21\,360                            \\ \hline
\end{tabular}

\subsubsection{Bandplan 12 m, 24\,890--24\,990 kHz}
\begin{tabular}{rrrll}
\multicolumn{2}{c}{\textbf{Frekvens}} & \textbf{BW} & \textbf{Trafik} & \textbf{Noteringar} \\ \hline
24\,890 & 24\,915 & 200  & CW         & Exklusivt CW, QRP 24\,906                             \\ \hline
24\,915 & 24\,929 & 500  & Smalband   & PSK 24\,920,150, Automatiska Digimoder 24\,925–-929 \\ \hline
24\,929 & 24\,931 & 200  & Fyrar      & Fyrar, IBP fyrnät                                    \\ \hline
24\,931 & 24\,990 & 2700 & Alla moder & Auto Digimoder 24\,931-–940                        \\
       &        &      &            & SSB QRP 24\,950, DV 24,960                            \\ \hline
\end{tabular}

\subsubsection{Bandplan 10 m, 28\,000-29\,700 kHz}
\begin{tabular}{rrrll}
\multicolumn{2}{c}{\textbf{Frekvens}} & \textbf{BW} & \textbf{Trafik} & \textbf{Noteringar} \\ \hline
28\,000 & 28\,070 & 200  & CW         & Exklusivt CW, QRS 28\,055, CW QRP 28\,060              \\ \hline
28\,070 & 28\,190 & 500  & Smalband   & PSK 28 120,150, Auto Digimoder inom 28\,120--150       \\ \hline
28\,190 & 28\,199 & 200  & Fyrar IBP  & Regionala fyrar med tidsdelning                        \\ \hline
28\,199 & 28\,201 & 200  & Fyrar IBP  & IBP fyrnät                                             \\ \hline
28\,201 & 28\,225 & 200  & Fyrar IBP  & kontinuerligt sändande fyrar                           \\ \hline
28\,225 & 28\,300 & 2700 & Alla moder & Övriga fyrar                                           \\ \hline
28\,300 & 28\,320 & 2700 & Alla moder & Digimoder och Automatiska Digimoder                    \\ \hline
28\,320 & 29\,100 & 2700 & Alla moder & DV 28\,330 kHz, SSB QRP 28\,360 kHz                    \\
        &         &      &            & Image 28\,680 kHz                                      \\ \hline
29\,100 & 29\,200 & 6000 & Alla moder & FM simplex, 10\,kHz kanaler                            \\
        &         &      &            & Maximalt $\pm$2,5 kHz dev., max 2,5\,kHz mod.frek.         \\ \hline
29\,200 & 29\,300 & 6000 & Alla moder & Digimoder och Automatiska Digimoder                    \\ \hline
29\,300 & 29\,510 & 6000 & Satellit   & Nerlänk fr. satellit. EJ SÄNDNING I SEGMENTET          \\ \hline
29\,510 & 29\,520 & 6000 & Skydd      & Skyddsfrekvens för satelliter. EJ SÄNDNING I SEGMENTET \\ \hline
29\,520 & 29\,590 & 6000 & Alla moder & FM Repeater in RH1--8, 100\,kHz duplex, 2.5\,kHz NBFM  \\ \hline
29\,600 & 29\,620 & 6000 & Alla moder & FM simplex, anrop 29\,600                              \\
        &         &      &            & FM simplex repeater 29\,610                            \\ \hline
29\,620 & 29\,700 & 6000 & Alla moder & FM Repeater ut RH1--8, 100\,kHz duplex                 \\ \hline
\end{tabular}
\newpage

\section{Bandplan VHF--UHF}
\index{bandplan!vhf}
\index{bandplan!uhf}

\begin{tabular}{rrrll}
	
	\textbf{Frekvens} &  & \textbf{BW} & \textbf{Trafik} & \textbf{Noteringar} \\ \hline
	
	50.000 & 50.100 & 500 Hz  & CW          & \textbf{CW anrp. 50.050 och 50.090 (interkont.)}             \\ \hline
	50.100 & 50.130 & 2.7 kHz & CW, SSB     & Interkontinental DX-trafik. Ej QSO inom Europa               \\ \hline
	50.100 & 50.200 & 2.7 kHz & CW,SSB      & DX 50.110--50.130, \textbf{50.110 50.150 anrop (interkont.)} \\ \hline
	50.200 & 50.300 & 2.7 kHz & CW,SSB      & Generell användning. 50.285 för crossband                    \\ \hline
	50.300 & 50.400 & 2.7 kHz & CW, MGM     & PSK 50.305, EME 50.310 – 50.320                              \\
	&        &         &             & MS 50.350 – 50.380                                           \\ \hline
	50.400 & 50.500 & 1 kHz   & CW, MGM     & Endast fyrar, 50.401 $\pm$500 Hz WSPR-fyrar                      \\ \hline
	51.210 & 51.390 & 12 kHz  & FM          & Repeater Repeater in, 20/10 kHz kanalavstånd                 \\
	&        &         &             & RF81 – RF99                                                  \\ \hline
	50.500 & 52.000 & 12 kHz  & Alla moder  & SSTV 50.510, RTTY 50.600, FM 51.510                          \\ \hline
	51.810 & 51.990 & 12 kHz  & FM Repeater & Repeater ut, 20/10 kHz kanalavstånd                          \\
	&        &         &             & RF81 – RF99                                                  \\ \hline
\end{tabular}

\subsubsection{Bandplan 2m 144--146 MHz}
\begin{tabular}{rrrll}
	
	\textbf{Frekvens} &  & \textbf{BW} & \textbf{Trafik} & \textbf{Noteringar} \\ \hline
	
	144.0000 & 144.1100  & 500 Hz  & CW, EME      & \textbf{CW anrop 144.050}               \\
	&           &         &              & MS random 144.100                       \\ \hline
	144.1100 & 144.1500  & 500 Hz  & CW, MGM      & EME MGM 144.120--144.160                \\
	&           &         &              & PSK31 cent. 144.138                     \\ \hline
	144.1500 & 144.1800  & 2.7 kHz & CW, SSB, MGM & EME 144.150--144.160                    \\
	&           &         &              & MGM 144.160--144.180 anrop 144.170      \\ \hline
	144.1800 & 144.3600  & 2.7 kHz & CW, SSB, MGM & MS SSB random 144.195--144.205          \\
	&           &         &              & \textbf{SSB anrop 144.300}              \\ \hline
	144.3600 & 144.3990  & 2.7 kHz & CW, SSB, MGM & MS MGM random anrop 144.370             \\ \hline
	144.4000 & 144.4900  & 500 Hz  & Fyr          & Exklusivt segment fyrar, ej QSO         \\ \hline
	144.5000 & 144.7940  & 20 kHz  & All mode     & SSTV, RTTY, FAX, ATV                    \\
	&           &         &              & Linjära transpondrar                    \\ \hline
	144.7940 & 144.9625  & 12 kHz  & MGM          & APRS 144.800                            \\ \hline
	144.9750 & 145.19350 & 12 kHz  & FM, DV       & Rpt in 144.975--145.1935                \\
	&           &         &              & RV46–-RV63, 12.5 kHz, 600 kHz skift     \\ \hline
	145.1940 & 145.2060  & 12 kHz  & FM rymd      & 145.200 för kom. m. bem. rymdfark.      \\ \hline
	145.2060 & 145.5625  & 12 kHz  & FM, DV       & FM 145.2125-–145.5875  V17–V47          \\
	&           &         &              & \textbf{FM anrop 145.500}, RTTY 145.300 \\
	&           &         &              & FM simpl. INET GW 145.2375, 2875, 3375  \\
	&           &         &              & DV anrop 145.375                        \\ \hline
	145.5750 & 145.7935  & 12 kHz  & FM, DV       & Rpt ut 145.575--145.7875                \\
	&           &         &              & RV46–RV63, 12.5 kHz kanalavstånd        \\ \hline
	145.794  & 145.806   & 12 kHz  & FM Rymd      & 145.800, 145.200 dplx m. bem. rymdfark. \\ \hline
	145.806  & 146.000   & 12 kHz  & All mode     & Exklusivt satellit                      \\ \hline
\end{tabular}

\subsubsection{Bandplan 70cm 432--438 MHz}
\begin{tabular}{rrrll}
	\textbf{Frekvens} &          & \textbf{BW} & \textbf{Trafik} & \textbf{Anmärkning}                               \\ \hline
	
	432.0000 & 432.0250 & 500 Hz  & CW           & EME exklusivt.                                    \\ \hline
	432.0250 & 432.1000 & 500 Hz  & CW, PSK31    & CW mellan 432.000--085, \textbf{CW anrop 432.050} \\
	&          &         &              & PSK31 432.088                                     \\ \hline
	432.1000 & 432.3990 & 2.7 kHz & CW, SSB, MGM & \textbf{SSB anrop 432.200}                        \\
	&          &         &              & Mikrovåg talkback 432.350, FSK441 432.370         \\ \hline
	432.4000 & 432.4900 & 500 Hz  & Fyr          & Exklusivt segment för fyrar                       \\ \hline
	432.5000 & 432.5940 & 12 kHz  & All mode     & Linjära transpondrar IN 432.500--600              \\ \hline
	432.5000 & 432.5750 & 12 kHz  & All mode     & NRAU Digital rep. in 432.500--575 2 MHz skift     \\ \hline
	432.5940 & 432.9940 & 12 kHz  & All mode     & Linjära transpondrar ut 432.600--800              \\ \hline
	432.5940 & 432.9940 & 12 kHz  & FM           & Rep. in 432.600--975 RU368--398 2 MHz skift       \\ \hline
	432.9940 & 433.3810 & 12 kHz  & FM           & Rep. in 433.000--375 RU368--398 1.6 MHz skift     \\ \hline
	433.3940 & 433.5810 & 12 kHz  & FM           & SSTV (FM/AFSK) 433.400                            \\
	&          &         &              & FM simplex U272--286 \textbf{anrop 433.500}       \\ \hline
	433.6000 & 434.0000 & 20 kHz  & All mode     & RTTY (FM/AFSK) 433.600                            \\
	&          &         &              & FAX 433.700, APRS 433.800                         \\ \hline
	434.0000 & 434.4940 & 20 kHz  & All mode     & NRAU Dig. kanaler 433.450, 434.475                \\ \hline
	434.5000 & 434.5940 & 20 kHz  & All mode     & NRAU Dig. rep. ut 434.500--575, 2 MHz skift       \\ \hline
	434.5940 & 434.9810 & 12 kHz  & FM           & NRAU Rep. ut 434.600--975 RU 368--RU398           \\
	&          &         &              & 12,5 kHz med 2 MHz skift                          \\ \hline
	435.000  & 438.000  & 20 kHz  & All mode     & Exklusivt satellit\\
\end{tabular}

\subsubsection{Bandplan 23cm 1240--1300 MHz}
\begin{tabular}{rrrll}
	\textbf{Frekvens}         &               & \textbf{BW} & \textbf{Trafik} & \textbf{Anmärkning}                                          \\ \hline
	1240.000         & 1243.250      & 20 kHz      & Alla moder      & 1240.000 - 1241.000 Digital kommunikation                    \\ \hline
	1243.250         & 1260.000      & 20 kHz      & ATV och Data    & Repeater ut 1258.150-1259.350, R20--68                       \\ \hline
	1260.000         & 1270.000      & 12 kHz      & Satellit        & Endast för satelliter alla moder                             \\ \hline
	1270.000         & 1272.000      & 20 kHz      & Alla moder      & Repeater in, 1270.025-1270.700, RS1--28                      \\
	&               &             &                 & Packet RS29--50                                              \\ \hline
	1272.000         & 1290.994      & 20 kHz      & ATV och Data    & Amatörtelevision ATV                                         \\ \hline
	1290.994         & 1291.481      & 20 kHz      & FM och DV       & Repeater in Repeat. in 1291.000--1291.475                    \\
	&               &             &                 & RM0 – RM19, 25 kHz, 6 MHz skift                              \\ \hline
	1291.494         & 1296.000      & 12 kHz      & Alla moder      &                                                              \\ \hline
	1296.000         & 1296.150      & 500 Hz      & CW,  MGM        & EME 1296.000--025, \textbf{CW anrop 1296.050}                \\
	&               &             &                 & PSK31 1296.138 MHz                                           \\ \hline
	1296.150         & 1296.400      & 2.7 kHz     & CW, SSB, MGM    & \textbf{SSB anrop 1296.200}                                  \\
	&               &             &                 & \textbf{FSK441 MS anrop 1296.370}                            \\ \hline
	1296.400         & 1296.600      & 2.7 kHz     & CW, SSB, MGM    & Linjära transpondrar infrekvens                              \\ \hline
	1296.600         & 1296.800      & 2.7 kHz     & CW, SSB, MGM    & SSTV/FAX 1296.500, MGM/RTTY 1296.600                         \\ \hline
	1296.600         & 1296.800      & 2.7 kHz     & CW, SSB, MGM    & Linjära transpondrar utfrekvens                              \\
	&               &             &                 & 1296.750-.800 lokala fyrar max 10 W                          \\ \hline
	1296.800         & 1296.994      & 500 Hz      & Fyrar           & Exklusivt segment för fyrar                                  \\ \hline
	1296.994         & 1297.481      & 20 kHz      & FM              & Repeater ut Repeater ut 1297.000--1297.475                   \\
	&               &             &                 & RM0 – RM19, 25 kHz, 6 MHz skift                              \\ \hline
	1297.494         & 1297.981      & 20 kHz      & FM simplex      & Simplex 25 kHz kanaler SM20--39                              \\
	&               &             &                 & \textbf{FM anrop 1297.500 SM20}                              \\ \hline
	1299.000         & 1299.750      & 150 kHz     & Alla moder      & 5 st 150 kHz kanaler för DD,                                 \\
	&               &             &                 & 1299.075, 225, 375, 525, och 675 $\pm$75 kHz                 \\ \hline
	1299.750         & 1300.000      & 20 kHz      & Alla moder      & 8 st FM/DV 25 kHz kan. 1299.775--1299.975
\end{tabular}

\clearpage



\twocolumn
