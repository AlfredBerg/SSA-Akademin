\section{Induktorn}
\harec{a}{2.3}{2.3}
\index{induktor}

\subsection{Allmänt}

När elektrisk ström flyter genom en ledare alstras ett magnetfält omkring
den. Så snart strömmens styrka eller riktning ändras uppstår en motsvarande
så kallad elektromotorisk kraft (EMK) som motverkar ändringen. Kraften finns i
magnetfältet i form av lagrad magnetisk energi.


\subsection{Självinduktion -- induktans}
\harec{a}{2.3.1}{2.3.1}
\index{induktans}
\index{självinduktion}
\index{elektromotorisk kraft (EMK)}
\index{EMK}
\index{induktor}

Magnetfältets förmåga att alstra en motverkande EMK kallas
\emph{självinduktion} (eng. \emph{self inductance}) eller
\emph{induktans} (eng. \emph{inductance}).
Ordet induktans kommer från latinets inducere, som betyder införa.

När en ledare som ingår i en sluten krets rör sig i ett magnetfält, kommer
en ström att flyta genom ledaren på grund av den EMK (spänning) som alstras.
Varje ändring av strömmen motverkas av det magnetfält som strömmen själv
alstrar.

När det uppstår självinduktion i en ledare kallas ledaren \emph{induktor}
(eng. \emph{inductor}).
Självinduktionen är jämnt utbredd över ledarens hela längd. När ett större
induktansvärde behövs på något särskilt ställe i strömkretsen, kan ledarens
längd ökas just där och lindas upp till en spole med lämplig form.
Hela spolen kallas då för induktor.

När ett motverkande magnetiskt fält alstras omkring en ledare genom att strömmen 
i den ändras, påverkas kretsens egenskaper och därmed dess utformning på olika sätt.
Vid snabba strömändringar, exempelvis vid hög frekvens, är motverkan större än vid
långsamma ändringar.
Vid konstant likström uppstår däremot ingen motverkan -- självinduktion.

Induktansen är efter resistansen och kapacitansen den vanligaste egenskapen i
en strömkrets.

\subsection{Försök med induktion}

\largefig{images/cropped_pdfs/bild_2_2-03.pdf}{Försök med induktion}{fig:BildII2-3}

\paragraph{Försök 1:}
Överst i bild \ref{fig:BildII2-3} är ett känsligt vridspoleinstrument kopplat
till en induktor.
Instrumentet bör ha noll på skalans mitt, så att strömriktningen syns.
En permanentmagnet används för att visa att självinduktion uppstår när
magneten förs fram och tillbaka genom induktorn.

Instrumentet ger utslag när magneten är i rörelse. Utslaget blir större vid
snabbare hastighetsändring. Utslagsriktningen växlar när magneten förs in i
respektive dras ut ur induktorn -- det uppstår en växelström.

En växelspänning uppstår över induktorn även när den ingår i en strömkrets som
sluts och bryts -- alltså utan en magnet som rör sig.

\paragraph{Försök 2:}
I mitten i bild \ref{fig:BildII2-3} har permanentmagneten bytts mot ännu en
induktor.
Utöver den första induktorn, som vi nu kallar sekundärlindning, kallar vi den
nya induktorn för primärlindning.

När vi släpper ström genom primärlindningen alstrar den ett magnetfält.
Först är strömmen noll för att sedan ändras till ett högt värde och därefter
återgå till noll. Det blir en strömstöt.

Varje ändring alstrar en mot-EMK, som bygger upp ett magnetfält, först i en
riktning och sedan i den andra. I båda fallen passerar fältet genom båda
lindningarna. Fältet från primärlindningen inducerar en spänningsstöt i
sekundärlindningen. Stöten har en riktning när primärlindningens strömkrets
sluts och motsatt riktning när den bryts -- en växelspänning alstras.
När sekundärlindningen ingår i en sluten krets uppstår en växelström genom
sekundärlindningen.

\paragraph{Försök 3:}
Nederst i bild \ref{fig:BildII2-3} ställer vi oss frågan vad som händer när
primärlindningen i försök 2 ansluts till en växelspänning, till exempel
med nätfrekvensen \SI{50}{\hertz}.
Använd för säkerhets skull en skyddstransformator mellan nätet och lindningen!

I sekundärlindningen uppstår då spänningsstötar vars polaritet i detta fall
växlar 100 gånger per sekund. Det uppstår alltså en växelspänning över
sekundärlindningen och om denna ingår i en sluten strömkrets uppstår det en
motsvarande växelström.

\mediumfig{images/cropped_pdfs/bild_2_2-04.pdf}{Schemasymboler för induktorer}{fig:BildII2-4}

\subsection{Olika utföranden}

Bild \ref{fig:BildII2-4} visar schemasymboler för ett antal vanliga induktorer.
Utöver dessa finns elektromagneter, drosslar, induktorer för resonanskretsar,
ramantenner och så vidare.

\subsection{Enheten henry (H)}
\harec{a}{2.3.2}{2.3.2}
\index{henry (H)}
\index{enheter!henry (H)}
\index{symbol!\(L\) induktans}

Måttenheten för självinduktion är \emph{henry (H)}.
1~henry (\SI{1}{\henry}) är självinduktionen i en induktor som alstrar en
motspänning av 1~volt vid en strömändring av 1~ampere under 1~sekund.
I formler betecknas induktans med symbolen L.
Sambandet är:
%%
\[\textit{volt} = \textit{henry} \cdot \textit{ampere}/\textit{sekund}\]
%%
\SI{1}{\henry} är en stor måttenhet.
För elektroniktillämpningar används därför ett mer hanterligt format.

\noindent\textbf{Exempel:}

\begin{center}
\begin{tabular}{ll}
\SI{1}{\henry} & = \SI{1000}{\milli\henry} \\
\SI{1}{\milli\henry} & = \(1 \cdot 10^{-3}\)\,H \\
\SI{1}{\milli\henry} & = \SI{1000}{\micro\henry} \\
\SI{1}{\micro\henry} & = \(1 \cdot 10^{-3}\)\,mH = \(1 \cdot 10^{-6}\)\,H
\end{tabular}
\end{center}

\subsection{Hur induktansen påverkas}
\harec{a}{2.3.3}{2.3.3}
\index{permeabilitet}
\index{relativa permeabiliteten}
\index{symbol!\(\mu_0\) permeabilitetskonstanten}
\index{symbol!\(\mu_r\) relativa permeabiliten}

Induktansen beror på induktorns mekaniska dimensioner, antalet lindningsvarv och
materialet i kärnan.

Induktansen i en cylindrisk induktor är proportionell mot tvärsnittsytan, omvänt
proportionell mot längden och proportionell mot kvadraten på lindningsvarvtalet.

Induktansen ökar om induktorn förses med en kärna av järn och minskar med en
kärna av omagnetisk, ledande metall, till exempel koppar, mässing eller aluminium.

Precis som för kondensatorn har materialet i en induktors kärna betydelse,
då dess \emph{permeabilitet} kan anta olika värden. Den absoluta permeabiliteten
\(\mu\) brukar delas upp i permeabiliteten för vakuum \(\mu_0\) och den
\emph{relativa permeabiliteten} \(\mu_r\) som gives av
%%
\[\mu = \mu_0\mu_r\]
%%
Den relativa permeabiliteten går att hitta i tabeller och varierar med material.
Permeabiliteten för vakuum är definierad som
%%
\[\mu_0 = 4\pi 10^{-7} \approx 1,256637 \cdot 10^{-6}\]

\subsection{Induktiv reaktans}
\harec{a}{2.3.4}{2.3.4}
\index{induktiv reaktans}
\index{reaktans!induktiv}
\index{symbol!\(X_L\) induktiv reaktans}

Till skillnad från när en resistor ansluts till en spänning, så blir
strömökningen i en induktor fördröjd. Orsaken är att en induktor inte bara har
en resistans, vilken ju inte påverkas av strömvariationer, utan har även en
\emph{induktiv reaktans} (eng. \emph{inductive reactance}) \(X_L\).
Ordet reaktans kommer från latinets re (åter) agere (verka).

\emph{Reaktans} -- växelströmsmotstånd eller skenbart motstånd -- uppträder så
länge som strömmen genom induktorn ändras. En induktor gör således också
motstånd mot varje strömändring och detta motstånd ökar med ökande
ändringshastighet.

En fullbordad pendling i en växelström kan ses som ett varv i en cirkel --
\ang{360} -- och en fullbordad sådan pendling kallas en period.

En period motsvarar omkretsen i en cirkel med radien r, där omkretsen är
\(2 \cdot \pi  \cdot r\). När strömmen växlar 1 gång per sekund har
pendlingen en frekvens [f] av 1~hertz [Hz].
Vid 50 växlingar per sekund har pendlingen en frekvens av \SI{50}{\hertz}.

Den \emph{Induktiva reaktansen \(X_L\)} -- växelströmsmotståndet i en induktor -- 
är en funktion av strömmens så kallade vinkelhastighet \(\omega = 2 \cdot \pi  \cdot f\)
och av induktansen L.

Den induktiva reaktansen är proportionell mot strömmens frekvens och mot
induktorns induktansvärde. Inga förluster uppstår i en ideal induktor, det vill säga 
en induktor som teoretiskt saknar resistans.
Sambandet är:
%%
\[X_L = 2\pi fL = \omega L\]
\[X_L [\Omega]\quad f [Hz]\quad L [H]\]
%%
\textbf{Exempel:}
%%
\[L = 1\ H\quad f = 50\ Hz\quad X_L = ?\]
\[X_L = 2\pi fL = 2\pi \cdot 50 \cdot 1 = 314\ \Omega\]
%%
\textbf{Exempel:}
%%
\[L = 1\ H\quad f = 5\ kHz\quad X_L = ?\]
\[X_L = 2\pi fL = 2\pi  \cdot 5 \cdot 10^3 \cdot 1 = 31\, 400\ \Omega\]

\subsection{Fasförskjutning mellan spänning och ström i en induktor}
\harec{a}{2.3.5}{2.3.5}
\index{induktor!fasförskjutning}

Med fasförskjutning menas den tidsmässiga förskjutningen mellan ström- och
spänningsförlopp. Strömmen genom en induktor når inte sitt toppvärde samtidigt
som spänningen över den. Orsaken är växlingarna mellan elektrisk och magnetisk
energi i induktorn.
Detta illustreras i bild \ref{fig:BildII3-11}.

I en ideal induktor är spänningen fasförskjuten \ang{90} före strömmen.
I praktiken är dock förskjutningen något mindre än \ang{90} på grund av
resistansen i induktorn.

\subsection{Q-faktor -- godhetstal}
\harec{a}{2.3.6}{2.3.6}
\index{Q-faktor!induktor}

\emph{Q-faktorn} kan avse två olika saker, som inte ska förväxlas.
Dessa är Q-faktorn för en komponent respektive Q-faktorn för en hel strömkrets.

Q-faktorn för en induktor är kvoten av dess reaktans och dess serieresistans.

\[Q_\text{komponent} = \dfrac{X_\text{komponent}}{R_\text{komponent}}\]

Q-faktorn för en hel resonanskrets beror däremot på bredden på det
frekvensband som en viss komponentkombination ger.
Q-faktorn för en resonanskrets är därför ett mått på dess
selektivitet (se kapitel \ref{Q-faktor}).

Q-faktorn för en ingående komponent påverkar Q-faktorn för en hel krets.
Däremot gäller inte det omvända.

\subsection{Yteffekt -- skin-effect}
\index{yteffekt}
\index{skin-effect}

I en ledare av homogent material fördelar sig en likström lika över hela
tvärsnittet. Men för en växelström minskar strömtätheten i ledarens mitt
och ökar i stället vid ytan.
Ju högre frekvensen är desto större är strömtätheten vid ytan.
Fenomenet kallas \emph{yteffekt} (eng. \emph{skin effect}) och uppträder i alla
ledare.

Det djup i ledarmaterialet där laddningstätheten sjunkit till \SI{37}{\percent}
av värdet vid ytan kallas \emph{skin depth}.
För koppar är detta djup ca \SI{70}{\milli\metre} vid \SI{100}{\hertz}.
Vid \SI{1}{\mega\hertz} har djupet minskat till \SI{0,07}{\milli\metre} och vid
\SI{100}{\mega\hertz} till \SI{0,0067}{\milli\metre}.
På grund av yteffekten är alltså materialet i mitten av homogena
ledare elektriskt mindre verksamt vid höga frekvenser. Resistansen för en viss ledare 
blir alltså större för växelström än för likström.

Utöver frekvensen påverkas yteffekten av ledarmaterialets elektriska och
magnetiska ledningsförmåga. För att få låg resistans i ledare för högfrekvent
ström är det viktigt att omkretsen är stor och att materialskiktet vid ytan har
hög ledningsförmåga. Därför är induktorerna i sändarslutsteg ofta 
försilvrade och består av rör med stor diameter eller av breda band.

\subsection{Temperaturkoefficient}

Liksom med resistorer påverkas även induktansen av temperaturen. Att sambandet
mellan induktans och temperatur är viktigt förstås av att
temperaturkoefficienten i den frekvensbestämmande induktorn i en oscillatorkrets
påverkar frekvensstabiliteten.

Eftersom metallen koppar utvidgar sig vid temperaturökning och induktorns
tvärsnittsyta då blir större, är temperaturkoefficienten vanligen positiv.
Temperaturkoefficienten \(\alpha_L\) anger induktansändringen per grad
temperaturändring.

Induktansändringen blir då $\Delta L = \pm \alpha _L \cdot L_k \cdot \Delta\vartheta$
där \(L_k\) är induktansvärdet vid den lägre temperaturen (oftast
\SI{20}{\degreeCelsius}) och \(\Delta\vartheta\) är temperaturändringen i
kelvin.

Kelvin [K] är den normerade måttenheten för absolut temperatur.
En ändring med \SI{1}{\kelvin} motsvarar en ändring med \SI{1}{\degreeCelsius}.

Induktorer kan innehålla kärnor av någon metallegering vars egenskaper också är
temperaturberoende.

I praktiken kan man knappast påverka temperaturkoefficienten i en induktor.
Eftersom en resonanskrets för det mesta även innehåller kondensatorer kan
man kompensera en positiv temperaturkoefficient i induktorn med en negativ
temperaturkoefficient i en kondensator.

\subsection{Förluster i kärnmaterial}

När ett magnetiskt växelfält passerar ett kärnmaterial kommer atomerna (som
är permanentmagneter) att ständigt inta nya lägen i materialet i takt med
fältets frekvens. Då uppstår virvelströmmar, så kallade järnförluster, som dels
påverkar materialets ledningsförmåga och som dels höjer temperaturen i kärnan 
och därmed i hela induktorn.
