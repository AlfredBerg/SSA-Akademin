\section{Frekvensblandare}
\label{blandare}
\index{blandare}
\index{frekvensblandare}
\index{mixer}

\subsection{Grundprinciper}

En anordning som blandar signaler för att skapa andra kallas som namnet säger
för \emph{blandare} (eng. \emph{mixer}).
Blandare används både i mottagare och sändare och funktionsprinciperna är lika
i båda fallen.
Vad som skiljer i stort är hur de används.

Det finns många blandarkopplingar varav de vanligaste beskrivs här.
Enkla typer med vissa nackdelar ställs mot sådana som är mer
komplicerade, men har fördelar.

\largefig{images/cropped_pdfs/bild_2_3-83.pdf}{Principer för frekvensblandning}{fig:BildII3-83}

Bild \ref{fig:BildII3-83} visar principerna för frekvensblandning.
När en linjär förstärkare matas med två signaler så sammanlagras de.
Den resulterande signalen vid varje tidpunkt är den förstärkta summan av de
inmatade signalerna.

När en olinjär förstärkare matas med två signaler så blandas de med varandra.
Förutom ingångssignalerna uppträder genom blandningen ytterligare signaler på
förstärkarutgången, så kallade blandningsprodukter.

Två av blandningsprodukterna är särskilt intressanta, det är summan och
skillnaden av ingångssignalernas frekvenser.
De oönskade övriga blandningsprodukterna filtreras bort med en avstämd krets
eller ett bandpassfilter.

\subsection{Obalanserad blandare}
\index{obalanserad blandare}
\index{blandare!obalanserad}

\largefig{images/cropped_pdfs/bild_2_3-84a.pdf}{Obalanserad blandare}{fig:BildII3-84a}

Bild \ref{fig:BildII3-84a} visar en obalanserad blandare.
Vi kan övertyga oss om att de fyra blandningsprodukterna verkligen uppstår.
Först undersöker vi den enklaste blandaren, som är ett olinjärt element i form
av en diod.

Det finns ingen förstärkare i kopplingen.
Signalspänningarna adderas genom att de två transformatorernas
sekundärlindningar är seriekopplade.
Dioden ''förvränger'' kraftigt summaspänningens kurvform.
Beroende av hur dioden är polariserad (vänd i kopplingen) blir den negativa
eller den positiva halvvågen bortskuren.

Signalen på blandarens utgång, alltså efter dioden, innehåller
bland annat frekvenserna \(f_1, f_2, f_2+f_1, f_2-f_1\).
Den lägsta frekvensen \(f_1\) kan lättast påvisas genom att ansluta ett
lågpassfilter till blandarens utgång.

Resultatet kan studeras med ett oscilloskop.
Liksom på bilden ser man då att kondensatorn laddas upp till den positiva
halvvågens toppvärde och med gott närmevärde följer kurvformen på \(f_1\).

\largefig{images/cropped_pdfs/bild_2_3-84b.pdf}{Obalanserad blandare med resonator}{fig:BildII3-84b}

Bild \ref{fig:BildII3-84b} visar en obalanserad blandare med en resonator.
En resonanskrets med lämplig bandbredd och som är avstämd till
resonansfrekvensen \(f_2\) ansluts nu till blandarens utgång.
En signal med frekvensen \(f_2\) kan då urskiljas och studeras i oscilloskopet.
Resonanskretsen tillförs energi under de positiva halvvågorna.
Energin i resonanskretsen kompletterar med den negativa halvvågen, varvid en
del av kretsens energi förbrukas.
Därför har de positiva och negativa halvvågorna inte samma amplitud (toppvärde).

Det syns i oscilloskopet hur amplituden ''svävar''.
Av detta dras slutsatsen att signalen består av fler frekvenser än \(f_2\).
Signalen är sammansatt av \(f_2, f_2+f_1\) och \(f_2-f_1\).
Signalen \(f_1\) ligger utanför resonanskretsens selektiva område och blir
därför bortfiltrerad (undertryckt).
Blandningsprodukterna \(f_2 + f_1\) och \(f_2 - f_1\) har båda en mindre
amplitud än \(f_2\).

Att det finns olika grundtoner och blandningsprodukter kan bevisas med en ännu
smalare resonanskrets med variabel frekvensavstämning, se bildens nedre del.
Vi har hittills utgått från en obalanserad blandare.
Andra blandartyper som den balanserade blandaren och den dubbelbalanserade
blandaren producerar färre blandningsprodukter.

\subsubsection{Balanserad blandare}
\index{balanserad blandare}
\index{blandare!balanserad}

\largefig{images/cropped_pdfs/bild_2_3-85.pdf}{Balanserad blandare}{fig:BildII3-85}

Bild \ref{fig:BildII3-85} visar en balanserad blandare.
Den balanserade blandaren har till skillnad från den obalanserade blandaren två
dioder och HF-transformatorernas ena lindning har mittuttag.
Ingången \(E_1\) ligger på den ena transformatorns primärlindning.
Ingången \(E_2\) 1igger över de båda mittuttagen.
Utgången ligger på den andra transformatorns sekundärlindning.

Ingången \(E_1\) matas med en signal med en låg frekvens \(f\).
Eftersom en av de båda dioderna alltid spärrar, flyter ingen ström.
De streckade pilarna visar endast i vilken riktning strömmen kunde flyta, om de
spärrande dioderna vore öppna.
Men så länge som ingen signal ligger på ingång \(E_2\), uppträder ingen
signal på utgången.

Signalen på \(E_1\) avlägsnas och i stället matas ingången med en hög
frekvens \(F\).
Under den positiva halvvågen är de båda dioderna öppna och genom båda flyter
lika stor ström.
De båda transformatorernas lindningshalvor genomflyts av lika ström i motsatt
riktning och då upphäver magnetfälten i lindningshalvorna varandra och ingen
signal uppträder på utgången.

När signaler läggs båda ingångarna händer följande:

Dioderna öppnar och stänger i takt med signalen på ingång \(E_2\), med
frekvensen \(F\).
Den mycket svagare signalen på ingång \(E_1\), med frekvensen \(f\), kan
alltefter polaritet passera diod \(D_1\) eller \(D_2\).
På återvägen överlagras signalen från \(E_1\) på signalen från \(E_2\).
Strömmarna i lindningshalvorna är olika stora.
Då uppträder en signal på utgången.
Efter blandaren följer ett filter som endast släpper igenom de önskade
blandningsprodukterna \(F + f\) eller \(F - f\).

\subsubsection{Dubbelbalanserad blandare}
\index{ringblandare}
\index{dubbelbalanserad blandare}
\index{blandare!ring}
\index{blandare!dubbelbalanserad}

\largefig{images/cropped_pdfs/bild_2_3-86.pdf}{Dubbelbalanserad blandare}{fig:BildII3-86}

Bild \ref{fig:BildII3-86} visar en dubbelbalanserad blandare.

En dubbelbalanserad blandare (även kallad \emph{ringblandare}) består av fyra
dioder, som är riktade åt samma håll i en ''diodring''.
Ingången matas med en signal med en låg frekvens \(f\).
Till skillnad mot i den balanserade blandaren flyter en ström genom dioderna
\(D_1\) och \(D_4\) respektive \(D_2\) och \(D_3\), men inte genom
utgångstransformatorn.
Ingen signal finns på utgången så länge som signalen \(F\) saknas.

Signalen på \(E_1\) avlägsnas och i stället matas ingången \(E_2\) med
en hög frekvens \(F\).
Till skillnad mot i den balanserade blandaren flyter en ström genom dioderna
\(D_1\) och \(D_2\) respektive \(D_3\) och \(D_4\).
Magnetfälten i transformatorernas lindningshalvor upphäver då varandra.
Ingen signal finns på utgången, så länge som signalen \(f\) saknas.

När signaler läggs på båda ingångarna händer följande:

De fyra dioderna kommer att öppna och stänga parvis.
Som i den balanserade blandaren överlagras strömmen från ingång \(E_1\) på den
ström som dioderna öppnar för.

Här utnyttjas båda halvperioderna av \(F\).
Strömmarna i lindningshalvorna blir olika stora.
På utgången uppträder då en signal.
Efter blandaren följer ett filter som släpper igenom de önskade
blandningsprodukterna.

\subsection{Jämförelse av blandare}

\largefig{images/cropped_pdfs/bild_2_3-87.pdf}{Jämförelse mellan olika blandare}{fig:BildII3-87}

Bild \ref{fig:BildII3-87} visar de tre beskrivna grundkopplingarna och de
jämförs med avseende på frekvensspektrum på utgången.

För den obalanserade blandaren uppträder summafrekvensen \(F + f\) och
skillnadsfrekvensen \(F - f\), vidare ingångsfrekvenserna \(f\) och \(F\),
deras övertoner \(2f\), \(3f\), \(4f\), \ldots respektive  \(2F\), \(3F\),
\(4F\), \ldots liksom deras blandningsprodukter \(F\pm 2f\), \(F\pm
3f\), \ldots och \(2F \pm f\), \(2F \pm 2f\), \(2F \pm 3f\) och så vidare.

För den balanserade blandaren saknas frekvensen \(F\) och dess övertoner.
Vidare bortfaller de jämna övertonerna av frekvensen \(f\).

För den dubbelbalanserade blandaren bortfaller ännu fler icke önskvärda
signaler, nämligen ingångssignalerna \(f\) och \(F\) och alla deras övertoner.
Endast blandningsprodukter av udda övertoner uppträder.

För en obalanserad blandare filtrerar resonanskretsen ut frekvenserna
\(F + f\), \(F - f\), och \(F\).
De balanserade blandarna saknar däremot frekvensen \(F\), den filtrerade
signalen innehåller endast blandningsprodukterna \(F + f\) och \(F - f\).
Om dessa båda blandningsprodukter är väl åtskilda eller resonanskretsen har en
bättre selektionsförmåga, då blir enbart summafrekvensen \(F + f\)
eller skillnadsfrekvensen \(F - f\) framfiltrerad.

Vi har visat tre typer av blandare med passiva komponenter.
Sådana innehåller olinjära dioder (germanium- eller kiseldioder).
Det finns även blandare med aktiva komponenter, det vill säga elektronrör eller
transistorer (bipolära, FET, MOSFET), men det skulle leda för långt
att gå in på alla olika lösningar.
Mer om hur frekvensblandning används för demodulering och modulering finns att
läsa i kapitel \ref{mottagare} om mottagare och i kapitel \ref{sändare} om
sändare. 

\subsection{Icke önskade övertoner och blandningsprodukter}

Varje olinjärt arbetande funktionssteg alstrar förutom nyttofrekvenser
även icke önskade signaler med andra frekvenser.
Både önskade och icke önskade signaler kan bestå av övertoner eller
blandningsprodukter (skillnads- och summatoner) eller bådadera.

Vissa av signalerna filtreras fram för att utgöra nyttosignaler.
Andra signaler filtreras bort, så att till exempel utsändning inte sker på fel
frekvenser.

\largefig{images/cropped_pdfs/bild_2_3-88.pdf}{Frekvensspektrum från en super-VFO}{fig:BildII3-88}

Bild \ref{fig:BildII3-88} visar ett frekvensspektrum från en super-VFO, som vi 
beskrivit i avsnitt \ref{superVFO}.
Vi ska nu undersöka vilka blandningsprodukter som uppstår i en sådan.
De två mest uppenbara frekvenserna är blandningsprodukterna (summan) i området
\SIrange{144}{146}{\mega\hertz} och (skillnaden) i området
\SIrange{128}{126}{\mega\hertz}.

Ut från blandaren finner vi ingångsfrekvensen \\ \SI{136}{\mega\hertz} och dess
övertoner \SI{272}{\mega\hertz}, \SI{408}{\mega\hertz} och så vidare såväl som
VFO-signalen och dess övertoner.
På bilden är VFO-frekvensen \SI{8}{\mega\hertz} och dess övertoner inritade, det
vill säga \SI{16}{\mega\hertz}, \SI{24}{\mega\hertz}, \SI{32}{\mega\hertz} och
så vidare.

Tyvärr bildar också de båda ingångssignalernas övertoner
blandningsprodukter vilket bilden visar.

Bandpassfiltret släpper igenom nyttofrekvensen, men dämpar alla övertoner och
blandningsprodukter.
Detta är enklare ju längre ifrån nyttosignalen de icke önskade signalerna
ligger.
I vårt exempel faller VFO-signalens övertoner inom bandpassfiltrets passband
på följande sätt:

\begin{align*}
  &15 \cdot 9,6   &= 144 \text{MHz} \quad \text{till} \quad 15 \cdot 9,733 &= 146 \text{MHz} \\
  &16 \cdot 9,0   &= 144 \text{MHz} \quad \text{till} \quad 16 \cdot 9,125 &= 146 \text{MHz} \\
  &17 \cdot 8,471 &= 144 \text{MHz} \quad \text{till} \quad 17 \cdot 8,588 &= 146 \text{MHz} \\
  &18 \cdot 8,0   &= 144 \text{MHz} \quad \text{till} \quad 18 \cdot 8,111 &= 146 \text{MHz} \\
\end{align*}

Eftersom det här handlar om 15:e -- 18:e övertonerna, blir amplituderna så små 
att vi kan bortse från dem.

Det är viktigt med goda filter i signalbehandlande funktionssteg.
En god regel är att på ett tidigt stadium filtrera bort oönskade
övertoner och blandningsprodukter -- helst i varje steg -- så att
onödigt komplexa signaler undviks.
Det är också viktigt med frekvensvalet, så att oönskade blandningsprodukter
kommer så långt bort från nyttofrekvensen som möjligt, liksom att endast mycket
höga övertoner med motsvarande små amplituder faller inom det nyttiga
frekvensområdet.
