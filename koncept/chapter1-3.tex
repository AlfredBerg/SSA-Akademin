\section{Elektriskt fält}
\harec{a}{1.3}{1.3}
\label{elektriskafält}
\index{elektriska fält}

\subsection{Potential}
\index{elektrisk potential}

Potentialskillnaden -- spänningen -- mellan olika laddade kroppar skapar
krafter mellan dem samt mellan dem och deras omgivning.
Detta fenomen kallas elektriskt kraftfält och är orsaken till att elektriskt
laddade kroppar kan komma i rörelse.

\subsection{Elektrisk laddning}
\index{elektrisk laddning}
\index{symbol!\(e\) elementarladdning}
\index{symbol!\(Q\) laddning}
\index{coulomb (C)}
\index{enheter!coulomb (C)}

Elektriska laddningar är grunden för elektricitetsläran.
Varje proton i atomkärnan är bärare av en positiv laddning.
Neutronerna i atomkärnan är elektriskt neutrala.
Antalet protoner i kärnan bestämmer därför kärnans totala positiva
laddning, kallat för kärnladdningstalet.
Elektronerna som kretsar omkring atomkärnan är bärare av var sin negativ
laddning.

Elementarladdningen [\emph{e}] är den laddning som finns i en elektron och har
länge ansetts vara den minsta möjliga laddningen.
Nutida elektronfysik konstaterar ännu mindre enheter, men det går vi inte in på
här.

Antalet protoner och elektroner i en atom är lika och elektronernas
negativa laddning blir då lika stor som protonernas positiva laddning.
När laddningar med olika polaritet är lika stora väger de ut varandra och blir
elektriskt neutrala till sin omgivning.

Måttenheten för elektrisk laddning är \(coulomb\ [C]\).
Laddningsmängden \(1\ coulomb\) motsvarar 6,25 triljoner (\(6,25\cdot10^{18}\))
elementarladdningar. Sambandet mellan laddning och ström är:
%%
\[Q = I \cdot t\]
%%
Laddning $[Q]$ är ström $[I]$ under tiden $[t]$:
%%
\[1\ C ~=~ 1\ A \cdot 1\ s ~=~ 1\ \textit{amperesekund}\ [1\ As]\]
\[1\ \textit{coulomb} ~=~ 1\ \textit{ampere} \cdot 1\ \textit{sekund}\]
%%

\smallfig{images/cropped_pdfs/bild_2_1-05.pdf}{Elektriska kraftfält}{fig:BildII1-5}

\clearpage % Layout

\subsection{Kraftfält omkring elektriska laddningar}

%%Bild \ref{fig:BildII1-5} visar elektriska kraftfält.

\noindent
Mellan elektriska laddningar bildas krafter (bild \ref{fig:BildII1-5}).

\begin{itemize}
  \item Varje laddning är omgiven av ett elektriskt kraftfält.
  \item Mellan positiva (+) elektriska laddningar och (--) negativa laddningar
  bildas krafter.
  \item Fältkrafternas styrka och riktning symboliseras som linjer mellan
  positiva och negativa laddningar, där styrkan är densamma utmed respektive
  linje.
\end{itemize}

%%(även 1.1) --- TODO: VA??

\begin{quote}
\emph{Kroppar med olika slags laddningar dras till varandra}

\emph{Kroppar med lika slags laddningar stöter bort varandra}

\emph{Oladdade kroppar påverkas inte och ger ingen kraftverkan.}
\end{quote}

\subsection{Elektrisk fältstyrka}
\harec{a}{1.3.1}{1.3.1}
\harec{a}{1.3.2}{1.3.2}
\index{elektrisk fältstyrka}
\index{symbol!\(E\) elektrisk fältstyrka}
\label{elektrisk_fälststyrka}

\smallfig{images/cropped_pdfs/bild_2_1-06.pdf}{Elektrisk fältstyrka}{fig:BildII1-6}

I en trådformad ledare, som det flyter likström igenom, fördelas strömmen lika
över tvärsnittet.
Om ledaren i stället är ett tunt plan, så blir strömfördelningen annorlunda.
Bild \ref{fig:BildII1-6} visar ett plan med två elektroder, som anslutits till en spänningskälla.
Utmed sträckan mellan elektroderna fördelas strömmen över planet så som
strömlinjerna på bilden.
Fördelningen beror på elektrodernas utformning och polaritet.
Strömtätheten är inte lika över hela planet, eftersom planet kan ses som många
parallellkopplade resistorer vars resistanser ökar med tilltagande
strömlinjelängd.

Strömtätheten i planet är större där resistansen mellan elektroderna är liten.
Närmast elektroderna där alla strömlinjer samlas är strömtätheten extremt hög.
Där strömtätheten är som störst finns den största potentialskillnaden
(spänningen) per längdenhet strömlinje.
Man kan mäta potentialerna i planet.
Spänningen mellan två punkter utmed en tänkt strömlinje är därvid proportionell
mot linjens längd mellan punkterna.
Halva spänningen finner man mitt emellan punkterna.

Elektriska fält är upplagrad energi.
Fältstyrkan kan bli så hög, att det blir en urladdning mellan polerna.
Koronaurladdning från ändarna av en antenn är ett annat tecken på hög
fältstyrka.
För att försvåra urladdning kan man öka elektrodytan, till exempel göra den klotformad.
Omvänt kan man medverka till urladdning genom att minska elektrodytan.
Ett exempel är åskledarens spets.

I bild \ref{fig:BildII1-6} \(U = f(l)\) visas spänningarna utmed
''mittströmslinjen'' igenom plus- och minuspolerna.
Kurvutseendet är typiskt även för omkringliggande linjer, oavsett längd.

Bilden framställer en ledare som ett idealt plan, medan den i praktiken är en
volym.
För att efterlikna en volym föreställer vi oss att bilden roterar omkring
mittströmslinjen, med fältlinjerna oförändrade.
Även om resistansen i den rotationskropp som uppstår är så hög att ingen ström
flyter, så är spänningsbilden fortfarande densamma.

Spänningsbilden gäller även för isolerande fasta material, gaser och vakuum.
Det finns alltså spänning mellan olika punkter även i ''friska luften''.
Denna spänningfältstyrka- kan mätas med särskilda instrument, så kallade
fältstyrkemätare.

Av brantheten på spänningskurvan i bilden framgår vilken delspänningen är per
dellängd av en spänningslinje.
Kvoten av delspänning och avståndet mellan mätpunkterna kallar man för
elektrisk fältstyrka.

I formler betecknas elektrisk fältstyrka med bokstaven \(E\).
Elektrisk fältstyrka mäts i volt per meter.
%%
\[
\begin{array}{ccc}
E=\dfrac{\Delta U}{\Delta l} &\quad& \dfrac{[volt]}{[meter]}
\end{array}
\]
%%
\subsection{Skärmning av elektriska fält}
\harec{a}{1.3.3}{1.3.3}
\index{elektriska fält!skärmning}
\label{elektrostatik skärmning}

I grunden finns det två slags fält, det elektriska och det magnetiska.
Dessutom finns det även elektromagnetiska fält, som är sammansatt av båda dessa.
Fält kan vara statiska eller dynamiska, varav här avses dynamiska.
Ett dynamiskt elektriskt fält genererar ett magnetiskt fält.
Omvänt generar ett dynamiskt magnetiskt fält ett dynamiskt elektriskt fält.
Denna växelverkan gör att fälten kan hållas igång av varandra med tillskott av
yttre energi.

Fält i rörelse alstrar elektromagnetisk strålning, som påverkar omgivningen.
När påverkan inte är önskvärd måste fältet skärmas av.
Ett sätt att skärma av ett elektriskt fält är en metallisk kapsling som
anslutits till apparatens jordreferens.
Skärmen behöver inte vara tät, men utförd så att all magnetiskt inducerad ström
i den bryts. (Jfr \ref{elektromagnetisk skärmning})
