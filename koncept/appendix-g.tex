\onecolumn

\section{VHF och högre}
\label{Bandplan VHF och högre}

Den vänstra delen är själva bandplanen, medan den högra delen rekommenderar
användning/mötespunkter.
(PTS bandplan och status för amatörradio i Sverige, framgår av Kapitel
\ref{bandplaner} samt bilaga \ref{svensk frekvensplan} svensk frekvensplan och
bilaga \ref{svenska repeatrar} frekvenser för svenska amatörradiorepeatrar.)

\subsection{50 MHz bandplan}
\label{50MHZbandplan}
Denna bandplan baseras på IARU Region~1 2014 \cite{IARU1}.

\begin{table}[h]
\caption{50 MHz Användning: Experimentband, landmobil- och, rundradio primära}
\begin{xtabular}{llll}
Segment & Trafiksätt & Delband & Rekommenderad användning \\
MHz     &            &         & \\ \hline
% \endhead
50,000 & CW &                 & \\
       &    & 50,000--50,030 & Fyrar\\
       &    & 50,050          & CW aktivitetscenter, internationellt\\
       &    & 50,090          & CW aktivitetscenter, interkontinentalt\\
50,100 &    &                 & \\ \hline

50,100 & Alla smalbandsmoder & &\\
       & (CW, SSB, AM, RTTY, & &\\
       & SSTV, ETC)          & &\\
       & Smalband = 2,7~kHz  & &\\
       & & 50,100--50,130 & interkontinentalt\\
       & & 50,110 & aktivitetscenter interkontinentalt\\
       & & 50,130--50,200 & internationellt\\
       & & 50,150 & aktivitetscenter internationellt\\
       & & 50,200--50,300 & Generell användning\\
       & & 50,285 & crossband\\
       & & 50,305 & aktivitetscenter PSK\\
       & & 50,310--50,320 & aktivitetscenter MS\\
       & Smalband 1000~Hz & 50,400--50,500 & exklusivt fyrar\\
       & & 50,401 & \(\pm\) 500~Hz WSPR fyrar\\
50,500 &    &                 & \\ \hline

50,500 & Alla moder & & \\
       & & 50,510 & SSTV (AFSK)\\
       & & 50,520--50,540 & FM simplex Internet gateways\\
       & & 50,550 & Bild arbetsfrekvens\\
       & & 50,600 & RTTY (FSK)\\
       & & 50,620--50,750 & Digital kommunikation\\
       & & 50,630 & DV anrop\\
51,190 &    &   & \\ \hline

51,210 & RF81 & & \\
       & \multicolumn{3}{l}{NBFM repeater infrekvenser, 20~kHz kanaldelning, 10~kHz kanalbredd} \\
51,390 & RF99 & & \\ \hline

51,410 & F41 & & \\
       & NBFM, simplex & 51,510 & NBFM anropsfrekvens\\
51,590 & F59 & & \\ \hline

51,810 & RF81 & & \\
       & \multicolumn{3}{l}{NBFM repeater utfrekvenser, 20~kHz kanaldelning, 10~kHz kanalbredd} \\
51,990 & RF99 & & \\
52,000 & & & \\
\end{xtabular}
\end{table}

\subsubsection*{Anmärkningar: sändningsslag}

\begin{itemize}
\item[a] Telegrafi är tillåtet över hela bandet, men är exklusivt i området
50,000--50,100~MHz.
\item[b] Under 50,500~MHz används inte FM.
\item[c] 50,110~MHz är interkontinental DX anropsfrekvens och bör inte användas för
trafik inom Europa.
\item[d] För kanaltrafik är kanaldelningen 20~kHz, förskjutet 10~kHz.
\end{itemize}

\subsection{144 MHz bandplan}
\label{144MHzbandplan}
Denna bandplan baseras på IARU Region~1 2016 \cite{IARU1}.

\small
\begin{table}[thp]
        \normalsize
  \caption{144 MHz Användning: Amatörradio primär}
  \begin{xtabular}{llll}
144,000 & & &\\
        & & 144,000--144,025 & satellit downlink exklusiv användning\\
        & CW(a) & 144,025--144,110 & EME\\
        & & 144,050 & CW anropsfrekvens \\
        & & 144,100 & CW MS referensfrekvens, random\\
        & & 144,110--144,160 & EME MGM\\
        & & 144,138 & aktivitetscenter PSK31\\
        & & 144,140--144,150 & CW, MGM\\
144,150 & SSB & 144,150--144,180 & CW SSB MGM\\
        & & 144,195--144,205 & SSB MS (Meteorscatter), Random\\
        & & 144,300 & SSB anropsfrekvens\\
        & & 144,370 & FSK441, Random calling\\
144,399 & & 144,390--144,399 & CW SSB MGM\\
144,400 & & &\\
        & CW MGM & & Exklusivt fyrar(b)\\
        & & 144,4920 & \(\pm\) 500~Hz WSPR beacons\\
144,491 & & &\\
144,500 & Alla moder (c) & 144,500 & Bild aktivitetscenter, SSTV FAX\\
        & & 144,525 & ATV SSB talk back center \\
        & & 144,600 & aktivitetscenter data MGM RTTY\\
144,794 & & 144,750 & ATV anropsfrekvens \\
144,9625 & & & \\
        & \multicolumn{3}{l}{Digital kommunikation (d)}\\
144,975 & RV46 & &\\
        & \multicolumn{3}{l}{NBFM repeater infrekvenser, 12,5~kHz kanalseparation, 600~kHz skift} \\
145,1875 & RV63 & & \\
145,200 & V16--V45 & 145,200 & Bemannad rymdtrafik, upplänk \\
        & 12,5 kHz NBFM & 145,375 & DV anropsfrekvens\\
        & simplex & 145,500 & (Mobil) anropsfrekvens\\
145,575 & RV46 & & \\
        & \multicolumn{3}{l}{NBFM repeater utfrekvenser, 12,5~kHz kanalseparation, 600~kHz skift} \\
145,7875 & RV63 & 145,800 & Bemannad rymdtrafik, nerlänk \\
145,794 & & & \\
        & \multicolumn{3}{l}{Satellitservice} \\
146,000 & & & \\
\end{xtabular}
\end{table}

\subsubsection*{Anmärkningar}

\noindent\textbf{Generella}

\begin{enumerate}[label=\alph*.]
\item Inga sändningar får ske på frekvenser under \SI{144,0025}{\mega\hertz}.
\item I Europa ska inga in- eller utfrekvenser för NBFM repeatrar
  förekomma inom segmentet 144,000--144,794~MHz.
\item Med undantag för satellitsegmentet tillåts inte in- eller
  utfrekvenser i 2-metersbandet för repeatrar i andra band.
\item Fyrar ska oavsett ERP ligga i fyrbandet.
\end{enumerate}

\noindent\textbf{Särskilda}

\begin{itemize}
\item[(a)] Telegrafi är tillåtet över hela bandet, exklusivt i segmentet
  144,035--144,150~MHz.
\item[(b)] Fyrar med ERP över 10~W koordineras av IARU Region~1 fyrkoordinator.
\item[(c)] Inga obemannade stationer ska användas i all mode-segmentet.
\item[(d)] Obemannade stationer tillåts endast i segmentet 144,800--144,990~MHz
förutsatt att de till fullo klarar 12,5~kHz kanaldelning.
\end{itemize}

\newpage

\subsection{432 MHz bandplan}
\label{432MHzbandplan}
Denna bandplan baseras på IARU Region~1 2014 \cite{IARU1}.

\begin{table}[thp]
\caption{432~MHz Användning: Amatörradio och radiolokalisering delat primär}
  \begin{xtabular}{llll}
432,000 &        & &\\
        & CW (a) & 432,000--432,025 & EME\\
        &        & 432,050 & CW aktivitetscenter\\
        & & 432,088 & PSK31 aktivitetscenter\\
432,150 & & & \\
        & SSB/CW & 432,200 & SSB aktivitetscenter\\
        & & 432,350 & Mikrovågor ''talk-back'' center\\
        & & 432,370 & FSK441, Random calling\\
        & CW MGM & 432,400--432,490 & Exklusivt fyrar(b)\\
432,500 & Alla moder & 432,500 & APRS\\
432,500 & RU361 & & Notera överenskommelse NRAU 2004\\
 & \multicolumn{3}{l}{NBFM repeater infrekvenser, 12,5~kHz kanalseparation, 2~MHz skift}\\
432,975 & RU399 & & \\
433,000 & RU368 & & Notera att repeater med 1,6~MHz skift ska fasas ut\\
 & \multicolumn{3}{l}{NBFM repeater infrekvenser, 12,5~kHz kanalseparation, 1,6~MHz skift}\\
433,3875 & RU399 & & \\
433,400 & & 433,400 & SSTV FM AFSK\\
        & FM/DV 12,5 kHz & 433,450 & DV anropsfrekvens\\
        & & 433,500 & (Mobil) FM anropskanal\\
433,5875 & U287 & & \\
433,600 &            & 433,600           & Data aktivitetscenter\\
        &            & 433,625--433,775 & Digital kommunikation \\
        & Alla moder & 433,700           & FAX (FM/AFSK)\\
        & & 434,000 & Digital wide band modes centerfrekvens\\
        &            & 434,450--434,575 & Digital kommunikation, ej mer än\\
434,500 &            &                   & 25~kHz kanalseparation\\
434,5125 & RU 361 & & Notera överenskommelse NRAU\\
        & \multicolumn{3}{l}{NBFM repeater utfrekvenser, 12,5~kHz kanalseparation}\\
434,9875 & RU 399 & & \\
435,000 & & & \\
        & \multicolumn{3}{l}{satellitservice} \\
438,000 & & & \\
\end{xtabular}
\end{table}

\subsubsection*{Anmärkningar}

\noindent\textbf{Generella}

\begin{enumerate}[label=\alph*.]
\item I Europa ska inga in- eller utfrekvenser för NBFM repeatrar
förekomma inom segmentet 432--432,6~MHz.
\item Fyrar ska oavsett ERP placeras i fyrbandet.
\end{enumerate}

\noindent\textbf{Särskilda}

\begin{itemize}
\item[(a)] Telegrafi är tillåtet över hela smalbandssegmentet, exklusivt i
segmentet 432,000--432,100~MHz.
\item[(b)] Fyrar med ERP över 10~W koordineras av IARU Region~1 fyrkoordinator.
\end{itemize}

\newpage

\subsection{1296 MHz bandplan}
\label{1296MHzbandplan}
Denna bandplan baseras på IARU Region~1 2014 \cite{IARU1}.

\begin{table}[h]
\caption{1296 MHz Användning: Amatörradio sekundär} 
  \begin{xtabular}{llll}
1240.000 & Alla moder & 1240,000--1241,000 & Digitala moder\\
         & & 1242,025--1242,250 & Repeater ut RS1--RS10\\
         & & 1242,275--1242,700 & Repeater ut RS11--RS28\\
         & & 1242,725--1243,250 & Digitala moder RS29--RS50\\
1243,250 & & & \\
         & Amatörtelevision & 1258,150--1259,350 & Repeater ut, R20--R68 \\
1260,000 & & & \\
         & satellitservice & & \\
1270,000 & & & \\
         & Alla moder & 1270,025--1270,700 & Repeater in, RS1--RS28\\
         &            & 1270,725--1271,250 & Packet duplex, RS29--RS50\\
1272,000 & & &\\
         & Amatörtelevision & &\\
1290,994 & FM/DV & 1291,000 & RM0--RM19\\
         & \multicolumn{3}{l}{NBFM repeater in 25~kHz kanalseparation, 6~MHz skift}\\
1291,475 & & &\\
1291,500 & & &\\
         & Alla moder & 1293,150--1294,350 & Repeater in, R20--R68\\
1296,000 & & &\\
         & CW(a) & 1296,000--1296,025 & EME\\
         & & 1296,138 & PSK31 aktivitetscentrum\\
1296,150 & & & \\
         &     & 1296,200           & Smalbands aktivitetscenter \\
         &     & 1296,400--1296,600 & Linjär transponder infrekvens \\
         & SSB & 1296,500           & Bild SSTV FAX\\
         &     & 1296,600           & Data smalband MGM RTTY\\
         &     & 1296,600--1296,800 & Linjär transponder utfrekvens\\
         &     & 1296,700--1296,800 & Lokala fyrar max 10W ERP\\
1296,800 & & & \\
         & CW MGM & & Exklusivt fyrar\\
1296,994 & & & \\
1297,000 & RMO & (används i Sverige) & \\
         & \multicolumn{3}{l}{NBFM repeater utfrekvenser, 25~kHz kanalseparation, 6~MHz skift} \\
1297,481 & RM19 & & \\
1297,500 & SM20 & 1297,500 & FM aktivitetscenter \\
         & \multicolumn{3}{l}{NBFM simplex kanaler, 25~kHz kanalseparation,}\\
1297,975 & SM39 & & \\
1298,000 & & & \\
         &            & 1298,025--1298,975 & Repeater ut, RS1--RS39\\
         & Alla moder & 1299,000--1299,750 & Digital kommunikation \\
         &            & 1298,750--1300,000 & FM/DV\\
\end{xtabular}
\end{table}

\subsubsection*{Anmärkningar}

\noindent\textbf{Särskilda}

\begin{itemize}
\item[(a)] Telegrafi är tillåtet över hela smalbandsegmentet, exklusivt
i segmentet 1296,000--1296,150~MHz.
\item[(b)] Fyrar med ERP över 10~W koordineras av IARU Region~1 fyrkoordinator.
\end{itemize}

\newpage

\subsection{2300 MHz bandplan}
\label{2300MHzbandplan}
Denna bandplan baseras på IARU Region~1 2014 \cite{IARU1}.
Delbandet 2300--2400~MHz är inte längre ett amatörband varför endast det
kvarvarande delbandet redovisas här.
 
\begin{table}[h]
  \caption{2300 Mhz Användning: Amatörradio sekundär}
  \begin{xtabular}{llll}
2400,000 & & & \\
         & \multicolumn{3}{l}{Satellitservice} \\
2450,000 & & & \\
\end{xtabular}
\end{table}

\subsection{5650 MHz bandplan}
\label{5650MHzbandplan}
Denna bandplan baseras på IARU Region~1 2014 \cite{IARU1}.

\begin{table}[h]
  \caption{5650 MHz Användning: Amatörradio sekundär} 
  \begin{xtabular}{llll}
5650,000 & & & \\
         & \multicolumn{3}{l}{Satellitservice, upplänk} \\
5670,000 & & & \\
5668,000 & & & \\
         & Smalband, CW/SSB/FM & 5668,200 & Aktivitetscenter \\
5670,000 & & & \\
         & Digital kommunikation & & \\
5700,000 & & & \\
         & Amatörtelevision & & \\
5720,000 & & & \\
         & Alla moder & & \\
5760,000 & & & \\
         & Smalband, CW/SSB/FM & 5760,200 & Aktivitetscenter \\
5762,000 & & & \\
         & Alla moder & & \\
5790,000 & & & \\
         & \multicolumn{3}{l}{Satellitservice, nerlänk} \\
5850,000 & & & \\
\end{xtabular}
\end{table}

\subsection{10 GHz bandplan}
\label{10GHzbandplan}
Denna bandplan baseras på IARU Region~1 2014 \cite{IARU1}.

\begin{table}[h]
  \caption{10000 MHz Användning: Amatörradio sekundär} 
  \begin{xtabular}{llll}
10000,000 & & & \\
          & \multicolumn{3}{l}{Digital kommunikation} \\
10150,000 & & & \\
          & \multicolumn{3}{l}{Alla moder: ATV, data, FM simplex/duplex/repeatrar} \\
10250,000 & & & \\
          & \multicolumn{3}{l}{Digital kommunikation} \\
10350,000 & & & \\
          & \multicolumn{3}{l}{Alla moder} \\
10368,000 & & & \\
          & Smalband CW/SSB/fyrar & 10368,200 & Aktivitetscenter \\
10370,000 & & & \\
          & \multicolumn{3}{l}{Alla moder} \\
10450,000 & & & \\
          & \multicolumn{3}{l}{Satellitservice} \\
10500,000 & & & \\
\end{xtabular}
\end{table}

\newpage

\subsection{24 GHz bandplan}
\label{24GHzbandplan}
Denna bandplan baseras på IARU Region~1 2014 \cite{IARU1}.

\begin{table}[h]
  \caption{24000 MHz Användning: Amatörradio sekundär} 
  \begin{xtabular}{llll}
24000,000 & & & \\
          & \multicolumn{3}{l}{Satellitservice} \\
24048,000 & & & \\
          & CW/SSB/fyrar & 24048,200 & Aktivitetscenter, smalbandsmoder \\
24050,000 & & & \\
          & Alla moder   & 24125,000 & Aktivitetscenter, bredbandiga moder \\
24250,000 & & & \\
\end{xtabular}
\end{table}

\subsection{47 GHz bandplan}
\label{47GHzbandplan}
Denna bandplan baseras på IARU Region~1 2014 \cite{IARU1}.

\begin{table}[h]
  \caption{47000 MHz Användning: Amatörradio primär} 
  \begin{xtabular}{llll}
47000,000 & & & \\
          & & 47088,200 & Aktivitetscenter, smalbandsmoder \\
47200,000 & & & \\
\end{xtabular}
\end{table}

\twocolumn
